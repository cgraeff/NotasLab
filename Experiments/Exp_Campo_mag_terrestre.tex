%%%%%%%%%%%%%%%%%%%%%%%%%%%%%%%%%%%%%%%%%%%%%%%%%%%%%%%%%%%%%%%%%%%%%%%%%%%%%%%
\chapter{Campo magnético terrestre} % Sem "Experiência 01" ou qualquer outro número
\label{Chap:CampoMagTerrestre}        % para poder trocar a ordem com facilidade
%%%%%%%%%%%%%%%%%%%%%%%%%%%%%%%%%%%%%%%%%%%%%%%%%%%%%%%%%%%%%%%%%%%%%%%%%%%%%%%

\begin{fullwidth}\it
	Que experimento faremos?
	Qual é o objetivo?
	O que veremos/revisaremos? (teoria física)
	Quais conceitos/técnicas de análise de dados utilizaremos?
\end{fullwidth}

%%%%%%%%%%%%%%%%%%%%%%%%%%%%%%%%%%%%%%%%%%%%%%%%%%%%%%%%%%%%%%%%%%%%%%%%%%%%%%%
\section{Campo magnético terrestre}
%%%%%%%%%%%%%%%%%%%%%%%%%%%%%%%%%%%%%%%%%%%%%%%%%%%%%%%%%%%%%%%%%%%%%%%%%%%%%%%

fazer uma introdução curta

%%%%%%%%%%%%%%%%%%%%%%%%%%%%%%%%%%%%%%%%%%%%%%%%
\subsection{Campo magnético devido a uma espira}
%%%%%%%%%%%%%%%%%%%%%%%%%%%%%%%%%%%%%%%%%%%%%%%%

% Determinar o campo magnético de uma espira ao longo do eixo $z$.

Podemos determinar o campo magnético de uma espira circular ao longo do eixo $z$ que passa por seu centro, perpendicularmente ao plano que contém a espira, através da lei de Biot-Savart:
\begin{equation}
    d\vec{B} = \frac{\mu_0}{4\pi} \frac{I\, d\vec{\ell} \times \hat{r}}{r^2}.
\end{equation}
%
No caso particular da espira circular, a equação acima descreve o campo $d\vec{B}$ gerado por um segmento de comprimento $d\vec{\ell}$ do condutor que porta uma corrente $I$.

Note que devido à simetria rotacional do problema, verificamos que sobre o eixo $z$ as componentes $B_x$ e $B_y$ do campo magnético são nulas, restando somente a componente $B_z$. Por essa razão, vamos nos preocupar somente com a determinação de $dB_z$ de agora em diante. Esta componente pode ser deteminada ao projetarmos o campo $dB$ na direção do eixo $z$ através de 
\begin{equation}
    dB_z = dB \sen\theta.
\end{equation}
%
Podemos determinar o módulo do campo magnético ao tomarmos a lei de Biot-Savart em módulo:
\begin{equation}
    dB = \frac{\mu_0}{4\pi} \frac{I\, |d\vec{\ell} \times \hat{r}|}{r^2}.
\end{equation}
%
Como $d\vec{\ell} \perp \hat{r}$, o módulo do produto vetorial entre esses dois vetores resulta em
\begin{equation}
    |d\vec{\ell} \times \hat{r}| = d\ell.
\end{equation}
%
Além disso, ainda que a distância $r$ entre o ponto onde estamos calculando campo e a posição do segmento $d\vec{\ell}$ pode ser escrita em termos de $z$ e do raio $R$ da espira:
\begin{equation}
    r = \sqrt{z^2 + R^2}.
\end{equation}
%
Assim,
\begin{align}
    dB_z &= dB \cdot \sin\theta \\
    &= \left(\frac{\mu_0}{4\pi} \cdot \frac{I \,d\ell}{z^2 + R^2}\right) \cdot \sen\theta \\
    &= \left(\frac{\mu_0}{4\pi} \cdot \frac{I \,d\ell}{z^2 + R^2}\right) \cdot \left(\frac{R}{\sqrt{z^2+R^2}}\right) \\
    &= \frac{\mu_0}{4\pi} \cdot \frac{IR \,d\ell}{(z^2 + R^2)^{\nicefrac{3}{2}}}.
\end{align}
%
Para determinarmos o campo total na direção do eixo $z$, basta agora realizarmos a integração ao longo do caminho formado pela espira:
\begin{align}
    B_z(z) &= \oint \frac{\mu_0}{4\pi} \cdot \frac{IR \,d\ell}{(z^2 + R^2)^{\nicefrac{3}{2}}} \\
    &= \frac{\mu_0}{4\pi} \cdot \frac{IR}{(z^2 + R^2)^{\nicefrac{3}{2}}} \oint d\ell \\
    &= \frac{\mu_0}{2} \cdot \frac{IR}{(z^2 + R^2)^{\nicefrac{3}{2}}}. \label{Eq:CampoMagneticoEspiraCircular}
\end{align}

No caso de um grupo de $N$ espiras de mesmas dimensões, próximas umas das outras, podemos escrever o campo magnético em termos da corrente $i$ de uma espira como
\begin{equation}\label{Eq:CampoMagneticoGrupoEspiras}
    B_z(z) = \frac{\mu_0}{2} \cdot \frac{iNR}{(z^2 + R^2)^{\nicefrac{3}{2}}}.
\end{equation}
    

%%%%%%%%%%%%%%%%%%%%%%%%%%%%%%%%%
\subsection{Bobinas de Helmholtz}
%%%%%%%%%%%%%%%%%%%%%%%%%%%%%%%%%

As bobinas de Helmholtz são uma configuração particular de dois grupos paralelos de espiras de mesmas dimensões, separadas por uma distância $h = R$. A corrente $i$ nas espiras e seu sentido  devem ser iguais nos dois grupos de espiras. Nesse caso, o campo na região central dos grupos de espiras é bastante uniforme, tornando esse tipo de aparato útil para diversos tipos de aplicação. 

Para determinarmos o campo na região central, basta somarmos o campo $B_1$ ---~devido ao primeiro grupo de espiras~--- e o campo $B_2$ ---~devido ao segundo grupo~---. Para que possamos fazer isso de maneira mais simples, podemos reescrever a Equação~\eqref{Eq:CampoMagneticoGrupoEspiras} como
\begin{equation}
    B_z(z) = \frac{\mu_0 N i}{2R} \cdot \xi(z),
\end{equation}
%
onde
\begin{equation}
    \xi(z) = \frac{1}{(1+(z/R)^2)^{\nicefrac{3}{2}}}.
\end{equation}

Assim, 
\begin{align}
    B_H(z) &= B_1 + B_2 \\
    &= \frac{\mu_0 N i}{2R} \cdot \xi(z - R/2) + \frac{\mu_0 N i}{2R} \cdot \xi(z + R/2) \\
    &= \frac{\mu_0 N i}{2R} \cdot (\xi(z - R/2) + \xi(z + R/2)).
\end{align}
%
Nas expressões acima, os fatores $\pm R/2$ atuam de maneira a deslocar a origem para o ponto central entre os grupos de espiras, lembrando que a expressão original ---~Equação~\eqref{Eq:CampoMagneticoEspiraCircular}~---usava como origem o centro da espira.

%%%
%%% Fazer um gráfico da expressão acima para que possamos ver a região de campo uniforme.
%%%

Expandindo o termo entre parentesis da expressão acima usando uma série de Taylor resulta em
\begin{equation}
    (\xi(z - R/2) + \xi(z + R/2)) = \frac{16\sqrt{5}}{25} - \frac{2304\sqrt{5}}{3125}\cdot \left(\frac{z}{R}\right)^4 + \mathcal{O}((z/R)^6).
\end{equation}
%
Assim, podemos escrever para $z = 0$, isto é, para a região central das espiras
\begin{equation}
    B_H = \frac{16\sqrt{5}}{25} \cdot \frac{\mu_0 N i}{2R}.
\end{equation}

%%%%%%%%%%%%%%%%%%%%%%%%%%%%%%%%%%%%%%%%%%%%%%%%%%%%%%
\subsection{Determinação do campo magnético terrestre}
%%%%%%%%%%%%%%%%%%%%%%%%%%%%%%%%%%%%%%%%%%%%%%%%%%%%%%

Determinar a relação entre o campo magnético da terra e o campo magnético da bobina

%%%%%%%%%%%%%%%%%%%%%%%%%%%%%%%%%%%%%%%%%%%%%%%%%%%%%%%%%%%%%%%%%%%%%%%%%%%%%%%
\section{Experimento}
%%%%%%%%%%%%%%%%%%%%%%%%%%%%%%%%%%%%%%%%%%%%%%%%%%%%%%%%%%%%%%%%%%%%%%%%%%%%%%%

%%%%%%%%%%%%%%%%%%%%%%
\subsection{Objetivos}
%%%%%%%%%%%%%%%%%%%%%%

\begin{itemize}
	\item Resultados concretos que devemos conseguir;
	\item Observar a relação de tal coisa com outra coisa;
	\item Calcular a constante $x$.
\end{itemize}

%%%%%%%%%%%%%%%%%%%%%%%%%%%%%%%%%%%%%%%%%%%%%%%%%%%%%%%%%%%%%%%%%%%%%%%%%%%%%%%
\section{Material Necessário}
%%%%%%%%%%%%%%%%%%%%%%%%%%%%%%%%%%%%%%%%%%%%%%%%%%%%%%%%%%%%%%%%%%%%%%%%%%%%%%%

\begin{itemize}
	\item Item 1;
	\item Item 2.
\end{itemize}

%%%%%%%%%%%%%%%%%%%%%%%%%%%%%%%%%%%%%%%%%%%%%%%%%%%%%%%%%%%%%%%%%%%%%%%%%%%%%%%
\section{Procedimento Experimental}
%%%%%%%%%%%%%%%%%%%%%%%%%%%%%%%%%%%%%%%%%%%%%%%%%%%%%%%%%%%%%%%%%%%%%%%%%%%%%%%

%%%%%%%%%%%%%%%%%%%%%
%\subsection{Parte A} % Se necessário
%%%%%%%%%%%%%%%%%%%%%
\begin{enumerate}
	\item Passo 1;
	\item Passo 2;
	\item Passo 3.
\end{enumerate}

%%%%%%%%%%%%%%%%%%%%%%%%%%%%%%%%%%%%%%%%%%%%%%%%%%%%%%%%%%%%%%%%%%%%%%%%%%%%%%%
%%%%%%%%%%%%%%%%%%%%%%%%%%%%%%%%%%%%%%%%%%%%%%%%%%%%%%%%%%%%%%%%%%%%%%%%%%%%%%%
%%%%%%%%%%%%%%%%%%%%%%%%%%%%%%%%%%%%%%%%%%%%%%%%%%%%%%%%%%%%%%%%%%%%%%%%%%%%%%%
%%%%%%%%%%%%%%%%%%%%%%%%%%%%%%%%%%%%%%%%%%%%%%%%%%%%%%%%%%%%%%%%%%%%%%%%%%%%%%%
\cleardoublepage

\noindent{}{\huge\textit{Campo magnético terrestre}}

\vspace{15mm}

\begin{fullwidth}
\noindent{}\makebox[0.6\linewidth]{Turma:\enspace\hrulefill}\makebox[0.4\textwidth]{  Data:\enspace\hrulefill}
\vspace{5mm}

\noindent{}\makebox[0.6\linewidth]{Aluno(a):\enspace\hrulefill}\makebox[0.4\textwidth]{  Matrícula:\enspace\hrulefill}

\noindent{}\makebox[0.6\linewidth]{Aluno(a):\enspace\hrulefill}\makebox[0.4\textwidth]{  Matrícula:\enspace\hrulefill}

\noindent{}\makebox[0.6\linewidth]{Aluno(a):\enspace\hrulefill}\makebox[0.4\textwidth]{  Matrícula:\enspace\hrulefill}

\noindent{}\makebox[0.6\linewidth]{Aluno(a):\enspace\hrulefill}\makebox[0.4\textwidth]{  Matrícula:\enspace\hrulefill}

\noindent{}\makebox[0.6\linewidth]{Aluno(a):\enspace\hrulefill}\makebox[0.4\textwidth]{  Matrícula:\enspace\hrulefill}
\end{fullwidth}

\vspace{5mm}

%%%%%%%%%%%%%%%%%%%%%%%%%%%%%%%%%%%%%%%%%%%%%%%%%%%%%%%%%%%%%%%%%%%%%%%%%%%%%%%
\section{Questionário}
%%%%%%%%%%%%%%%%%%%%%%%%%%%%%%%%%%%%%%%%%%%%%%%%%%%%%%%%%%%%%%%%%%%%%%%%%%%%%%%

\begin{question}[type={exam}]{1}
Apresente os resultados de maneira clara e organizada. Mostre os cálculos requisitados de maneira clara e sucinta, evidenciando o raciocínio desenvolvido.
\end{question}

\begin{question}[type={exam}]{1}
Preencha as colunas de dados experimentais das tabelas com o número adequado de algarismos significativos e unidades.
\end{question}

\begin{question}[type={exam}]{2}
Lorem ipsum dolor sit amet, consectetuer adi-
piscing elit. Ut purus elit, vestibulum ut, placerat ac, adipiscing vitae,
felis. Curabitur dictum gravida mauris. Nam arcu libero, nonummy
eget, consectetuer id, vulputate a, magna. Donec vehicula augue
eu neque. Pellentesque habitant morbi tristique senectus et netus
et malesuada fames ac turpis egestas. Mauris ut leo. Cras viverra
metus rhoncus sem. Nulla et lectus vestibulum urna fringilla ultrices.
\end{question}

\begin{question}[type={exam}]{2}
Lorem ipsum dolor sit amet, consectetuer adi-
piscing elit. Ut purus elit, vestibulum ut, placerat ac, adipiscing vitae,
felis. Curabitur dictum gravida mauris. Nam arcu libero, nonummy
eget, consectetuer id, vulputate a, magna. Donec vehicula augue
eu neque. Pellentesque habitant morbi tristique senectus et netus
et malesuada fames ac turpis egestas. Mauris ut leo. Cras viverra
metus rhoncus sem. Nulla et lectus vestibulum urna fringilla ultrices.
\end{question}

\begin{question}[type={exam}]{2}
Lorem ipsum dolor sit amet, consectetuer adi-
piscing elit. Ut purus elit, vestibulum ut, placerat ac, adipiscing vitae,
felis. Curabitur dictum gravida mauris. Nam arcu libero, nonummy
eget, consectetuer id, vulputate a, magna. Donec vehicula augue
eu neque. Pellentesque habitant morbi tristique senectus et netus
et malesuada fames ac turpis egestas. Mauris ut leo. Cras viverra
metus rhoncus sem. Nulla et lectus vestibulum urna fringilla ultrices.
\end{question}

\begin{question}[type={exam}]{2}
Lorem ipsum dolor sit amet, consectetuer adi-
piscing elit. Ut purus elit, vestibulum ut, placerat ac, adipiscing vitae,
felis. Curabitur dictum gravida mauris. Nam arcu libero, nonummy
eget, consectetuer id, vulputate a, magna. Donec vehicula augue
eu neque. Pellentesque habitant morbi tristique senectus et netus
et malesuada fames ac turpis egestas. Mauris ut leo. Cras viverra
metus rhoncus sem. Nulla et lectus vestibulum urna fringilla ultrices.
\end{question}

\begin{question}[type={exam}]{2}
Lorem ipsum dolor sit amet, consectetuer adi-
piscing elit. Ut purus elit, vestibulum ut, placerat ac, adipiscing vitae,
felis. Curabitur dictum gravida mauris. Nam arcu libero, nonummy
eget, consectetuer id, vulputate a, magna. Donec vehicula augue
eu neque. Pellentesque habitant morbi tristique senectus et netus
et malesuada fames ac turpis egestas. Mauris ut leo. Cras viverra
metus rhoncus sem. Nulla et lectus vestibulum urna fringilla ultrices.
\end{question}
\vfill
%%%%%%%%%%%%%%%%%%%%%%%%%%%%%%%%%%%%%%%%%%%%%%%%%%%%%%%%%%%%%%%%%%%%%%%%%%%%%%%
\pagebreak
\section{Tabelas}
%%%%%%%%%%%%%%%%%%%%%%%%%%%%%%%%%%%%%%%%%%%%%%%%%%%%%%%%%%%%%%%%%%%%%%%%%%%%%%%

