%%%%%%%%%%%%%%%%%%%%%%%%%%%%%%%%%%%%%%%%%%%%%%%%%%%%%%%%%%%%%%%%%%%%%%%%%%%%%%%
\chapter{Campo magnético terrestre} % Sem "Experiência 01" ou qualquer outro número
\label{Chap:CampoMagTerrestre}        % para poder trocar a ordem com facilidade
%%%%%%%%%%%%%%%%%%%%%%%%%%%%%%%%%%%%%%%%%%%%%%%%%%%%%%%%%%%%%%%%%%%%%%%%%%%%%%%

\begin{fullwidth}\it
	Vamos utilizar as bobinas de Helmholtz para gerar um campo magnético perpendicular ao campo magnético da Terra, possibilitando que determinemos a intensidade da componente horizontal deste último. Vamos calcular a intensidade do campo gerado por uma espira circular portando corrente e estender esse resultado para determinar o campo gerado pelas bobinas de Helmholtz. Finalmente, vamos usar uma linearização e uma regressão linear para determinar o resultado para a intensidade do camp magnético terrestre. 
\end{fullwidth}

%%%%%%%%%%%%%%%%%%%%%%%%%%%%%%%%%%%%%%%%%%%%%%%%
\section{Campo magnético devido a uma espira}
%%%%%%%%%%%%%%%%%%%%%%%%%%%%%%%%%%%%%%%%%%%%%%%%

Podemos determinar o campo magnético de uma espira circular ao longo do eixo $z$ que passa por seu centro, perpendicularmente ao plano que contém a espira, através da lei de Biot-Savart:
\begin{equation}
    d\vec{B} = \frac{\mu_0}{4\pi} \frac{I\, d\vec{\ell} \times \hat{r}}{r^2}.
\end{equation}
%
No caso particular da espira circular, a equação acima descreve o campo $d\vec{B}$ gerado por um segmento de comprimento $d\vec{\ell}$ do condutor que porta uma corrente $I$.

\begin{figure}
\centering
\begin{tikzpicture}[>=Stealth, scale = 4, rotate around y = -25]
	\draw[->] (0,0,0) -- (2,0,0) node[below left]{$z$};
	\draw[->] (0,0,0) -- (0,0.75,0) node[below left]{$x$};
	\draw[->] (0,0,0) -- (0,0,0.7) node[left]{$y$};
	
 	\begin{scope}[canvas is zy plane at x=0]
		\draw[densely dotted, gray, very thin, step=2mm] (-0.6,-0.6) grid (0.6,0.6);
		\draw (0,0) circle (0.5cm);
		\draw (0,0) circle (0.48cm);

		\fill (85:0.48) rectangle node[above left]{$d\vec{\ell}$}(95:0.5);
		
		\draw[->, dashed] (0,0) -- node[below right]{$R$} (-30:0.48);

		\draw[->] (70:0.55) arc(70:50:0.55) node[midway, above]{$I$};
	\end{scope}
	
 	\begin{scope}[canvas is xy plane at z=0]
		\draw[->] (0,0.5) coordinate (dl) -- node[above]{$\vec{r}$} (1.5,0) coordinate (Z);
		\draw[->] (1.5,0) -- +(0.17,0.25) coordinate (dB) node[right]{$d\vec{B}$};
		
		\pic[draw, "$\cdot$", angle radius = 3mm, angle eccentricity = 0.5]{angle = dB--Z--dl};
	\end{scope}

\end{tikzpicture}
\caption{Campo $d\vec{B}$ devido à corrente $I$ no segmento $d\vec{\ell}$ da espira.}
\end{figure}

Note que devido à simetria rotacional do problema, verificamos que sobre o eixo $z$ as componentes $B_x$ e $B_y$ do campo magnético são nulas, restando somente a componente $B_z$. Por essa razão, vamos nos preocupar somente com a determinação de $dB_z$ de agora em diante. Esta componente pode ser deteminada ao projetarmos o campo $dB$ na direção do eixo $z$ através de 
\begin{equation}
    dB_z = dB \sen\theta.
\end{equation}
%
Podemos determinar o módulo do campo magnético ao tomarmos a lei de Biot-Savart em módulo:
\begin{equation}
    dB = \frac{\mu_0}{4\pi} \frac{I\, |d\vec{\ell} \times \hat{r}|}{r^2}.
\end{equation}
%
Como $d\vec{\ell} \perp \hat{r}$, o módulo do produto vetorial entre esses dois vetores resulta em
\begin{equation}
    |d\vec{\ell} \times \hat{r}| = d\ell.
\end{equation}
%
Além disso, ainda que a distância $r$ entre o ponto onde estamos calculando campo e a posição do segmento $d\vec{\ell}$ pode ser escrita em termos de $z$ e do raio $R$ da espira:
\begin{equation}
    r = \sqrt{z^2 + R^2}.
\end{equation}
%
Assim,
\begin{align}
    dB_z &= dB \cdot \sin\theta \\
    &= \left(\frac{\mu_0}{4\pi} \cdot \frac{I \,d\ell}{z^2 + R^2}\right) \cdot \sen\theta \\
    &= \left(\frac{\mu_0}{4\pi} \cdot \frac{I \,d\ell}{z^2 + R^2}\right) \cdot \left(\frac{R}{\sqrt{z^2+R^2}}\right) \\
    &= \frac{\mu_0}{4\pi} \cdot \frac{IR \,d\ell}{(z^2 + R^2)^{\nicefrac{3}{2}}}.
\end{align}
%
Para determinarmos o campo total na direção do eixo $z$, basta agora realizarmos a integração ao longo do caminho formado pela espira:
\begin{align}
    B_z(z) &= \oint \frac{\mu_0}{4\pi} \cdot \frac{IR \,d\ell}{(z^2 + R^2)^{\nicefrac{3}{2}}} \\
    &= \frac{\mu_0}{4\pi} \cdot \frac{IR}{(z^2 + R^2)^{\nicefrac{3}{2}}} \oint d\ell \\
    &= \frac{\mu_0}{4\pi} \cdot \frac{IR}{(z^2 + R^2)^{\nicefrac{3}{2}}} \cdot 2\pi R \\
    &= \frac{\mu_0}{2} \cdot \frac{IR^2}{(z^2 + R^2)^{\nicefrac{3}{2}}}. \label{Eq:CampoMagneticoEspiraCircular}
\end{align}

No caso de um grupo de $N$ espiras de mesmas dimensões, próximas umas das outras, podemos escrever o campo magnético em termos da corrente $i$ de uma espira como
\begin{equation}\label{Eq:CampoMagneticoGrupoEspiras}
    B_z(z) = \frac{\mu_0}{2} \cdot \frac{iNR^2}{(z^2 + R^2)^{\nicefrac{3}{2}}}.
\end{equation}
    

%%%%%%%%%%%%%%%%%%%%%%%%%%%%%%%%%
\subsection{Bobinas de Helmholtz}
%%%%%%%%%%%%%%%%%%%%%%%%%%%%%%%%%

As bobinas de Helmholtz são uma configuração particular de dois grupos paralelos de espiras de mesmas dimensões, separadas por uma distância $h = R$. A corrente $i$ nas espiras e seu sentido  devem ser iguais nos dois grupos de espiras. Nesse caso, o campo na região central dos grupos de espiras é bastante uniforme, tornando esse tipo de aparato útil para diversos tipos de aplicação. 

\begin{figure*}\forcerectofloat
\centering
\begin{tikzpicture}[>=Stealth, scale = 4, rotate around y = -15]

	\draw[dashed] (0,0,0) -- (0.25,0,0); % x
	\draw[dashed,->] (0,0,0) -- (0,0.8,0); % y
	\draw[dashed,->] (0,0,0) -- (0,0,0.7); % z
	
	\draw[->] (0.25,0,0) -- (1.25,0,0) node[below left]{$z'$}; % x
	\draw[->] (0.25,0,0) -- (0.25,0.8,0) node[below left]{$x'$}; % y
	\draw[->] (0.25,0,0) -- (0.25,0,0.7) node[below right]{$y'$}; % z
	
 	\begin{scope}[canvas is zy plane at x=0]
		\draw[densely dotted, gray, very thin, step=2mm] (-0.6,-0.6) grid (0.6,0.6);
		\draw (0,0) circle (0.5cm);
		\draw (0,0) circle (0.48cm);

		%\draw[->, dashed] (0,0) -- node[below right]{$R$} (-30:0.48);

		\draw[->] (-30:0.55) arc(-30:-50:0.55) node[midway, below left]{$i$};
	\end{scope}
	
 	\begin{scope}[canvas is zy plane at x=0.5]
 		\draw[dashed, ->] (0,0) -- (0,0.8);
 		\draw[dashed, ->] (0,0) -- (0.7,0);
 		
		% \draw[densely dotted, gray, very thin, step=2mm] (-0.6,-0.6) grid (0.6,0.6);
		\draw (0,0) circle (0.5cm);
		\draw (0,0) circle (0.48cm);

		%\draw[->, dashed] (0,0) -- node[below right]{$R$} (-30:0.48);

		\draw[->] (-30:0.55) arc(-30:-50:0.55) node[midway, below left]{$i$};
	\end{scope}
	
 	\begin{scope}[canvas is xy plane at z=0]
		\draw[<->] (0,0.9) -- node[above]{$h=R$} (0.5,0.9);
		\draw (0.5,0.92) -- (0.5,0.88);
		\draw (0,0.92) -- (0,0.88);
	\end{scope}
	
	\draw[fill = white, draw = black] (0,0,0) circle (0.15mm);
	\draw[fill = white, draw = black] (0.5,0,0) circle (0.15mm);
	\fill (0.25,0,0) circle (0.15mm);

	\begin{scope}[rotate around y = 3, shift = {(2.25,0.17,0)}]
		\draw[dashed] (0,0,0) -- (0.25,0,0); % x
		\draw[dashed,->] (0,0,0) -- (0,0.8,0); % y
		\draw[dashed,->] (0,0,0) -- (0,0,0.7); % z
		
		\draw[->] (0.25,0,0) -- (1.25,0,0) node[below left]{$z'$}; % x
		\draw[->] (0.25,0,0) -- (0.25,0.8,0) node[below left]{$x'$}; % y
		\draw[->] (0.25,0,0) -- (0.25,0,0.7) node[below right]{$y'$}; % z
		
	 	\begin{scope}[canvas is zy plane at x=0]
			\draw[densely dotted, gray, very thin, step=2mm] (-0.6,-0.6) grid (0.6,0.6);
			\draw (0,0) circle (0.5cm);
			\draw (0,0) circle (0.48cm);

			%\draw[->, dashed] (0,0) -- node[below right]{$R$} (-30:0.48);

			\draw[->] (-30:0.55) arc(-30:-50:0.55) node[midway, below left]{$i$};
		\end{scope}
		
	 	\begin{scope}[canvas is zy plane at x=0.5]
	 		\draw[dashed, ->] (0,0) -- (0,0.8);
	 		\draw[dashed, ->] (0,0) -- (0.7,0);
	 		
			% \draw[densely dotted, gray, very thin, step=2mm] (-0.6,-0.6) grid (0.6,0.6);
			\draw (0,0) circle (0.5cm);
			\draw (0,0) circle (0.48cm);

			%\draw[->, dashed] (0,0) -- node[below right]{$R$} (-30:0.48);

			\draw[->] (-30:0.55) arc(-30:-50:0.55) node[midway, below left]{$i$};
		\end{scope}
		
	 	\begin{scope}[canvas is xy plane at z=0]
			\draw[<->] (0,0.9) -- node[above]{$h=R$} (0.5,0.9);
			\draw (0.5,0.92) -- (0.5,0.88);
			\draw (0,0.92) -- (0,0.88);
		\end{scope}
		
		\draw[fill = white, draw = black] (0,0,0) circle (0.15mm);
		\draw[fill = white, draw = black] (0.5,0,0) circle (0.15mm);
		\fill (0.25,0,0) circle (0.15mm);
	
	\end{scope}
\end{tikzpicture}
\caption{\emph{Figura estereoscópica:} Bobinas de Helmholtz.}
\end{figure*}

Para determinarmos o campo na região central, basta somarmos o campo $B_1$ ---~devido ao primeiro grupo de espiras~--- e o campo $B_2$ ---~devido ao segundo grupo~---. Note que a Equação~\eqref{Eq:CampoMagneticoGrupoEspiras} foi deduzida considerando que a origem do sistema de coordenadas estava localizada no centro da espira. Agora gostaríamos que a posição central entre as espiras fosse a origem de um eixo $z'$ e que as bobinas estivessem localizadas nas posições $\pm R/2$. Nesse caso, temos que as variáveis $z_1$ e $z_2$ das expressões originais podem ser escritas em termos de $z'$ como
\begin{align}
    z_1 &= z' + R/2 \\
    z_2 &= z' - R/2,
\end{align}
%
para as bobinas localizadas nas posições $z'_1 = -R/2$ e $z'_2 = R/2$. Assim,
\begin{align}
    B_H(z') &= B_1 + B_2 \\
    &= \frac{\mu_0 N i}{2} \cdot \frac{R^2}{[(z' + R/2)^2 + R^2]^{\nicefrac{3}{2}}} + \frac{\mu_0 N i}{2} \cdot \frac{R^2}{[(z' - R/2)^2 + R^2]^{\nicefrac{3}{2}}} \\
    &= \frac{\mu_0 N i}{2} \cdot \left(\frac{R^2}{[(z' + R/2)^2 + R^2]^{\nicefrac{3}{2}}} + \frac{R^2}{[(z' - R/2)^2 + R^2]^{\nicefrac{3}{2}}}\right). \label{Eq:CampoMagneticoBobinasDeHelmholtz}
\end{align}

\begin{marginfigure}
\centering
\begin{tikzpicture}[>=Stealth, yscale = 1.5, xscale = 1.25, extended line/.style={shorten >=-#1,shorten <=-#1},
 extended line/.default=3mm]] 
    \draw[->] (0,0) -- (0,1.25) node[below left] {$B_z$};
	\draw[->] (-1.75,0) -- (1.75,0) node[below left] {$z'$};

    % Desenhar função:
    \def\R{1}
    \draw[smooth, name path=plot,samples=1000,domain=-1.5:1.5]
    plot(\x,{pow(\R, 2)/(2 * pow(pow(\x - \R/2, 2) + pow(\R, 2), 3/2)) + pow(\R, 2)/(2 * pow(pow(\x + \R/2, 2) + pow(\R, 2), 3/2))});

    \draw[smooth, dashed, name path=plot2,samples=1000,domain=-1.5:1.5]
    plot(\x,{pow(\R, 2)/(2 * pow(pow(\x - \R/2, 2) + pow(\R, 2), 3/2))});
    
    \draw[smooth, dashed, name path=plot3,samples=1000,domain=-1.5:1.5]
    plot(\x,{pow(\R, 2)/(2 * pow(pow(\x + \R/2, 2) + pow(\R, 2), 3/2))});
    
    % Note que o 1 abaixo na verdade corresponde ao valor de \R acima
    \draw (0, 0.0) -- +(0,-0.1) node[below]{$0$};
    \draw (1, 0.0) -- +(0,-0.1) node[below]{$R$};
    \draw (-1, 0.0) -- +(0,-0.1) node[below]{$-R$};
    
	\end{tikzpicture}
\caption{Intensidade do campo magnético no espaço entre as bobinas. As linhas tracejadas denotam os campos devidos às duas bobinas separadamente. Note que o as intensidades máximas dessas duas curvas correspondem às posições $\pm R/2$, que são as posições das bobinas no eixo $z$. A linha cheia denota o campo total devido às duas bobinas. Note que entre $z' = -R/2$ e $z' = R/2$ temos que a intensidade é aproximadamente contante. \label{Fig:CampoMagneticoBobinasDeHelmholtz}}
\end{marginfigure}

Na Figura~\ref{Fig:CampoMagneticoBobinasDeHelmholtz} temos um gráfico da intensidade do campo magnético na região central entre as bobinas e podemos verificar que seu valor é aproximadamente constante. Para determinar tal valor, basta fazermos $z' = 0$ na Equação~\eqref{Eq:CampoMagneticoBobinasDeHelmholtz}:
\begin{align}
    B_H(z'=0) &=  \frac{\mu_0 N i}{2} \cdot \left(\frac{R^2}{[(z' + R/2)^2 + R^2]^{\nicefrac{3}{2}}} + \frac{R^2}{[(z' - R/2)^2 + R^2]^{\nicefrac{3}{2}}}\right) \\
    &= \frac{\mu_0 N i}{2} \cdot \left(\frac{R^2}{[(0 + R/2)^2 + R^2]^{\nicefrac{3}{2}}} + \frac{R^2}{[(0 - R/2)^2 + R^2]^{\nicefrac{3}{2}}}\right) \\
    &=  \frac{\mu_0 N i}{2} \cdot \left(\frac{R^2}{[R^2/4 + R^2]^{\nicefrac{3}{2}}} + \frac{R^2}{[R^2/4)^2 + R^2]^{\nicefrac{3}{2}}}\right) \\
    &= \frac{\mu_0 N i}{2} \cdot \left(\frac{R^2}{[5R^2/4]^{\nicefrac{3}{2}}} + \frac{R^2}{[5R^2/4)^2]^{\nicefrac{3}{2}}}\right).
\end{align}
%
Note que os dois termos dentro dos parêntesis são iguais, logo,
\begin{align}
    B_H(z'=0) &= \frac{\mu_0 N i}{2} \cdot \left(2\frac{R^2}{[5R^2/4]^{\nicefrac{3}{2}}}\right) \\
    &= \frac{\mu_0 N i}{2} \cdot \left(2\frac{R^2}{R^3[5/4]^{\nicefrac{3}{2}}}\right) \\
    &= \frac{\mu_0 N i}{R} \cdot \left(\frac{1}{[5/4]^{\nicefrac{3}{2}}}\right) \\
    &= \frac{\mu_0 N i}{R} \cdot \left(\frac{4}{5}\right)^{\rlap{\nicefrac{3}{2}}}.
\end{align}

%%%%%%%%%%%%%%%%%%%%%%%%%%%%%%%%%%%%%%%%%%%%%%%%%%%%%%
\subsection{Determinação do campo magnético terrestre}
%%%%%%%%%%%%%%%%%%%%%%%%%%%%%%%%%%%%%%%%%%%%%%%%%%%%%%

Podemos utilizar uma bússola e as bobinas de Helmholtz para determinar a intensidade da componente horizontal do campo magnético da Terra. Isso é possível pois uma bússola na verdade simplesmente aponta na direção do campo magnético total no ponto onde está localizada, não necessariamente para o verdadeiro norte magnético.

\begin{marginfigure}
\centering
\begin{tikzpicture}[>=Stealth] 
    \draw[->] (0,0) -- (0,3) node[below left] {$x'$};
	\draw[->] (0,0) -- (4,0) node[below left] {$z'$};

    \draw[thick, ->] (0,0) -- (0,2) node[below left] {$\vec{B}_H$};
	\draw[thick, ->] (0,0) -- (2.5,0) node[below left] {$\vec{B}_T$};
	
	\draw[thick, ->] (0,0) -- (2.5,2) node[right] {$\vec{B}$};
	
	\draw[dotted] (0,2) -- (2.5,2) -- (2.5,0);
	
	\draw[->] (1,0) arc (0:38.66:1) node[midway, right]{$\phi$};

\end{tikzpicture}
\caption{O campo total $\vec{B}$ é dado pela soma vetorial dos campos da Terra $\vec{B}_T$ e das bobinas $\vec{B}_H$. \label{Fig:CampoMagneticoTotalSobreBussola}}
\end{marginfigure}

Nesse caso, se dispusermos a bússola e as bobinas de forma que:
\begin{itemize}
    \item A bússola esteja localizada na região central entre as bobinas e aponte inicialmente para o norte;
    \item O eixo das bobinas (denotado por $z'$ na discussão acima) aponte perpendicularmente à direção da bússola;
\end{itemize}
%
teremos uma situação como a mostrada na Figura~\ref{Fig:CampoMagneticoTotalSobreBussola}. Note que ao controlarmos a corrente que passa pelas bobinas, determinamos o campo magnético $\vec{B}_H$ e a direção do vetor $\vec{B}$, uma vez que o campo $\vec{B}_T$ é constante.

Para determinarmos o campo magnético da Terra, verificamos que ---~usando a definição da função tangente~---
\begin{equation}
    \tan \phi = \frac{B_H}{B_T}.
\end{equation}
%
Consequentemente, podemos escrever a relação
\begin{equation}
    B_H = B_T \cdot \tan \phi.
\end{equation}
%
Portanto, se fizermos medidas da deflexão da agulha da bússola e do campo magnético $B_H$ correspondente, podemos determinar o campo magnético $B_T$ da Terra ao fazermos uma regressão linear dos dados de um gráfico $B_H \times \tan\phi$.

Note que podemos inverter o sentido da corrente nas bobinas, com uma consequente alteração no sentido do campo $\vec{B}_H$, o que causará uma deflexão da agulha da bússola no sentido oposto ao registrado com o sentido inicial para a corrente. Podemos usar desse artifício para obter mais dados experimentais e melhorar o resultado obtido para o valor do campo magnético $B_T$. Para isso, vamos registrar os dados para o ângulo $\phi$ contidos no quarto quadrante como ângulos negativos, medidos no sentido horário a partir do eixo $z'$. Correspondentemente, vamos considerar os valores de $B_H$ que apontam no sentido negativo de $x'$ como negativos.


%%%%%%%%%%%%%%%%%%%%%%%%%%%%%%%%%%%%%%%%%%%%%%%%%%%%%%%%%%%%%%%%%%%%%%%%%%%%%%%
\section{Experimento}
%%%%%%%%%%%%%%%%%%%%%%%%%%%%%%%%%%%%%%%%%%%%%%%%%%%%%%%%%%%%%%%%%%%%%%%%%%%%%%%

%%%%%%%%%%%%%%%%%%%%%%
\subsection{Objetivos}
\label{Sec:ObjetivosCampoMagneticoTerrestre}
%%%%%%%%%%%%%%%%%%%%%%

\begin{itemize}
	\item Determinar a componente horizontal do campo magnético da Terra.
\end{itemize}

%%%%%%%%%%%%%%%%%%%%%%%%%%%%%%%%%%%%%%%%%%%%%%%%%%%%%%%%%%%%%%%%%%%%%%%%%%%%%%%
\section{Material Necessário}
%%%%%%%%%%%%%%%%%%%%%%%%%%%%%%%%%%%%%%%%%%%%%%%%%%%%%%%%%%%%%%%%%%%%%%%%%%%%%%%

\begin{itemize}
	\item Bobinas de Helmholtz;
	\item Fonte de tensão regulável;
	\item Multímetro;
	\item Cabos para ligação.
\end{itemize}

%%%%%%%%%%%%%%%%%%%%%%%%%%%%%%%%%%%%%%%%%%%%%%%%%%%%%%%%%%%%%%%%%%%%%%%%%%%%%%%
\section{Procedimento Experimental}
%%%%%%%%%%%%%%%%%%%%%%%%%%%%%%%%%%%%%%%%%%%%%%%%%%%%%%%%%%%%%%%%%%%%%%%%%%%%%%%

\begin{marginfigure}[2cm]
\centering
\begin{circuitikz}[american]
	\draw (0,1) to[battery1, v=$V$, i=$i$] (0,3) to[switch] (2,3) to[L, l=$L$] (2,1) to[L, l=$L$] (2,-1) to[smeter, t=A] (0,-1) to[vR, l=$R$] (0,1);
	%\draw (2,3) -- (3.5,3)  (3.5,0) -- (2,0);
\end{circuitikz}
\caption{Circuito para as bobinas de Helmholtz. Note que o resistor variável mostrado na figura pode estar dentro da própria fonte de tensão, se ela tiver controle de corrente.}
\end{marginfigure}

%%%%%%%%%%%%%%%%%%%%%
%\subsection{Parte A} % Se necessário
%%%%%%%%%%%%%%%%%%%%%
\begin{enumerate}
	\item Ligue a fonte e ajuste um valor de tensão de \np[V]{1,0} e torne a desligá-la. Não alteraremos mais esse valor;\footnote{Nesse experimento, vamos usar os controles de corrente. Durante o experimento, você verificará que os valores de tensão lidos não corresponderão ao que foi ajustado nesse passo, mas tal valor funcionará como um teto de tensão.}
	\item Disponha a bússola sobre a mesa. Note que é necessário que a superfície da mesa esteja razoavelmente bem nivelada.
	\item Certifique-se de que não haja nenhum material magnético próximo da bússola.\footnote{Como o experimento é bastante sensível, mesmo materiais ferromagnéticos podem atrapalhar as medidas. Note ainda que alguns relógios de pulso contém ímãs, podendo distorcer a direção do campo ao fazermos os ajustes.}
	\item Verifique a direção do norte com a bússola;
	\item Disponha as bobinas de Helmholtz de forma que o eixo das bobinas estejam perpendiculares à direção do norte;
	\item Coloque o apoio de madeira entre as bobinas e coloque a bússola sobre ele. Tome cuidado para que a marca do norte na escala da bússola esteja apontando na direção norte indicada pela agulha. Provavelmente será necessário fazer pequenos ajustes, mas isso pode ser feito adiante, uma vez que ao conectar os cabos inevitavelmente serão feitos pequenos deslocamentos do conjunto;
	\item Ligue o terminal negativo da fonte de tensão ao terminal negativo de uma das bobinas;
	\item Ligue o terminal positivo da bobina ao terminal negativo da outra bobina;
	\item Ligue o terminal positivo da bobina ao terminal comum do multímetro;
	\item Ligue o terminal marcado como $\texttt{mA}\mathdirectcurrent$ ao terminal positivo da fonte;
	\item Ajuste os controles de corrente para o \emph{mínimo};
	\item Verifique a posição da bússola e das bobinas de Helmholtz, certificando-se de que marca do norte coincida com a direção apontada pela agulha da bússola e que o eixo das bobinas aponta perpendicularmente à direção indicada pela agulha.
	\item Ligue a fonte e proceda a ajustar a corrente lentamente, até que ocorra um desvio de \degree{10} na direção da agulha da bússola em relação à direção inicial.
	\item Anote o valor do ângulo e a corrente correspondente na Tabela~\ref{Tab:DadosCampoMagneticoTerrestre}.
	\item Volte a aumentar a corrente, registrando seu valor a cada \degree{10}.
	\item Após atingir \degree{80}, ajuste os controles de corrente para o valor mínimo, desligue a fonte e inverta os cabos ligados a ela.
	\item Religue a fonte e aumente o valor de corrente progressivamente, anotando seu valor a cada \degree{10} até que sejam atingidos \degree{80}. Note que os sinais do ângulo e da corrente devem ser opostos aos obtidos antes de invertermos as conexões, não esqueça de registrá-los. 
\end{enumerate}

%%%%%%%%%%%%%%%%%%%%%%%%%%%%%%%%%%%%%%%%%%%%%%%%%%%%%%%%%%%%%%%%%%%%%%%%%%%%%%%
%%%%%%%%%%%%%%%%%%%%%%%%%%%%%%%%%%%%%%%%%%%%%%%%%%%%%%%%%%%%%%%%%%%%%%%%%%%%%%%
%%%%%%%%%%%%%%%%%%%%%%%%%%%%%%%%%%%%%%%%%%%%%%%%%%%%%%%%%%%%%%%%%%%%%%%%%%%%%%%
%%%%%%%%%%%%%%%%%%%%%%%%%%%%%%%%%%%%%%%%%%%%%%%%%%%%%%%%%%%%%%%%%%%%%%%%%%%%%%%
\cleardoublepage

\noindent{}{\huge\textit{Campo magnético terrestre}}

\vspace{15mm}

\begin{fullwidth}
\noindent{}\makebox[0.6\linewidth]{Turma:\enspace\hrulefill}\makebox[0.4\textwidth]{  Data:\enspace\hrulefill}
\vspace{5mm}

\noindent{}\makebox[0.6\linewidth]{Aluno(a):\enspace\hrulefill}\makebox[0.4\textwidth]{  Matrícula:\enspace\hrulefill}

\noindent{}\makebox[0.6\linewidth]{Aluno(a):\enspace\hrulefill}\makebox[0.4\textwidth]{  Matrícula:\enspace\hrulefill}

\noindent{}\makebox[0.6\linewidth]{Aluno(a):\enspace\hrulefill}\makebox[0.4\textwidth]{  Matrícula:\enspace\hrulefill}

\noindent{}\makebox[0.6\linewidth]{Aluno(a):\enspace\hrulefill}\makebox[0.4\textwidth]{  Matrícula:\enspace\hrulefill}

\noindent{}\makebox[0.6\linewidth]{Aluno(a):\enspace\hrulefill}\makebox[0.4\textwidth]{  Matrícula:\enspace\hrulefill}
\end{fullwidth}

\vspace{5mm}

%%%%%%%%%%%%%%%%%%%%%%%%%%%%%%%%%%%%%%%%%%%%%%%%%%%%%%%%%%%%%%%%%%%%%%%%%%%%%%%
\section{Questionário}
%%%%%%%%%%%%%%%%%%%%%%%%%%%%%%%%%%%%%%%%%%%%%%%%%%%%%%%%%%%%%%%%%%%%%%%%%%%%%%%

\begin{question}[type={exam}]{2}
Preencha as colunas de dados experimentais das tabelas com o número adequado de algarismos significativos e unidades.
\end{question}

\begin{question}[type={exam}]{2}
Determine o valor do campo magnético $B_H$ e preencha a terceira coluna da Tabela~\ref{Tab:DadosCampoMagneticoTerrestre}.
\end{question}

\begin{question}[type={exam}]{4}
Considerando os dados de $B_H$ e de $\phi$
\begin{enumerate}[label=\roman*.]
    \item Elabore um gráfico linearizado dos dados.
    \item Determine a componente horizontal do campo magnético terrestre através de uma regressão linear dos dados. 
\end{enumerate}
\end{question}

\begin{question}[type={exam}]{2}
Considerando os objetivos do experimento, listados na Seção~\ref{Sec:ObjetivosCampoMagneticoTerrestre}, e os resultados obtidos nas questões anteriores, discuta quais objetivos foram atingidos com sucesso, justificando suas conclusões. Se algum objetivo não foi atingido, discuta quais são os possíveis motivos do insucesso e que providências podem ser tomadas para que eles sejam alcançados.
\end{question}

\vfill
%%%%%%%%%%%%%%%%%%%%%%%%%%%%%%%%%%%%%%%%%%%%%%%%%%%%%%%%%%%%%%%%%%%%%%%%%%%%%%%
\pagebreak
\section{Tabelas}
%%%%%%%%%%%%%%%%%%%%%%%%%%%%%%%%%%%%%%%%%%%%%%%%%%%%%%%%%%%%%%%%%%%%%%%%%%%%%%%

\begin{table*}[!h]
\centering
\begin{tabular}{lp{25mm}p{25mm}cp{25mm}l}
\toprule
	& \multicolumn{2}{l}{\textbf{Dados da bobinas}} \\
	\cmidrule{2-3}
	& Diâmetro: \cellcolor[gray]{0.89} & \cellcolor[gray]{0.92} \\
	& $N$: \cellcolor[gray]{0.95} & \cellcolor[gray]{0.97} \\
	\cmidrule{2-3}
\\
	& \multicolumn{2}{l}{\textbf{Dados experimentais}} & & \multicolumn{2}{l}{\textbf{Campo}}\\
	\cmidrule{2-3}\cmidrule{5-5}
	& $\phi$ & $i$ && $B_H$ & \\
	\cmidrule{2-3}\cmidrule{5-5}
	& \cellcolor[gray]{0.89} & \cellcolor[gray]{0.92} & &\cellcolor[gray]{0.89} & \\
	& \cellcolor[gray]{0.95} & \cellcolor[gray]{0.97} & &\cellcolor[gray]{0.95} & \\
	& \cellcolor[gray]{0.89} & \cellcolor[gray]{0.92} & &\cellcolor[gray]{0.89} & \\
	& \cellcolor[gray]{0.95} & \cellcolor[gray]{0.97} & &\cellcolor[gray]{0.95} & \\
	& \cellcolor[gray]{0.89} & \cellcolor[gray]{0.92} & &\cellcolor[gray]{0.89} & \\
	& \cellcolor[gray]{0.95} & \cellcolor[gray]{0.97} & &\cellcolor[gray]{0.95} & \\
	& \cellcolor[gray]{0.89} & \cellcolor[gray]{0.92} & &\cellcolor[gray]{0.89} & \\
	& \cellcolor[gray]{0.95} & \cellcolor[gray]{0.97} & &\cellcolor[gray]{0.95} & \\
	& \cellcolor[gray]{0.89} & \cellcolor[gray]{0.92} & &\cellcolor[gray]{0.89} & \\
	& \cellcolor[gray]{0.95} & \cellcolor[gray]{0.97} & &\cellcolor[gray]{0.95} & \\
	& \cellcolor[gray]{0.89} & \cellcolor[gray]{0.92} & &\cellcolor[gray]{0.89} & \\
	& \cellcolor[gray]{0.95} & \cellcolor[gray]{0.97} & &\cellcolor[gray]{0.95} & \\
	& \cellcolor[gray]{0.89} & \cellcolor[gray]{0.92} & &\cellcolor[gray]{0.89} & \\
	& \cellcolor[gray]{0.95} & \cellcolor[gray]{0.97} & &\cellcolor[gray]{0.95} & \\
	& \cellcolor[gray]{0.89} & \cellcolor[gray]{0.92} & &\cellcolor[gray]{0.89} & \\
	& \cellcolor[gray]{0.95} & \cellcolor[gray]{0.97} & &\cellcolor[gray]{0.95} & \\
	& \cellcolor[gray]{0.89} & \cellcolor[gray]{0.92} & &\cellcolor[gray]{0.89} & \\
	& \cellcolor[gray]{0.95} & \cellcolor[gray]{0.97} & &\cellcolor[gray]{0.95} & \\
	\cmidrule{2-6}
\bottomrule
\end{tabular}
\caption[][5mm]{Dados para a deflexão da agulha de uma bússola em função da corrente nas bobinas de Helmholtz.}
\label{Tab:DadosCampoMagneticoTerrestre}
\end{table*}

