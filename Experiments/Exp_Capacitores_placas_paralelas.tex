%%%%%%%%%%%%%%%%%%%%%%%%%%%%%%%%%%%%%%%%%%%%%%%%%%%%%%%%%%%%%%%%%%%%%%%%%%%%%%%
\chapter{Capacitores de placas paralelas} % Sem "Experiência 01" ou qualquer outro número
\label{Chap:CapacPlacasPar}        % para poder trocar a ordem com facilidade
%%%%%%%%%%%%%%%%%%%%%%%%%%%%%%%%%%%%%%%%%%%%%%%%%%%%%%%%%%%%%%%%%%%%%%%%%%%%%%%

\begin{fullwidth}\it
	Vamos realizar um experimento usando um capacitor de placas paralelas cuja distância entre as placas é regulável. Isso nos permitirá observar a relação entre a capacitância e a distância, verificanco a expressão teórica para a relação entre essas duas grandezas, bem como empregar um outro meio dielétrico entre as placas que não seja o ar, permitindo que determinemos sua constante dielétrica.
\end{fullwidth}

%%%%%%%%%%%%%%%%%%%%%%%%%%%%%%%%%%%%%%%%%%%%%%%%%%%%%%%%%%%%%%%%%%%%%%%%%%%%%%%
\section{Capacitância}
%%%%%%%%%%%%%%%%%%%%%%%%%%%%%%%%%%%%%%%%%%%%%%%%%%%%%%%%%%%%%%%%%%%%%%%%%%%%%%%

Se conectarmos uma bateria a dois condutores\footnote{Na verdade, mesmo um só condutor também tem uma capacitância.} quaisques ---~dispostos um próximo ao outro, porém desconectados~---, haverá um fluxo de cargas elétricas até que se estabeleça uma diferença de potencial entre os dois condutores igual à diferença de potencial entre os polos da bateria.

Dependendo de características geométricas desses dois condutores, como formato, área, e distância entre eles, a quantidade de carga necessária para estabelecer uma diferença de potencial igual à da bateria será diferente. Para uma dada configuração geométrica dos condutores, damos à razão entre a quantidade de carga $Q$ e a diferença de potencial $V$ correspondente o nome de \emph{capacitância}:
\begin{equation}\label{Eq:DefCapacitancia}
	C = \frac{Q}{V}.
\end{equation}
%
Podemos considerar a capacitância como uma medida da quantidade de carga que um corpo condutor pode armazenar por unidade de potencial elétrico. A unidade da capacitância é o \emph{farad}, representado por $\rm{F}$, sendo que
\begin{equation}
	\np[F]{1} = \np[C/V]{1}.
\end{equation}
%
Ao conjunto de condutores damos o nome de \emph{capacitor}.

%%%%%%%%%%%%%%%%%%%%%%%%%%%%%%%%%%%%%%%%%%%%
\subsection{Capacitores de placas paralelas}
%%%%%%%%%%%%%%%%%%%%%%%%%%%%%%%%%%%%%%%%%%%%

Se considerarmos um capacitor cujos condutores são duas placas paralelas de área $A$, separadas por uma distância $d$, podemos determinar a sua capacitância analiticamente.

Para isso, basta considerarmos que uma das placas tem carga $Q$ e a outra tem carga $-Q$. Nesse caso, podemos dizer que a densidade superficial de carga $\sigma$ é dada por
\begin{equation}
	\sigma = \frac{Q}{A}.
\end{equation}
%
Usando a Lei de Gauss, obtemos
\begin{align}
	E &= \frac{\sigma}{\epsilon_0} \\
	&= \frac{Q}{\epsilon_0 A}.
\end{align}
%
Como o potencial no espaço entre as placas é dado por $Ed$, podemos escrever
\begin{equation}
	V = \frac{Qd}{\epsilon_0 A}.
\end{equation}
%
Finalmente, substituindo na equação para a capacitância
\begin{align}
	C &= \frac{Q}{V} \\
	&= \frac{Q}{Qd/(\epsilon_0 A)} \\
	&= \frac{\epsilon_0 A}{d}.
\end{align}

%%%%%%%%%%%%%%%%%%%%%%%%
\subsection{Dielétricos}
%%%%%%%%%%%%%%%%%%%%%%%%

Vamos considerar agora a inserção de um meio não condutor qualquer entre as placas do capacitor. Se carregarmos as placas de um capacitor de placas paralelas e as conectarmos a um voltímetro ideal ---~isto é, um voltímetro com resistência infinita e que por isso não permite que corrente flua através dele~--- vamos verificar um valor de diferença de potencial dada pela relação
\begin{equation}
    V = \frac{Q}{C}.
\end{equation}
%
Se inserirmos o material não-condutor entre as placas, verificaremos que \emph{ocorre uma diminuição na diferença de potencial registrada pelo voltímetro}. Se calcularmos a razão entre a tensão $V_0$ antes da inserção do dielétrico e a tensão $V$ depois da inserção, obtemos
\begin{equation}
    \kappa = \frac{V_0}{V}.
\end{equation}
%
A constante $\kappa$ é denominada como \emph{constante dielétrica} e é característica de cada material não-condutor, sendo que $\kappa > 1$.

Calculando o novo valor de capacitância através da Equação~\eqref{Eq:DefCapacitancia} e usando a relação $V = V_0 / \kappa$ que podemos obter da equação acima, resulta em
\begin{align}
    C &= \frac{Q}{V} \\
    &= \frac{Q}{V_0 / \kappa} \\
    &=\kappa \frac{Q}{V_0}.
\end{align}
%
Como não ocorre mudança na quantidade de carga armazenada nas placas ao inserirmos o dielétrico, temos que
\begin{equation}
    Q = Q_0,
\end{equation}
%
assim,
\begin{align}
    C &= \kappa \frac{Q_0}{V_0} \\
    &= \kappa \, C_0.
\end{align}
%
O valor $C_0$ representa o valor de capacitância antes da inserção do dielétrico. Como $\kappa > 1$, verificamos que ocorre a insersão do dielétrico leva a um \emph{aumento da capacitância}.

O aumento da capacitância pode ser entendido através de uma análise microscópica do comportamento das cargas que compõe o dielétrico: quando ele é inserido entre as placas carregadas do capacitor, as eletrosferas e núcleos atômicos sofrem um pequeno deslocamento no sentido de se aproximar das placas carregadas positiva e negativamente, respectivamente. Isso dá origem a um campo elétrico devido ao próprio dielétrico e que tem sentido oposto ao campo gerado pelas placas do capacitor ---~denominamos esse fenômeno como uma \emph{polarização} do dielétrico~---. Portanto, o campo elétrico do dielétrico polarizado ``compensa'' parcialmente o campo gerado pelas placas do capacitor, o que diminui o campo elétrico total e, consequentemente, a diferença de potencial entre as placas.\footnote{A combinação do campo elétrico devido às placas do capacitor e devido à polarização do dielétrico é conhecido como \emph{deslocamento elétrico}.}

É importante notar que na discussão para definir a capacitância não consideramos a existência do ar como um meio polarizável. Na verdade, o ar é um dielétrico que sofre uma polarização como outro qualquer, portanto existe um valor de $\kappa_{\text{ar}}$ que relaciona a capacitância $C_{\text{ar}}$ de um capacitor quando há ar entre as suas placas, com o seu valor de capacitância no vácuo. No entanto, o valor de $\kappa_{\text{ar}}$ é muito próximo de 1 e podemos desprezar os efeitos do ar como um meio dielétrico na maior parte dos casos.

%%%%%%%%%%%%%%%%%%%%%%%%%%%%%%%%%%%%%%%%%%%%%%%%%%%%%%%%%%%%%%%%%%%%%%%%%%%%%%%
\section{Experimento}
%%%%%%%%%%%%%%%%%%%%%%%%%%%%%%%%%%%%%%%%%%%%%%%%%%%%%%%%%%%%%%%%%%%%%%%%%%%%%%%

%%%%%%%%%%%%%%%%%%%%%%
\subsection{Objetivos}
%%%%%%%%%%%%%%%%%%%%%%

\begin{itemize}
	\item Observar a relação entre a capacitância e a distância entre as placas condutoras;
	\item Determinar a constante dielétrica do material inserido entre as placas;
\end{itemize}

%%%%%%%%%%%%%%%%%%%%%%%%%%%%%%%%%%%%%%%%%%%%%%%%%%%%%%%%%%%%%%%%%%%%%%%%%%%%%%%
\section{Material Necessário}
%%%%%%%%%%%%%%%%%%%%%%%%%%%%%%%%%%%%%%%%%%%%%%%%%%%%%%%%%%%%%%%%%%%%%%%%%%%%%%%

\begin{itemize}
	\item Capacitor de placas paralelas móveis;
	\item Multímetro e cabos de conexão;
	\item Lâmina dielétricas;
	\item Micrômetro e paquímetro.
\end{itemize}

%%%%%%%%%%%%%%%%%%%%%%%%%%%%%%%%%%%%%%%%%%%%%%%%%%%%%%%%%%%%%%%%%%%%%%%%%%%%%%%
\section{Procedimento Experimental}
%%%%%%%%%%%%%%%%%%%%%%%%%%%%%%%%%%%%%%%%%%%%%%%%%%%%%%%%%%%%%%%%%%%%%%%%%%%%%%%

%%%%%%%%%%%%%%%%%%%%%
%\subsection{Parte A} % Se necessário
%%%%%%%%%%%%%%%%%%%%%
\begin{enumerate}
	\item Utilizando um paquímetro, verifique as dimensões das placas do capacitor;
	\item Utilizando um micrômetro, meça a espessura de uma lâmina dielétrica e a coloque entre as placas do capacitor de formaque que não reste ar entre as placas e a lâmina. Anote o valor obtido na Tabela~\ref{Tab:ValoresCapacitancia};\label{Item:CapPlacasPar:MedidaMicrometroLamina}
	\item Afira o valor de capacitância e o anote na Tabela~\ref{Tab:ValoresCapacitancia};
	\item Tomando cuidado para não mover as placas do capacitor, remova a lâmina dielétrica e afira o novo valor de capacitância e o anote na Tabela~\ref{Tab:ValoresCapacitancia}
	\item Repita os passos acima a partir do item \ref{Item:CapPlacasPar:MedidaMicrometroLamina}, anotando o valor de distância total entre as placas e as capacitâncias com e sem dielétrico na Tabela~\ref{Tab:ValoresCapacitancia}.
\end{enumerate}

%%%%%%%%%%%%%%%%%%%%%%%%%%%%%%%%%%%%%%%%%%%%%%%%%%%%%%%%%%%%%%%%%%%%%%%%%%%%%%%
%%%%%%%%%%%%%%%%%%%%%%%%%%%%%%%%%%%%%%%%%%%%%%%%%%%%%%%%%%%%%%%%%%%%%%%%%%%%%%%
%%%%%%%%%%%%%%%%%%%%%%%%%%%%%%%%%%%%%%%%%%%%%%%%%%%%%%%%%%%%%%%%%%%%%%%%%%%%%%%
%%%%%%%%%%%%%%%%%%%%%%%%%%%%%%%%%%%%%%%%%%%%%%%%%%%%%%%%%%%%%%%%%%%%%%%%%%%%%%%
\cleardoublepage

\noindent{}{\huge\textit{Capacitores de placas paralelas}}

\vspace{15mm}

\begin{fullwidth}
\noindent{}\makebox[0.6\linewidth]{Turma:\enspace\hrulefill}\makebox[0.4\textwidth]{  Data:\enspace\hrulefill}
\vspace{5mm}

\noindent{}\makebox[0.6\linewidth]{Aluno(a):\enspace\hrulefill}\makebox[0.4\textwidth]{  Matrícula:\enspace\hrulefill}

\noindent{}\makebox[0.6\linewidth]{Aluno(a):\enspace\hrulefill}\makebox[0.4\textwidth]{  Matrícula:\enspace\hrulefill}

\noindent{}\makebox[0.6\linewidth]{Aluno(a):\enspace\hrulefill}\makebox[0.4\textwidth]{  Matrícula:\enspace\hrulefill}

\noindent{}\makebox[0.6\linewidth]{Aluno(a):\enspace\hrulefill}\makebox[0.4\textwidth]{  Matrícula:\enspace\hrulefill}

\noindent{}\makebox[0.6\linewidth]{Aluno(a):\enspace\hrulefill}\makebox[0.4\textwidth]{  Matrícula:\enspace\hrulefill}
\end{fullwidth}

\vspace{5mm}

%%%%%%%%%%%%%%%%%%%%%%%%%%%%%%%%%%%%%%%%%%%%%%%%%%%%%%%%%%%%%%%%%%%%%%%%%%%%%%%
\section{Questionário}
%%%%%%%%%%%%%%%%%%%%%%%%%%%%%%%%%%%%%%%%%%%%%%%%%%%%%%%%%%%%%%%%%%%%%%%%%%%%%%%

\begin{question}[type={exam}]{2}
Preencha as colunas de dados experimentais das tabelas com o número adequado de algarismos significativos e unidades.
\end{question}

\begin{question}[type={exam}]{4} Considerando os dados de capacitância quando há \emph{somente ar} entre as placas do capacitor,
\begin{enumerate}[label=\roman*.]
    \item Faça um gráfico do valor de capacitância em função da distância de separação entre as placas.
    \item Faça um ajuste de uma curva do tipo $y = A + B \cdot x^{-1}$ para os dados obtidos.\footnote{Nem todo software é capaz de realizar esse tipo de ajuste, porém as calculadoras geralmente têm suporte a esse procedimento. Alternativamente, vocês podem fazer uma linearização dos dados e realizar uma regressão linear.} 
    \item Determine o valor da permissividade do vácuo $\epsilon_0$ através dos valores dos  coeficientes $A$ e $B$ calculados.
    \item Considerando o valor obtido no item anterior, calcule o erro percentual em relação ao valor de referência para $\epsilon_0$.
\end{enumerate}
\end{question}

\begin{question}[type={exam}]{4}
Considerando os dados de capacitância quando as placas do capacitor estão separadas pelas \emph{lâminas dielétricas},
\begin{enumerate}[label=\roman*.]
    \item Faça um gráfico do valor de capacitância em função da distância de separação entre as placas.
    \item Faça um ajuste de uma curva do tipo $y = A + B \cdot x^{-1}$ para os dados obtidos.\footnote{Novamente, pode ser necessário fazer uma linearização dos dados e realizar uma regressão linear, caso o software utilizado não tenha suporte a ajustes do tipo $y = A + B \cdot x^{-1}$.} 
    \item Através dos coeficientes $A$ e $B$ obtidos no item anterior, determine o valor da constante dielétrica $\kappa$ para o material utilizado. 
\end{enumerate}
\end{question}

\vfill
%%%%%%%%%%%%%%%%%%%%%%%%%%%%%%%%%%%%%%%%%%%%%%%%%%%%%%%%%%%%%%%%%%%%%%%%%%%%%%%
\pagebreak
\section{Tabelas}
%%%%%%%%%%%%%%%%%%%%%%%%%%%%%%%%%%%%%%%%%%%%%%%%%%%%%%%%%%%%%%%%%%%%%%%%%%%%%%%

\begin{table*}[!h]
\centering
\begin{tabular}{lp{25mm}p{25mm}p{25mm}l}
\toprule
	& \multicolumn{2}{l}{\textbf{Dimensões}} \\
	\cmidrule{2-3}
	& Diâmetro: \cellcolor[gray]{0.89} & \cellcolor[gray]{0.92} \\
	& Espessura: \cellcolor[gray]{0.95} & \cellcolor[gray]{0.97} \\
	\cmidrule{2-3}
\\
	& \multicolumn{4}{l}{\textbf{Dados experimentais}} \\
	\cmidrule{2-4}
	& $C$ & $C_0$ & $d$ & \\
	\cmidrule{2-4}
	& \cellcolor[gray]{0.89} & \cellcolor[gray]{0.92} & \cellcolor[gray]{0.89} & \\
	& \cellcolor[gray]{0.95} & \cellcolor[gray]{0.97} & \cellcolor[gray]{0.95} & \\
	& \cellcolor[gray]{0.89} & \cellcolor[gray]{0.92} & \cellcolor[gray]{0.89} & \\
	& \cellcolor[gray]{0.95} & \cellcolor[gray]{0.97} & \cellcolor[gray]{0.95} & \\
	& \cellcolor[gray]{0.89} & \cellcolor[gray]{0.92} & \cellcolor[gray]{0.89} & \\
	& \cellcolor[gray]{0.95} & \cellcolor[gray]{0.97} & \cellcolor[gray]{0.95} & \\
	& \cellcolor[gray]{0.89} & \cellcolor[gray]{0.92} & \cellcolor[gray]{0.89} & \\
	& \cellcolor[gray]{0.95} & \cellcolor[gray]{0.97} & \cellcolor[gray]{0.95} & \\
	& \cellcolor[gray]{0.89} & \cellcolor[gray]{0.92} & \cellcolor[gray]{0.89} & \\
	& \cellcolor[gray]{0.95} & \cellcolor[gray]{0.97} & \cellcolor[gray]{0.95} & \\
	\cmidrule{2-4}
\bottomrule
\end{tabular}
\caption[][5mm]{Dados para o movimento efetuado pela esfera sob ação da gravidade e do arrasto.}
\label{Tab:ValoresCapacitancia}
\end{table*}

