%%%%%%%%%%%%%%%%%%%%%%%%%%%%%%%%%%%%%%%%%%%%%%%%%%%%%%%%%%%%%%%%%%%%%%%%%%%%%%%
\chapter{Superfícies Equipotenciais} % Sem "Experiência 01" ou qualquer outro número
\label{Chap:SupEquip}        % para poder trocar a ordem com facilidade
%%%%%%%%%%%%%%%%%%%%%%%%%%%%%%%%%%%%%%%%%%%%%%%%%%%%%%%%%%%%%%%%%%%%%%%%%%%%%%%

\begin{fullwidth}\it
	Faremos um experimentos para determinar as superfícies equipotenciais de um arranjo de dipolar cargas elétricas. Através delas, seremos capazes de determinar a direção das linhas de campo elétrico.
\end{fullwidth}

%%%%%%%%%%%%%%%%%%%%%%%%%%%%%%%%%%%%%%%%%%%%%%%%%%%%%%%%%%%%%%%%%%%%%%%%%%%%%%%
\section{Potencial elétrico}
%%%%%%%%%%%%%%%%%%%%%%%%%%%%%%%%%%%%%%%%%%%%%%%%%%%%%%%%%%%%%%%%%%%%%%%%%%%%%%%

O trabalho realizado por uma força $\vec{F}$ pode ser calculado no caso mais geral através da \emph{integral de linha}\footnote{Também chamada de \emph{integral de caminho}.}
\begin{equation}
    W = \int_C \vec{F}(\vec{r}) \cdot d\vec{r},
\end{equation}
%
onde $C$ representa um caminho no espaço. Além disso, se a força é conservativa, podemos calcular o trabalho como o negativo da diferença de valores de uma função escalar $U(\vec{r})$ conhecida como energia potencial:
\begin{align}
    W &= \Delta U(\vec{r}) \\
    &= U(\vec{r}_f) - U(\vec{r}_i).
\end{align}

No caso particular da força eletrostática, é conveniente denotar a força $F(\vec{r})$ como sendo o produto entre o campo elétrico e o valor $q_0$ da carga que se desloca e cujo trabalho estamos interessados em determinar. Assim, podemos escrever
\begin{equation}\label{Eq:EnergiaPotencialParaDeslEmCampoEletrico}
    \Delta U(\vec{r}) = - \int_C q_0 \cdot \vec{E}(\vec{r}) \cdot d\vec{r}.
\end{equation}

Em fenômenos eletromagnéticos é bastante comum que não haja um interesse particular na quantidade de trabalho realizado em uma carga específica, mas sim na quantidade de trabalho realizada \emph{por unidade de carga}. Dessa maneira, definimos uma nova grandeza, conhecida como \emph{potencial elétrico},\footnote{Geralmente só dizemos \emph{potencial}, deixando o `elétrico' subentendido.} definida como
\begin{equation}
    V(\vec{r}) \equiv \frac{U(\vec{r})}{q_0}.
\end{equation}
%
Substituindo essa definição na expressão~\eqref{Eq:EnergiaPotencialParaDeslEmCampoEletrico}, obtemos
\begin{equation}\label{Eq:ExprPotencialEletrico}
    \Delta V(\vec{r}) = - \int_C \vec{E}(\vec{r}) \cdot d\vec{r}. \mathnote{Definição de potencial elétrico.}
\end{equation}

O potencial elétrico $V(\vec{r})$ é um campo escalar cujo valor é de especial interesse em aplicações práticas. Devido a características dos sistemas elétricos que empregamos no dia a dia, raramente nos preocupamos com sua dependência na posição $\vec{r}$ e estamos simplesmente interessados na \emph{diferença} de potencial $\Delta V$ entre dois pontos quaisquer de um circuito elétrico. Essa diferença é o que medimos com um multímetro e denominamos como \emph{tensão}, ou \emph{diferença de potencial}, ou ainda \emph{voltagem} (os três termos são equivalentes). Suas unidades são $\rm{J}/\rm{C}$ (`joule por coulomb'). 

Note que, da mesma maneira que o trabalho está relacionado à \emph{diferença de energia potencial} entre dois pontos, a tensão está relacionada à \emph{diferença de potencial elétrico}, isto é, não há interesse no valor particular de potencial elétrico, mas sim na diferença de valores entre dois pontos quaisquer de um circuito.\footnote{Isso é um reflexo do fato de que se o potencial é igual entre dois pontos, então a energia potencial também é, o que faz com que não haja trabalho algum realizado no deslocamento de uma carga entre as duas posições.} Quando tratamos um sistema de cargas de um ponto de vista da Física, costumamos escolher $\vec{r} = \infty$ como sendo o ponto onde o potencial é nulo. Em circuitos, é comum adotar o polo negativo de baterias como sendo o ponto de potencial nulo, ou no caso de sistemas mais complexos, \emph{o potencial do solo} é considerado nulo: essa é a razão pela qual dizemos que há um \emph{aterramento} de um sistema.

%%%%%%%%%%%%%%%%%%%%%%%%%%%%%%%%%%%%%%%%
\section{Potencial de uma carga pontual}
%%%%%%%%%%%%%%%%%%%%%%%%%%%%%%%%%%%%%%%%

O potencial de uma carga pontual pode ser calculado de maneira simples através da Equação~\eqref{Eq:ExprPotencialEletrico} e da expressão para o campo elétrico gerado por ela, isto é,
\begin{equation}
    \vec{E}(\vec{r}) = \frac{kq}{r^2}.
\end{equation}
%
Realizando a integração para esse caso particular, obtemos
\begin{align}
    \Delta V(\vec{r}) &= - \int_C \vec{E}(\vec{r}) \cdot d\vec{r} \\
    &= - \int_C \frac{kq}{r^2} \hat{r} \cdot d\vec{r}.
\end{align}
%
Como a força depende somente da distância entre as cargas, podemos tratar a integral acima como unidimensional, o que pode ser verificado matematicamente pelo produto escalar $\hat{r}\cdot d\vec{r} = dr$. Considerando um caminho que leve de um ponto $\vec{r}_A = \infty$ a um ponto qualquer $r_B = r$, temos\footnote{Estamos considerando que a carga que gera o campo se encontra na origem do sistema de coordenadas.} 
\begin{align}
    \Delta V(\vec{r}) &= - \int_{r_A}^{r_B} \frac{kq}{r^2} \,dr \\
    V(r_B) - V(r_A) &= - kq \int_{\infty}^{r} (r')^{-2} \,dr' \\
    &= kq [(r')^{-1} + C]_{\infty}^{r} \\
    &= \frac{kq}{r}.
\end{align}
%
Note que o potencial na posição $r_A = \infty$ é nulo, assim podemos escrever que o potencial elétrico devido a uma carga é dado por
\begin{equation}
    V(r) = \frac{kq}{r}.
\end{equation}

No caso uma configuração de cargas mais complexa, a visualização do potencial elétrico não é simples. Nesse caso precisamos calcular o potencial através da Equação~\eqref{Eq:ExprPotencialEletrico}, porém somando sobre todas as cargas elétricas:
\begin{equation}\label{Eq:ExprPotencialEletricoConjCargas}
    \Delta V_T(\vec{r}) = - \sum_{i=1}^{N}\int_C \vec{E}_i(\vec{r}_i) \cdot d\vec{r}. \mathnote{Potencial elétrico de um conjunto de cargas.}
\end{equation}
%
Note que o potencial elétrico $V_T$ devido a uma distribuição de várias cargas pode ser calculado através de uma simples soma, pois ele é definido como sendo o trabalho total realizado sobre a carga de teste devido às diferentes cargas da distribuição e \emph{o trabalho é uma grandeza aditiva}. Alternativamente, podemos calcular o campo elétrico total $E_T(\vec{r}) = \sum_{i = 1}^{N}$ e depois determinar o resultado da integral.

%%%%%%%%%%%%%%%%%%%%%%%%%%%
\paragraph{Dipolo elétrico}
%%%%%%%%%%%%%%%%%%%%%%%%%%%

Para o caso de um \emph{dipolo elétrico}, isto é, uma distribuição de cargas elétricas formada por uma carga negativa e uma positiva ---~ambas de mesmo valor $q$~---, temos a soma dos potenciais de duas cargas pontuais.
\begin{figure}[h]
\centering
\begin{tikzpicture}[>=Stealth]
    \draw[->] (-3,0) -- (3,0) node[below left]{$x$};
    \draw[->] (0,-1) -- (0,2) node[below left]{$y$};
    
    \draw(3,2) circle (1pt);
    \draw[->] (0,0) -- node[sloped, below]{$\vec{r}$} (3,2);
    \draw[->] (-2,0) -- node[sloped, above]{$\vec{r}_{-}$} (3,2);
    \draw[->] (2,0) -- node[right]{$\vec{r}_{+}$} (3,2);
    
    \fill[white] (-2,0) circle (2mm);
    \fill[white] (2,0) circle (2mm);
    \draw[pattern = north west lines, pattern color = gray] (-2,0) circle (2mm) node{$-$};
    \draw[pattern = north west lines, pattern color = gray] (2,0) circle (2mm) node{$+$};
    
    \draw[|<->] (-2, -0.5) -- node[below]{$a$}(0, -0.5);
    \draw[<->|] (0, -0.5) -- node[below]{$a$}(2, -0.5);
    
\end{tikzpicture}
\caption{Dipolo elétrico.}
\end{figure}

\noindent{}Considerando que as cargas estão nas posições $-a$ e $a$ de um eixo $x$ cuja origem se encontra no ponto médio entre as cargas, podemos escrever o potencial como
\begin{align}
    V(\vec{r}) &= \frac{k(-q)}{r_{-}} + \frac{kq}{r_{+}} \\
    &= kq \cdot\left(\frac{-1}{\sqrt{(r_x+a)^2 + r_y^2}} + \frac{1}{\sqrt{(r_x-a)^2 + r_y^2}}\right), \label{Eq:PotDipoloEletricoXY}
\end{align}
%
onde $r_x$ e $r_y$ são as componentes do vector $\vec{r}$. Para o caso particular do potencial sobre o eixo $x$, é possível escrever uma expressão simples:
\begin{align}
    V(x) &= \frac{k\cdot(-q)}{x-a} + \frac{k\cdot q}{x + a} \\
    &= -\frac{2kqa}{x^2 - a^2}.
\end{align}

%%%%%%%%%%%%%%%%%%%%%%%%%%%%%%%%%%%%
\section{Superfícies equipotenciais}
%%%%%%%%%%%%%%%%%%%%%%%%%%%%%%%%%%%%

Se considerarmos uma configuração qualquer de cargas, existe uma superfície contígua formada por pontos onde o potencial tem o mesmo valor. Tal superfície é o que denominamos de \emph{superfície equipotencial}. No caso de um campo elétrico uniforme, por exemplo, o trabalho realizado sobre uma carga em um deslocamento ao longo do campo é dado por
\begin{align}
    \Delta V(\vec{r}) &= - \int_C \vec{E}(\vec{r}) \cdot d\vec{r} \\
    &= - \int_C E \versi \cdot d\vec{r} \\
    &= - E \int_{x_i}^{x_f} dx \\
    &= - E \, \Delta x.
\end{align}

\begin{marginfigure}
\centering
\begin{tikzpicture}[>=Stealth]
    \foreach \y in {0, 0.5, ..., 4}
        \draw[->] (0, \y) -- (3,\y);
        
    \draw[->] (0,-0.5) -- (3, -0.5) node[below left]{$\vec{E}$};
    
    \foreach \x in {0.5, 1, ..., 2.5}
        \draw[dashed] (\x, -1) -- (\x, 4.5);
        
\end{tikzpicture}
\caption{Equipotenciais no caso de um campo elétrico uniforme. Como a figura é bidimensional, as superfícies se reduzem a linhas equipotenciais.}
\end{marginfigure}
%
\noindent{}É possível notar através dessa expressão que o potencial é o mesmo em planos perpendiculares ao campo elétrico. Na verdade, localmente ---~isto é, se considerarmos uma região pequena do espaço~--- \emph{a superfície equipotencial sempre será perpendicular ao campo elétrico}. Essa observação é bastante importante pois uma vez conhecida uma dessas duas grandezas, ela nos trás informação sobre a outra.
    
\begin{marginfigure}
\begin{tikzpicture}[>=Stealth]

    \draw[->] (-2.5,0) -- (2.5,0) node[below left]{$x$};
    \draw[->] (0,-2.5) -- (0,2.5) node[below left]{$y$};
    
    \fill[white] (0,0) circle (2mm);
    \draw[pattern = north west lines, pattern color = gray] (0,0) circle (2mm) node{$+$};
    
    \foreach \V in {1, ..., 8}{
        \def\radius{(2/\V)}
        \draw (0,0) circle (\radius);
    }
        
\end{tikzpicture}
\caption{Em um plano $xy$ as equipotenciais são linhas círculos concêntricos, no caso de uma carga pontual.\label{Fig:EquipotenciaisCargaPontualXY}}
\end{marginfigure}

Outro caso em que as superfícies equipotenciais são relativamente simples é o de uma carga pontual: como o potencial depende somente da distância até a carga, as superfícies equipotenciais são esferas concêntricas e em cujos centros se encontra a carga elétrica em questão. Na Figura~\ref{Fig:EquipotenciaisCargaPontualXY} temos a representação das equipotenciais de uma carga pontual para o caso bidimensional. Note ainda que quanto mais próximas estão as equipotenciais em uma dada região do figura, maior é a intensidade do campo elétrico.

%%%%%%%%%%%%%%%%%%%%%%%%%%%%%%%%%%
\paragraph{Potencial de um dipolo}
%%%%%%%%%%%%%%%%%%%%%%%%%%%%%%%%%%

Um caso mais interessante é o das superfícies equipotenciais de um dipolo elétrico. Devido ao fato de que temos uma interação com duas cargas, uma exercendo uma força atrativa sobre a carga de teste e outra exercendo uma força negativa, temos um potencial mais complexo, onde as equipotenciais não são simples de calcular. A Figura~\ref{Fig:EquipotenciaisDipoloXY} mostra a configuração das equipotenciais para o caso bidimensional.

\begin{figure}
\centering
%
% Gráficos das linhas de campo:
%
% Para fazer os gráficos, preciso escrever x e y como funções de um
% parâmetro (o \x nos comandos abaixo). Para isso, fazermos a seguinte
% mudança de variáveis (vi isso nesse site: https://math.stackexchange.com/questions/1262174/area-enclosed-by-an-equipotential-curve-for-an-electric-dipole-on-the-plane):
%
% \sqrt{(x + a)^2 + y^2} = u + v
% \sqrt{(x - a)^2 + y^2} = u - v,
%
% de onde vem que
%
% x = uv/a
% y = \pm \sqrt{-(a^2-u^2)(a^2-v^2)}/a.
%
% Como o argumento da raíz tem que ser positivo, ou v = [0, a] e u = [a, \infty], ou o
% contrário. Vamos usar só o primeiro quadrante, por isso y > 0. Usando a mudança de
% variáveis, podemos reescrever a soma dos dois potenciais das cargas do dipolo
%
% V = kq (1/\sqrt{(x - a)^2 + y^2} - 1/\sqrt{(x + a) + y^2})
%
% como
%
% V/kq = c = 1/(u-v) - 1/(u+v).
%
% Resolvendo para u, temos
%
% u = \sqrt{v^2 - 2v/c}.
%
% Podemos usar v como um parâmetro para desenhar a curva no primeiro quadrante
% se usarmos as expressões para x e y como funções de u e v, e a relação acima.
% Para obter os resultados nos outros quadrantes, precisamos simplesmente fazer
% a mesma curva refletida nos eixos, usando sinais. O domínio utilizado para v
% é de 0 até a, mas sempre pode dar dá problema numérico no cálculo da raíz, por
% isso temos que calcular o valor para o qual o argumento da raíz é maior que zero.
% Fazendo um gráfico do argumento da raíz,
%
% f(v) = - (v^2 + 2v/c - a^2)(v^2 - a^2)
%
% Verificamos que ele é positivo entre a única raíz v_0 que há no intervalo [0,a] e o valor de a. Calculando a raíz, obtemos
%
% \sqrt{a^2c^2 + 1}/c - 1/c.
%
% No entanto, como sempre podemos ter algum problema de precisão, a raíz pode
% dar negativa para um ponto ou outro. Assim é interessante usar um max(0, f(v))
% dentro da raíz, garantindo que não haja nenhum problema. Por outro lado, usando
% o valor de a como sendo o limite superior do domínio, também pode haver algum
% problema de arredondamento e as linhas não ficam fechadas certinho pois a curva
% não chega a tocar o eixo horizontal. Nesse caso, podemos resolver usando um limite
% um pouco maior, mas não podemos exagerar pois aparece uma linha horizontal no gráfico
% (pois a coordenada x continua mudando e y permanece constante, eu acho).
%
% Para determinar o valor da constante c, precisamos saber o valor da carga q.
% Não é a melhor solução do mundo, mas considerei que os polos têm diferença de
% potencial ddp e estão separados por uma distância 2a. Assim c = 2V / (ddp \cdot a)
% e temos algo relativamente simples. Se considerarmos dois cilindros condutores,
% a condição de contorno correta é que o potencial seja zero na superfície de um
% deles e V na superfície do outro, então o mais indicado seria determinar q usando
% o valor de potencial V a uma distância (a - R) da origem, sobre o eixo x, onde R
% é o raio do cilindro condutor. 
%
% Finalmente, se quisermos fazer um gráfico pra duas cargas de mesmo sinal, no
% potencial o sinal entre os termos vai mudar. Considerando o que tem no site,
% o que deve acontecer é que será possível escrever v em função de u, assim u
% passará a ser o parâmetro para desenhar as curvas.
%
\begin{tikzpicture}[>=Stealth, scale = 0.35]
    
    % Polos
    \fill[white] (-5,0) circle (5mm);
    \fill[white] (5,0) circle (5mm);
    \draw[pattern = north west lines, pattern color = gray] (-5,0) circle (5mm) node{$-$};
    \draw[pattern = north west lines, pattern color = gray] (5,0) circle (5mm) node{$+$};
    
    % Equipotenciais
    \draw[thick] (0,-10) -- (0,10);
    \clip (-15,10) rectangle (15,-10);
    
    \def\ddp{22}
    \def\a{5}
    \def\ds{5.01} % Na verdade ds = a, mas precisa ser um pouco maior por causa de erros de arredondamento
        
    \foreach \V in {2,4,6,8,10,12,14,16,18,20,22,24,26,28}{
        \def\c{(2 * \V / (\ddp * \a))}
        
        \def\di{(sqrt(\a * \a * \c * \c + 1) / \c - 1 / \c)}
        
        \draw [thick,  domain=\di:\ds, samples=1000] plot ({\x*sqrt(\x*\x + 2*\x/\c)/\a},{sqrt(max(0, -(\x*\x + 2*\x/\c - \a*\a)*(\x*\x - \a*\a)))/\a});
        \draw [thick,  domain=\di:\ds, samples=1000] plot ({\x*sqrt(\x*\x + 2*\x/\c)/\a},{-sqrt(max(0, -(\x*\x + 2*\x/\c - \a*\a)*(\x*\x - \a*\a)))/\a});
        \draw [thick,  domain=\di:\ds, samples=1000] plot ({-\x*sqrt(\x*\x + 2*\x/\c)/\a},{sqrt(max(0, -(\x*\x + 2*\x/\c - \a*\a)*(\x*\x - \a*\a)))/\a});
        \draw [thick,  domain=\di:\ds, samples=1000] plot ({-\x*sqrt(\x*\x + 2*\x/\c)/\a},{-sqrt(max(0, -(\x*\x + 2*\x/\c - \a*\a)*(\x*\x - \a*\a)))/\a});
    }
    
\end{tikzpicture}
\caption{Equipotenciais de um dipolo em um plano $xy$ que contém as cargas. \label{Fig:EquipotenciaisDipoloXY}}
\end{figure}



%%%%%%%%%%%%%%%%%%%%%%%%%%%%%%%%%%%%%%%%%%%%%
%\subsection{Se necessário, usar subsections}
%%%%%%%%%%%%%%%%%%%%%%%%%%%%%%%%%%%%%%%%%%%%%

%%%%%%%%%%%%%%%%%%%%%%%%%%%%%%%%%%%%%%%%%%%%%%%%%%%%%%%%%%%%%%%%%%%%%%%%%%%%%%%
\section{Experimento}
%%%%%%%%%%%%%%%%%%%%%%%%%%%%%%%%%%%%%%%%%%%%%%%%%%%%%%%%%%%%%%%%%%%%%%%%%%%%%%%

%%%%%%%%%%%%%%%%%%%%%%
\subsection{Objetivos}
\label{Sec:ObjetivosSupEquip}
%%%%%%%%%%%%%%%%%%%%%%

\begin{itemize}
	\item Obter as linhas equipotenciais de um dipolo elétrico em um plano que contém as cargas;
	\item Obter as linhas de campo elétrico a partir das linhas de potencial elétrico;
	\item Obter um gráfico do potencial ao longo da linha reta que une as cargas.
\end{itemize}

%%%%%%%%%%%%%%%%%%%%%%%%%%%%%%%%%%%%%%%%%%%%%%%%%%%%%%%%%%%%%%%%%%%%%%%%%%%%%%%
\section{Material Necessário}
%%%%%%%%%%%%%%%%%%%%%%%%%%%%%%%%%%%%%%%%%%%%%%%%%%%%%%%%%%%%%%%%%%%%%%%%%%%%%%%

\begin{itemize}
    \item Placa de Petri;
	\item Eletrodos com suporte;
    \item Becker com água;
	\item Papel milimetrado;
	\item Fonte regulável;
	\item Multímetro;
	\item Cabos para ligações: banana-jacaré (3), ponta de prova (1);
	
\end{itemize}

%%%%%%%%%%%%%%%%%%%%%%%%%%%%%%%%%%%%%%%%%%%%%%%%%%%%%%%%%%%%%%%%%%%%%%%%%%%%%%%
\section{Procedimento Experimental}
%%%%%%%%%%%%%%%%%%%%%%%%%%%%%%%%%%%%%%%%%%%%%%%%%%%%%%%%%%%%%%%%%%%%%%%%%%%%%%%

\ctikzset{resistors/scale=0.7}
\begin{marginfigure}
    \centering
    \begin{circuitikz}[american, scale = 1, >=Stealth]
        \draw (0,0) circle (2cm);
        \draw (-1.5,0) to[short, *-] (-2.5,0) to (-2.5,4) to[battery1, v=$V$] (2.5,4) to (2.5,0) to[short, -*] (1.5,0);
        \draw (-1,4) to[short,*-] (-1,3) to[smeter, t=V,-*] (1,3);
        
        \draw (1,3) .. controls (2,4) and (2.5,3) .. (1.5,2);
        
        \draw[very thick,->] (1.5,2) -- (1, 0.5);
        
        \clip (0,0) circle (2cm);
        \draw[step=.3cm,very thin, densely dotted] (-2,-2) grid (2,2);
        
        \draw[thin] (-1.25,0) -- (1.25,0);
        \draw[thin] (0,-1.25) -- (0,1.25);
    \end{circuitikz}
    \caption{Esquema do circuito elétrico para o experimento de superfícies equipotenciais.}
\end{marginfigure}

\begin{enumerate}
	\item Tome duas folhas de papel milimetrado e desenhe sistemas de referência cujas origens se encontrem no centro e cujas marcas estejam distanciadas de \np{1,00} em \np[cm]{1,00}. Ambos os eixos devem variar de \np[cm]{-5,00} a \np[cm]{+5,00};
	\item Disponha a placa de Petri sobre o papel milimetrado de forma que seu centro coincida com a origem do sistema de referência;
	\item Ligue os eletrodos à fonte usando os cabos fornecidos;
	\item Disponha os eletrodos na placa de Petri de forma que eles estejam localizados sobre o eixo $x$, a uma distância de \np[cm]{5,00} da origem;
	\item Ajuste o multímetro para a função de voltímetro e ligue o terminal comum ao eletrodo negativo usando o cabo fornecido;
	\item Ligue a ponta de prova ao terminal marcado com \texttt{V} do multímetro;
	\item Ajuste o multímetro para a escala de \np[V]{20}, corrente contínua (\np[V\mathdirectcurrent]{20}).
	\item Ligue a fonte regulável e a ajuste para uma tensão de \np[V]{15,0};
	\item Coloque água na placa de Petri;
	\item Com a ponta de prova do multímetro,\footnote{Tome cuidado para que a ponta de prova seja inserida na cuba verticalmente, perpendicular ao plano da mesa.} afira a tensão na origem do sistema de coordenadas e a cada \np[cm]{1,00} ao longo do eixo $y$ (tanto para valores positivos, quanto negativos). Na outra folha de papel milimetrado, marque um ponto na posição correspondente à da medida e anote os valores de tensão ao lado dos pontos;
	\item Afira a tensão ao longo do eixo $x$ desde o eletrodo negativo até o positivo a cada \np[cm]{1,00}.\footnote{Use as próprias marcas do eixo $x$ e os pontos intermediários entre elas, assim os valores ficarão mais simples e a origem do sistema de referência também estará incluída nos dados.} Anote os valores obtidos na Tabela~\ref{Tab:ValoresPotencialEletricoEixoX};
	\item Para a coluna $x = \np[cm]{-4,00}$ inicie com a ponta de prova em $y = \np[cm]{-5,00}$ e busque os valores de tensão dos pontos aferidos sobre o eixo $x$, percorrendo a coluna paralelamente ao eixo $y$. Note que talvez não seja possível encontrar todos os pontos;
	\item Repita o procedimento do item acima varrendo todas as colunas até $x = \np[cm]{+4,0}$.
\end{enumerate}

\begin{figure}
\centering
\begin{tikzpicture}[>=Stealth]


    
    % Grid + eixos
    \draw[->] (-6, 0) -- (6, 0) node[below left]{$x$};
    \draw[->] (0, -6) -- (0, 6) node[below left]{$y$};
    
    \foreach \x in {-5, -4, ..., 5}
        \foreach \y in {-5, -4, ..., 5}
            \draw[draw = black, fill=white](\x, \y) circle (1pt);
%    \draw (0,0) circle (1pt);
    
    % Polos
    \fill[white] (-5,0) circle (3mm);
    \fill[white] (5,0) circle (3mm);
    \draw[pattern = north west lines, pattern color = gray] (-5,0) circle (3mm) node{$-$};
    \draw[pattern = north west lines, pattern color = gray] (5,0) circle (3mm) node{$+$};
    
    % Equipotenciais
    \clip (-5.3,5.3) rectangle (5.3,-5.3);
    
    \def\ddp{22}
    \def\a{5}
        
    \foreach \V in {2,4,6,8,10,12,14,16,18,20,22,24,26,28}{
        \def\c{(2 * \V / (\ddp * \a))}
        
        \def\di{(sqrt(\a * \a * \c * \c + 1) / \c - 1 / \c)}
        
        \draw [gray, thick,  domain=\di:\a, samples=1000] plot ({\x*sqrt(\x*\x + 2*\x/\c)/\a},{sqrt(max(0, -(\x*\x + 2*\x/\c - \a*\a)*(\x*\x - \a*\a)))/\a});
        \draw [gray, thick,  domain=\di:\a, samples=1000] plot ({\x*sqrt(\x*\x + 2*\x/\c)/\a},{-sqrt(max(0, -(\x*\x + 2*\x/\c - \a*\a)*(\x*\x - \a*\a)))/\a});
        \draw [gray, thick,  domain=\di:\a, samples=1000] plot ({-\x*sqrt(\x*\x + 2*\x/\c)/\a},{sqrt(max(0, -(\x*\x + 2*\x/\c - \a*\a)*(\x*\x - \a*\a)))/\a});
        \draw [gray, thick,  domain=\di:\a, samples=1000] plot ({-\x*sqrt(\x*\x + 2*\x/\c)/\a},{-sqrt(max(0, -(\x*\x + 2*\x/\c - \a*\a)*(\x*\x - \a*\a)))/\a});
    }
    
    % Varredura
    \draw[dashed, ->] (-4, -5.5) -- (-4, -2.5);
    \draw[dashed, ->] (-3, -5.5) -- (-3, -3.5);
    \draw[dashed, ->] (-2, -5.5) -- (-2, -4.5);
    
\end{tikzpicture}
\caption{Varredura da área entre os dois polos em busca das linhas equipotenciais. Inicie no canto inferior esquerdo e varra as colunas verticalmente, como mostrado na figura. Busque os valores encontrados no eixo $x$. Note que nem todos os valores serão encontrados na região de varredura.\label{Fig:EquipotenciaisDipolo}}
\end{figure}

%%%%%%%%%%%%%%%%%%%%%%%%%%%%%%%%%%%%%%%%%%%%%%%%%%%%%%%%%%%%%%%%%%%%%%%%%%%%%%%
%%%%%%%%%%%%%%%%%%%%%%%%%%%%%%%%%%%%%%%%%%%%%%%%%%%%%%%%%%%%%%%%%%%%%%%%%%%%%%%
%%%%%%%%%%%%%%%%%%%%%%%%%%%%%%%%%%%%%%%%%%%%%%%%%%%%%%%%%%%%%%%%%%%%%%%%%%%%%%%
%%%%%%%%%%%%%%%%%%%%%%%%%%%%%%%%%%%%%%%%%%%%%%%%%%%%%%%%%%%%%%%%%%%%%%%%%%%%%%%
\cleardoublepage

\noindent{}{\huge\textit{Superfícies equipotenciais}}

\vspace{15mm}

\begin{fullwidth}
\noindent{}\makebox[0.6\linewidth]{Turma:\enspace\hrulefill}\makebox[0.4\textwidth]{  Data:\enspace\hrulefill}
\vspace{5mm}

\noindent{}\makebox[0.6\linewidth]{Aluno(a):\enspace\hrulefill}\makebox[0.4\textwidth]{  Matrícula:\enspace\hrulefill}

\noindent{}\makebox[0.6\linewidth]{Aluno(a):\enspace\hrulefill}\makebox[0.4\textwidth]{  Matrícula:\enspace\hrulefill}

\noindent{}\makebox[0.6\linewidth]{Aluno(a):\enspace\hrulefill}\makebox[0.4\textwidth]{  Matrícula:\enspace\hrulefill}

\noindent{}\makebox[0.6\linewidth]{Aluno(a):\enspace\hrulefill}\makebox[0.4\textwidth]{  Matrícula:\enspace\hrulefill}

\noindent{}\makebox[0.6\linewidth]{Aluno(a):\enspace\hrulefill}\makebox[0.4\textwidth]{  Matrícula:\enspace\hrulefill}
\end{fullwidth}

\vspace{5mm}

%%%%%%%%%%%%%%%%%%%%%%%%%%%%%%%%%%%%%%%%%%%%%%%%%%%%%%%%%%%%%%%%%%%%%%%%%%%%%%%
\section{Questionário}
%%%%%%%%%%%%%%%%%%%%%%%%%%%%%%%%%%%%%%%%%%%%%%%%%%%%%%%%%%%%%%%%%%%%%%%%%%%%%%%

\begin{question}[type={exam}]{2}
Preencha as colunas de dados experimentais das tabelas com o número adequado de algarismos significativos e unidades.
\end{question}

\begin{question}[type={exam}]{2}
\begin{enumerate}[label=\roman*.]
    \item Na folha de papel milimetrado onde foram marcados os valores de potencial, ligue os pontos equipotenciais obtidos de uma maneira suave.\footnote{Estamos interessados em verificar se obtemos uma figura similar à Figura~\ref{Fig:EquipotenciaisDipolo}}
    \item Considerando que as superfícies equipotenciais são sempre perpendiculares às linhas de campo elétrico, desenhe as linhas de campo de uma maneira estimada.
\end{enumerate}
\end{question}

\begin{question}[type={exam}]{2}
Faça um gráfico do potencial elétrico em função da posição $x$ para os dados contidos na Tabela~\ref{Tab:ValoresPotencialEletricoEixoX}.
\end{question}

\begin{question}[type={exam}]{2}
Usando os dados da Tabela~\ref{Tab:ValoresPotencialEletricoEixoX}, estime a intensidade do campo elétrico ao longo do eixo $x$ através de
\begin{equation}
    |E| = \left|\frac{\partial V}{\partial x}\right| \approx \left|\frac{\Delta V}{\Delta x}\right|.
\end{equation}
%
Faça um gráfico de $E(x)$ para o resultado obtido. Para os pontos das extremidades, use os valores de potencial dos eletrodos ao calcular o valor de $\Delta V$.
\end{question}

\begin{question}[type={exam}]{2}
Considerando os objetivos do experimento, listados na Seção~\ref{Sec:ObjetivosSupEquip}, e os resultados obtidos nas questões anteriores, discuta quais objetivos foram atingidos com sucesso, justificando suas conclusões. Se algum objetivo não foi atingido, discuta quais são os possíveis motivos do insucesso e que providências podem ser tomadas para que eles sejam alcançados.
\end{question}

\vfill
%%%%%%%%%%%%%%%%%%%%%%%%%%%%%%%%%%%%%%%%%%%%%%%%%%%%%%%%%%%%%%%%%%%%%%%%%%%%%%%
\pagebreak
\section{Tabelas}
%%%%%%%%%%%%%%%%%%%%%%%%%%%%%%%%%%%%%%%%%%%%%%%%%%%%%%%%%%%%%%%%%%%%%%%%%%%%%%%


\begin{table}
\label{Tab:ValoresPotencialEletricoEixoX}
	\begin{center}
		\begin{tabular}{cp{25mm}p{25mm}c}
		\toprule
		&\multicolumn{2}{l}{\textbf{Dados para o potencial.}} \\
		\cmidrule{2-3}
		& $x$ & $V$ \\
		\cmidrule{2-3}
		& \cellcolor[gray]{0.89} & \cellcolor[gray]{0.92} & \\
		& \cellcolor[gray]{0.95} & \cellcolor[gray]{0.97} \\
		& \cellcolor[gray]{0.89} & \cellcolor[gray]{0.92} \\
		& \cellcolor[gray]{0.95} & \cellcolor[gray]{0.97} \\
		& \cellcolor[gray]{0.89} & \cellcolor[gray]{0.92} \\
		& \cellcolor[gray]{0.95} & \cellcolor[gray]{0.97} \\
		& \cellcolor[gray]{0.89} & \cellcolor[gray]{0.92} \\
		& \cellcolor[gray]{0.95} & \cellcolor[gray]{0.97} \\
		& \cellcolor[gray]{0.89} & \cellcolor[gray]{0.92} \\
		& \cellcolor[gray]{0.95} & \cellcolor[gray]{0.97} \\
		& \cellcolor[gray]{0.89} & \cellcolor[gray]{0.92} \\
		& \cellcolor[gray]{0.95} & \cellcolor[gray]{0.97} \\
		& \cellcolor[gray]{0.89} & \cellcolor[gray]{0.92} \\
		& \cellcolor[gray]{0.95} & \cellcolor[gray]{0.97} \\
		& \cellcolor[gray]{0.89} & \cellcolor[gray]{0.92} \\
		& \cellcolor[gray]{0.95} & \cellcolor[gray]{0.97} \\
		\cmidrule{2-3}
		\bottomrule
		\end{tabular}
	\end{center}
	\caption{Dados para as leituras de tensão em função da posição.}
\end{table}
