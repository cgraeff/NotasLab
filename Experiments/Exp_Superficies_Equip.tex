%%%%%%%%%%%%%%%%%%%%%%%%%%%%%%%%%%%%%%%%%%%%%%%%%%%%%%%%%%%%%%%%%%%%%%%%%%%%%%%
\chapter{Superfícies Equipotenciais} % Sem "Experiência 01" ou qualquer outro número
\label{Chap:SupEquip}        % para poder trocar a ordem com facilidade
%%%%%%%%%%%%%%%%%%%%%%%%%%%%%%%%%%%%%%%%%%%%%%%%%%%%%%%%%%%%%%%%%%%%%%%%%%%%%%%

\begin{fullwidth}\it
	Faremos um experimentos para determinar as superfícies equipotenciais de um arranjo de dipolar cargas elétricas. Através delas, seremos capazes de determinar a direção das linhas de campo elétrico.
\end{fullwidth}

%%%%%%%%%%%%%%%%%%%%%%%%%%%%%%%%%%%%%%%%%%%%%%%%%%%%%%%%%%%%%%%%%%%%%%%%%%%%%%%
\section{Potencial elétrico de um arranjo dipolar de cargas elétricas}
%%%%%%%%%%%%%%%%%%%%%%%%%%%%%%%%%%%%%%%%%%%%%%%%%%%%%%%%%%%%%%%%%%%%%%%%%%%%%%%

Começar de energia potencial, depois definir o potencial elétrico, depois falar o que é um dipolo.

%%%%%%%%%%%%%%%%%%%%%%%%%%%%%%%%%%%%%%%%%%%%%
%\subsection{Se necessário, usar subsections}
%%%%%%%%%%%%%%%%%%%%%%%%%%%%%%%%%%%%%%%%%%%%%

%%%%%%%%%%%%%%%%%%%%%%%%%%%%%%%%%%%%%%%%%%%%%%%%%%%%%%%%%%%%%%%%%%%%%%%%%%%%%%%
\section{Experimento}
%%%%%%%%%%%%%%%%%%%%%%%%%%%%%%%%%%%%%%%%%%%%%%%%%%%%%%%%%%%%%%%%%%%%%%%%%%%%%%%

%%%%%%%%%%%%%%%%%%%%%%
\subsection{Objetivos}
%%%%%%%%%%%%%%%%%%%%%%

\begin{itemize}
	\item Obter as linhas equipotenciais de um dipolo elétrico em um plano que contém as cargas;
	\item Obter as linhas de campo elétrico a partir das linhas de potencial elétrico;
	\item Obter um gráfico do potencial ao longo da linha reta que une as cargas.
\end{itemize}

%%%%%%%%%%%%%%%%%%%%%%%%%%%%%%%%%%%%%%%%%%%%%%%%%%%%%%%%%%%%%%%%%%%%%%%%%%%%%%%
\section{Material Necessário}
%%%%%%%%%%%%%%%%%%%%%%%%%%%%%%%%%%%%%%%%%%%%%%%%%%%%%%%%%%%%%%%%%%%%%%%%%%%%%%%

\begin{itemize}
	\item Papel milimetrado;
	\item Fonte regulável;
	\item Multímetro;
	\item Placa de Petri com água;
	\item Eletrodos com suporte;
	\item Cabos de ligação (1 ponta de prova, 3 banana-jacaré);
	\item Becker com água.
\end{itemize}

%%%%%%%%%%%%%%%%%%%%%%%%%%%%%%%%%%%%%%%%%%%%%%%%%%%%%%%%%%%%%%%%%%%%%%%%%%%%%%%
\section{Procedimento Experimental}
%%%%%%%%%%%%%%%%%%%%%%%%%%%%%%%%%%%%%%%%%%%%%%%%%%%%%%%%%%%%%%%%%%%%%%%%%%%%%%%

%%%%%%%%%%%%%%%%%%%%%
%\subsection{Parte A} % Se necessário
%%%%%%%%%%%%%%%%%%%%%
\begin{enumerate}
	\item Tome duas folhas de papel milimetrado e desenhe sistemas de referência cujas origens se encontrem no centro e cujas marcas estejam distanciadas de 1 em \np[cm]{1,00}. Ambos os eixos devem variar de \np[cm]{-5,00} a \np[cm]{+5,00};
	\item Disponha a placa de Petri sobre o papel milimetrado de forma que seu centro coincida com a origem do sistema de referência;
	\item Ligue os eletrodos à fonte usando os cabos fornecidos;
	\item Disponha os eletrodos na placa de Petri de forma que eles estejam localizados sobre o eixo $x$, a uma distância de \np[cm]{5,00} da origem;
	\item Ajuste o multímetro para a função de voltímetro e ligue o terminal comum ao eletrodo negativo usando o cabo fornecido;
	\item Ligue a ponta de prova ao terminal marcado com \texttt{V$\mathdirectcurrent$} do multímetro;
	\item Ligue a fonte regulável e a ajuste para uma tensão de \np[V]{4,0};
	\item Coloque água na placa de Petri;
	\item Com a ponta de prova do multímetro, afira a tensão na origem do sistema de coordenadas e a cada \np[cm]{1,00} ao longo do eixo $y$ (tanto para valores positivos, quanto negativos). Na outra folha de papel milimetrado, marque um ponto na posição correspondente à da medida e anote os valores de tensão ao lado dos pontos;
	\item Afira a tensão ao longo do eixo $x$ desde o eletrodo negativo até o positivo a cada \np[cm]{1,00}.\footnote{Use as próprias marcas do eixo $x$ e os pontos intermediários entre elas, assim os valores ficarão mais simples e a origem do sistema de referência também estará incluída nos dados.} Anote os valores obtidos na Tabela~\ref{Tab:ValoresPotencialEletricoEixoX};
	\item Para a coluna $x = \np[cm]{-4,00}$ inicie com a ponta de prova em $y = \np[cm]{-5,00}$ e busque os valores de tensão dos pontos aferidos sobre o eixo $y$;
	\item Repita o procedimento do item acima varrendo todas as colunas até $x = \np[cm]{+4,0}$.
\end{enumerate}

%%%%%%%%%%%%%%%%%%%%%%%%%%%%%%%%%%%%%%%%%%%%%%%%%%%%%%%%%%%%%%%%%%%%%%%%%%%%%%%
%%%%%%%%%%%%%%%%%%%%%%%%%%%%%%%%%%%%%%%%%%%%%%%%%%%%%%%%%%%%%%%%%%%%%%%%%%%%%%%
%%%%%%%%%%%%%%%%%%%%%%%%%%%%%%%%%%%%%%%%%%%%%%%%%%%%%%%%%%%%%%%%%%%%%%%%%%%%%%%
%%%%%%%%%%%%%%%%%%%%%%%%%%%%%%%%%%%%%%%%%%%%%%%%%%%%%%%%%%%%%%%%%%%%%%%%%%%%%%%
\cleardoublepage

\noindent{}{\huge\textit{Superfícies equipotenciais}}

\vspace{15mm}

\begin{fullwidth}
\noindent{}\makebox[0.6\linewidth]{Turma:\enspace\hrulefill}\makebox[0.4\textwidth]{  Data:\enspace\hrulefill}
\vspace{5mm}

\noindent{}\makebox[0.6\linewidth]{Aluno(a):\enspace\hrulefill}\makebox[0.4\textwidth]{  Matrícula:\enspace\hrulefill}

\noindent{}\makebox[0.6\linewidth]{Aluno(a):\enspace\hrulefill}\makebox[0.4\textwidth]{  Matrícula:\enspace\hrulefill}

\noindent{}\makebox[0.6\linewidth]{Aluno(a):\enspace\hrulefill}\makebox[0.4\textwidth]{  Matrícula:\enspace\hrulefill}

\noindent{}\makebox[0.6\linewidth]{Aluno(a):\enspace\hrulefill}\makebox[0.4\textwidth]{  Matrícula:\enspace\hrulefill}

\noindent{}\makebox[0.6\linewidth]{Aluno(a):\enspace\hrulefill}\makebox[0.4\textwidth]{  Matrícula:\enspace\hrulefill}
\end{fullwidth}

\vspace{5mm}

%%%%%%%%%%%%%%%%%%%%%%%%%%%%%%%%%%%%%%%%%%%%%%%%%%%%%%%%%%%%%%%%%%%%%%%%%%%%%%%
\section{Questionário}
%%%%%%%%%%%%%%%%%%%%%%%%%%%%%%%%%%%%%%%%%%%%%%%%%%%%%%%%%%%%%%%%%%%%%%%%%%%%%%%

\begin{question}[type={exam}]{2}
Preencha as colunas de dados experimentais das tabelas com o número adequado de algarismos significativos e unidades.
\end{question}

\begin{question}[type={exam}]{3}
Ligue os pontos equipotenciais obtidos de uma maneira suave. Considerando que as superfícies equipotenciais são sempre perpendiculares às linhas de campo elétrico, desenhe as linhas de campo, de uma maneira estimada.
\end{question}

\begin{question}[type={exam}]{7}
Faça um gráfico do potencial elétrico em função da posição $x$ para os dados contidos na Tabela~\ref{Tab:ValoresPotencialEletricoEixoX}.
\end{question}

\begin{question}[type={exam}]{2}
Estime o campo elétrico ao longo do eixo $x$ através de
\begin{equation}
    E = -\frac{\partial V}{\partial x} \approx \frac{\Delta V}{\Delta x}.
\end{equation}
%
Faça um gráfico de $E(x)$ para o resultado obtido.
\end{question}

\vfill
%%%%%%%%%%%%%%%%%%%%%%%%%%%%%%%%%%%%%%%%%%%%%%%%%%%%%%%%%%%%%%%%%%%%%%%%%%%%%%%
\pagebreak
\section{Tabelas}
%%%%%%%%%%%%%%%%%%%%%%%%%%%%%%%%%%%%%%%%%%%%%%%%%%%%%%%%%%%%%%%%%%%%%%%%%%%%%%%


\begin{table}
\label{Tab:ValoresPotencialEletricoEixoX}
	\begin{center}
		\begin{tabular}{cp{25mm}p{25mm}c}
		\toprule
		&\multicolumn{2}{l}{\textbf{Dados para o potencial.}} \\
		\cmidrule{2-3}
		& $x$ & $V$ \\
		\cmidrule{2-3}
		& \cellcolor[gray]{0.89} & \cellcolor[gray]{0.92} & \\
		& \cellcolor[gray]{0.95} & \cellcolor[gray]{0.97} \\
		& \cellcolor[gray]{0.89} & \cellcolor[gray]{0.92} \\
		& \cellcolor[gray]{0.95} & \cellcolor[gray]{0.97} \\
		& \cellcolor[gray]{0.89} & \cellcolor[gray]{0.92} \\
		& \cellcolor[gray]{0.95} & \cellcolor[gray]{0.97} \\
		& \cellcolor[gray]{0.89} & \cellcolor[gray]{0.92} \\
		& \cellcolor[gray]{0.95} & \cellcolor[gray]{0.97} \\
		& \cellcolor[gray]{0.89} & \cellcolor[gray]{0.92} \\
		& \cellcolor[gray]{0.95} & \cellcolor[gray]{0.97} \\
		& \cellcolor[gray]{0.89} & \cellcolor[gray]{0.92} \\
		& \cellcolor[gray]{0.95} & \cellcolor[gray]{0.97} \\
		& \cellcolor[gray]{0.89} & \cellcolor[gray]{0.92} \\
		& \cellcolor[gray]{0.95} & \cellcolor[gray]{0.97} \\
		& \cellcolor[gray]{0.89} & \cellcolor[gray]{0.92} \\
		& \cellcolor[gray]{0.95} & \cellcolor[gray]{0.97} \\
		\cmidrule{2-3}
		\bottomrule
		\end{tabular}
	\end{center}
	\caption{Dados para as leituras de tensão em função da posição.}
\end{table}
