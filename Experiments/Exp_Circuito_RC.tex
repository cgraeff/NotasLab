%%%%%%%%%%%%%%%%%%%%%%%%%%%%%%%%%%%%%%%%%%%%%%%%%%%%%%%%%%%%%%%%%%%%%%%%%%%%%%%
\chapter{Circuito RC} % Sem "Experiência 01" ou qualquer outro número
\label{Chap:CircuitoRC}        % para poder trocar a ordem com facilidade
%%%%%%%%%%%%%%%%%%%%%%%%%%%%%%%%%%%%%%%%%%%%%%%%%%%%%%%%%%%%%%%%%%%%%%%%%%%%%%%

\begin{fullwidth}\it
	Realizaremos um experimento de carregamento e descarregamento de um capacitor com o objetivo de observar o comportamento de um sistema onde a corrente não é constante.
\end{fullwidth}

%%%%%%%%%%%%%%%%%%%%%%%%%%%%%%%%%%%%%%%%%%%%%%%%%%%%%%%%%%%%%%%%%%%%%%%%%%%%%%%
\section{Circuito RC}
%%%%%%%%%%%%%%%%%%%%%%%%%%%%%%%%%%%%%%%%%%%%%%%%%%%%%%%%%%%%%%%%%%%%%%%%%%%%%%%

Os circuitos formados por uma fonte de tensão, um resistor, e um capacitor, ou por um resistor e um capacitor carregado, são os exemplos mais simples de circuitos onde a corrente não é constante. Verificaremos abaixo como determinar a carga do capacitor em cada uma dessas situações como funções do tempo.

%%%%%%%%%%%%%%%%%%%%%%%%%%%%%%%%%%%%%%%%%%%%
\subsection{Cargarregamento de um capacitor}
%%%%%%%%%%%%%%%%%%%%%%%%%%%%%%%%%%%%%%%%%%%%

\begin{marginfigure}
\centering
\begin{circuitikz}[american]
	\draw (0,0) to[battery1, v=$V$] (0,3) to[normal open switch] (3,3) to[R, l=$R$] (3,0) to[C, l=$C$] (0,0);
\end{circuitikz}
\caption{Circuito $RC$ para a análise do processo de carregamento do capacitor.\label{Fig:CircuitoRCCarga}}
\end{marginfigure}


Podemos determinar a corrente como uma função do tempo para o primeiro caso considerando que o capacitor está inicialmente descarregado e aplicando a lei de Kirchhoff ao circuito mostrado na Figura~\ref{Fig:CircuitoRCCarga}:
\begin{equation}
	V_b - \frac{q}{C} - iR = 0,
\end{equation}
%
onde $V_b$ representa a tensão da bateria, $R$ a resistência do resistor, $q$ a carga contida no capacitor, $C$ a capacitância do capacitor, e $i$ representa a corrente que flui no circuito.

Como
\begin{equation}
	i = \frac{dq}{dt},
\end{equation}
%
podemos escrever
\begin{equation}
	V_b - \frac{q}{C} - R\frac{dq}{dt} = 0.
\end{equation}
%
Reordenando os termos, obtemos
\begin{align}
	\frac{dq}{dt} &= \frac{V_b}{R} - \frac{q}{RC}\\
	&= \frac{V_bC - q}{RC} \\
	&= - \frac{q - V_bC}{RC}
\end{align}
%
Podemos reescrever a equação acima e integrar os dois lados:
\begin{align}
	\frac{dq}{q - V_bC} &= -\frac{1}{RC} dt \\
	\int_0^q \frac{dq}{q - V_bC} &= -\frac{1}{RC} \int_0^t dt \\
	\int \frac{dq'}{q'} &= -\frac{1}{RC} \\
	\ln(q') + \alpha &= -\frac{1}{RC} \\
	\left[\ln(q - V_bC) + \alpha\right]_0^q &= -\frac{t}{RC} \\
	\ln\left(\frac{q - V_bC}{-V_bC}\right) &= -\frac{t}{RC}.
\end{align}
%
Tomando a exponencial de ambos os lados, resulta em
\begin{align}
	\frac{q-V_bC}{-V_bC} &= \exp\left(-\frac{t}{RC}\right) \\
	q - V_bC &= -V_bC \exp\left(-\frac{t}{RC}\right) \\
	q &= V_bC - V_bC \exp\left(-\frac{t}{RC}\right) \\
	q(t) &= V_bC \cdot \left[1 - \exp\left(-\frac{t}{RC}\right)\right].
\end{align}
%
Note que a tensão em um capacitor está relacionada à sua carga através de
\begin{equation}
    V_C = \frac{q}{C}.
\end{equation}

\begin{marginfigure}
\centering
\begin{tikzpicture}[>=Stealth, yscale = 1.5, extended line/.style={shorten >=-#1,shorten <=-#1},
 extended line/.default=3mm]] 
    \draw[->] (0,0) -- (0,1.75) node[below left] {$V$};
	\draw[->] (0,0) -- (4,0) node[below left] {$t$};

    % Desenhar função:
    \draw[smooth, name path=plot,samples=1000,domain=0:3.5]
    plot(\x,{1 * (1 - exp(-\x/1))});
    
    \draw[dotted] (0,1) node[left]{$V_b$} -- (3.5,1);

	\end{tikzpicture}
\caption{Gráfico da tensão como função do tempo registrada no capacitor durante o processo de carregamento.\label{Fig:GraficoCargaCapacitor}}
\end{marginfigure}

\noindent{}Se dividirmos a expressão para $q(t)$ por $C$ de ambos os lados da equação, obtemos finalmente
\begin{equation}
    V_C(t) = V_b \cdot \left[1 - \exp\left(-\frac{t}{RC}\right)\right].
\end{equation}
%
Fazendo um gráfico da expressão acima, obtemos a Figura~\ref{Fig:GraficoCargaCapacitor}. Note que no processo de carregamento a tensão registrada no capacitor tende à tensão $V$ da bateria. 


%%%%%%%%%%%%%%%%%%%%%%%%%%%%%%%%%%%%%%%%%
\section{Descarregamento de um capacitor}
%%%%%%%%%%%%%%%%%%%%%%%%%%%%%%%%%%%%%%%%%

\begin{marginfigure}[4cm]
\centering
\begin{circuitikz}[american]
	\draw (0,0) -- (0,3) to[normal open switch] (3,3) to[R, l=$R$] (3,0) to[C, l=$C$] (0,0);
\end{circuitikz}
\caption{Circuito $RC$ para a análise do processo de descarregamento do capacitor.\label{Fig:CircuitoRCDescarga}}
\end{marginfigure}

Para o processo de descarregamento do capacitor ao aplicarmos a lei de Kirchhoff temos que
\begin{equation}
    -\frac{q}{C} - i R = 0.
\end{equation}
%
Novamente, usando $i = dq/dt$, e reorganizando os termos obtemos
\begin{align}
    R\frac{dq}{dt} + \frac{q}{C} &= 0 \\
    \frac{dq}{dt} + \frac{q}{RC} &= 0 \\
    \frac{dq}{dt} &= - \frac{q}{RC} \\
    \frac{dq}{q} &= - \frac{dt}{RC}.
\end{align}
%
Integrando ambos os lados da equação acima, temos
\begin{align}
    \int_{q_i}^{q_f} \frac{dq}{q} &= -\frac{1}{RC} \int_{t_i}^{t_f} dt \\
    \left[\ln q + C\right]_{q_i}^{q_f} &= - \frac{q}{RC} \left[t + C\right]_{t_i}^{t_f} \\
    \ln \frac{q_f}{q_i} &= -\frac{\Delta t}{RC} \\
    \frac{q_f}{q_i} &= \exp\left(-\frac{\Delta t}{RC}\right) \\
    q_f &= q_i \exp\left(-\frac{\Delta t}{RC}\right).
\end{align}
%
Se fizermos $t_ = 0$, $t_f = t$, e $q_i = q_0$, podemos representar a carga como uma função do tempo através de
\begin{equation}
    q(t) = q_0 \exp\left(-\frac{t}{RC}\right).
\end{equation}

\begin{marginfigure}
\centering
\begin{tikzpicture}[>=Stealth, yscale = 1.5, extended line/.style={shorten >=-#1,shorten <=-#1},
 extended line/.default=3mm]] 
    \draw[->] (0,0) -- (0,1.75) node[below left] {$V$};
	\draw[->] (0,0) -- (4,0) node[below left] {$t$};

    % Desenhar função:
    \draw[smooth, name path=plot,samples=1000,domain=0:3.5]
    plot(\x,{1 * exp(-\x/1)}) node[above]{$V_C(t)$};
    
    \draw[dotted] (0,1) node[left]{$V_0$} -- (3.5,1);

	\end{tikzpicture}
\caption{Gráfico da tensão como função do tempo registrada no capacitor durante o processo de descarregamento.\label{Fig:GraficoDescargaCapacitor}}
\end{marginfigure}

\noindent{}Novamente, se dividirmos ambos os lados da equação por $C$, obtemos a tensão no capacitor:
\begin{equation*}
    V_C(t) = V_0 \exp\left(-\frac{t}{RC}\right).
\end{equation*}

%%%%%%%%%%%%%%%%%%%%%%%%%%%%%%%%%%%%%%%%%%%%
\subsection{Objetivos}
\label{Sec:ObjetivosCargaEDescargaCapacitor}
%%%%%%%%%%%%%%%%%%%%%%%%%%%%%%%%%%%%%%%%%%%%

\begin{itemize}
	\item Verificar a validade da expressão para a tensão durante o carregamento do capacitor;
	\item Verificar a validade da expressão para a tensão durante o descarregamento do capacitor;
	\item Determinar o valor do produto da resistência e capacitância do circuito, isto é, o valor de $RC$, a partir dos dados experimentais;
\end{itemize}

%%%%%%%%%%%%%%%%%%%%%%%%%%%%%%%%%%%%%%%%%%%%%%%%%%%%%%%%%%%%%%%%%%%%%%%%%%%%%%%
\section{Material Necessário}
%%%%%%%%%%%%%%%%%%%%%%%%%%%%%%%%%%%%%%%%%%%%%%%%%%%%%%%%%%%%%%%%%%%%%%%%%%%%%%%

\begin{itemize}
	\item Fonte de tensão regulável;
	\item Protoboard;
	\item Resistores;
	\item Capacitores eletrolíticos;
	\item Multímetro;
	\item Cabos para ligações;
	\item Cronômetro.
\end{itemize}

%%%%%%%%%%%%%%%%%%%%%%%%%%%%%%%%%%%%%%%%%%%%%%%%%%%%%%%%%%%%%%%%%%%%%%%%%%%%%%%
\section{Procedimento Experimental}
%%%%%%%%%%%%%%%%%%%%%%%%%%%%%%%%%%%%%%%%%%%%%%%%%%%%%%%%%%%%%%%%%%%%%%%%%%%%%%%

%%%%%%%%%%%%%%%%%%%%%%%%%%%%%%%%%%%%%%%%%
\subsection{Carregamento de um capacitor}
%%%%%%%%%%%%%%%%%%%%%%%%%%%%%%%%%%%%%%%%%

\begin{marginfigure}[2cm]
\centering
\begin{circuitikz}[american]
	\draw (0,0) to[battery1, v=$V$, i=$i$] (0,3) to[switch] (2,3) to[eC, l=$C$] (2,0) to[R, l=$R$] (0,0);
	\draw (2,3) -- (3.5,3) to[smeter, t=V] (3.5,0) -- (2,0);
\end{circuitikz}
\caption{Carga de um capacitor eletrolítico.}
\end{marginfigure}

{\it
\textbf{Atenção:} Não ligue a fonte de tensão ao circuito antes de se certificar que as ligações estejam corretas, pois há risco de o capacitor ser danificado, entrar em combustão, e mesmo explodir.\footnote{Provavelmente o risco de ferimentos é menor do que o de se queimar no experimento de dilatação térmica, mas mesmo assim, tomem cuidado.} Verifique o valor máximo de tensão que o capacitor suporta e nunca exceda esse valor.
}
	
\begin{enumerate}
    \item Conecte o capacitor ao protoboard;
	\item \textbf{Atenção:} Note que o capacitor eletrolítico tem seus terminais marcados como `$+$' e `$-$', ou pelo menos um dos terminais é marcado. Conecte um ou mais resistores em série ao protoboard de maneira que um dos terminais esteja na mesma linha condutora que o terminal \emph{negativo}\footnote{Tanto faz, na verdade, mas nos próximos passos eu vou assumir que ele esteja ligado ao terminal negativo.} do capacitor;
	\item Anote o valor total de resistência na Tabela~\ref{Tab:CargaEDescargaCapacitor}, na seção de valores nominais.
	\item Use os cabos banana-jacaré para ligar o multímetro aos terminais do capacitor (iremos medir valores de tensão em corrente contínua);
	\item \textbf{Atenção:} Vamos preparar a ligação do circuito à fonte. Como estamos usando um capacitor eletrolítico, não podemos ligá-lo com as polaridades invertidas, pois isso fará com que o componente seja destruído.
	    \begin{enumerate}
	        \item Ligue um cabo banana-jacaré do terminal negativo da fonte ao terminal livre do resistor;
	        \item Ligue um cabo banana-jacaré ao terminal positivo do capacitor, deixando a outra extremidade desconectada. Adiante ela será ligada ao terminal positivo da fonte.
        \end{enumerate}
    \item Ligue a fonte. Ajuste os limitadores de corrente para os seus valores máximos e ajuste a tensão para um valor de \np[V]{5,0};
	\item Ajuste o multímetro na função voltímetro de corrente contínua, escolhendo uma escala que permita medir o valor de tensão ajustado na fonte regulável.
	\item Zere o cronômetro;
    \item Na primeira linha da Tabela~\ref{Tab:CargaEDescargaCapacitor}, na seção de carregamento, anote o valor de tempo de \np[s]{0} e o valor correspondente de tensão de \np[V]{0}; 
    \item Prepare-se para anotar os valores de tempo e de tensão a cada \np[s]{15,0};
	\item Início do processo de carregamento:\footnote{Os valores de tempo sugeridos foram calculados com base em um valor de $RC$ de aproximadamente \np[s]{140}.}
	    \begin{enumerate}
	        \item Ao mesmo tempo que a extremidade livre do cabo banana-jacaré ligado ao terminal positivo do capacitor é conectado ao terminal positivo da fonte, acione o cronômetro.
	        \item Quando o cronômetro atingir \np[s]{15,0}, registre o valor de tempo $t$ registrado pelo cronômetro e o valor de tensão $V_C$ registrado pelo multímetro na Tabela~\ref{Tab:CargaEDescargaCapacitor};
	        \item Continue anotando os valores de $t$ e $V_C$ a cada \np[s]{15,0}, anotando-os na tabela, interrompendo as medidas após \np[s]{300}.
	        \item A finalizar o carregamento, remova os cabos dos terminais do resistor e do capacitor. Tome cuidado para que nenhum condutor toque o circuito pois vamos utilizar o capacitor carregado no processo de descarga na seção seguinte.
	        \item Desconecte os cabos da fonte e a desligue.
        \end{enumerate}
\end{enumerate}

%%%%%%%%%%%%%%%%%%%%%%%%%%%%%%%%%%%%%%%%%
\subsection{Descarregamento de um capacitor}
%%%%%%%%%%%%%%%%%%%%%%%%%%%%%%%%%%%%%%%%%

\begin{marginfigure}
\centering
\begin{circuitikz}[american]
	\draw (0,0) -- (0,3) to[switch] (2,3) to[eC, l=$C$, i<=$i$] (2,0) to[R, l=$R$] (0,0);
	\draw (2,3) -- (3.5,3) to[smeter, t=V] (3.5,0) -- (2,0);
\end{circuitikz}
\caption{Descarga de um capacitor eletrolítico.}
\end{marginfigure}

{\it
Ao final do procedimento da seção anterior, o capacitor estará carregado, ainda que parcialmente. Vamos agora realizar o descarregamento.
}
\begin{enumerate}
	\item Zere o cronômetro;
	\item Conecte um cabo jacaré-jacaré ao terminal do resistor;
	\item Na primeira linha da Tabela~\ref{Tab:CargaEDescargaCapacitor}, na seção de descarregamento, anote o valor de tempo de \np[s]{0} e o valor correspondente de tensão lido no multímetro;\footnote{Como o multímetro não é ideal, a tensão do capacitor sempre estará diminuindo, uma vez que existe uma pequana corrente. Procure não demorar para iniciar o processo de descarga após verificar o valor inicial de tensão.}
	\item Prepare-se para anotar os valores de tempo e de tensão a cada \np[s]{15};
	\item Início do processo de descarregamento:
	\begin{enumerate}
	    \item Ao mesmo tempo que a extremidade livre do cabo é ligada ao terminal positivo do capacitor, acione o cronômetro;
	    \item Quando o cronômetro atingir \np[s]{15}, registre o valor de tempo $t$ registrado pelo cronômetro e o valor de tensão $V_C$ registrado pelo multímetro na Tabela~\ref{Tab:CargaEDescargaCapacitor};
        \item Continue anotando os valores de $t$ e $V_C$ a cada \np[s]{15}, anotando-os na tabela, interrompendo as medidas após \np[s]{300}.
    \end{enumerate}
\end{enumerate}

%%%%%%%%%%%%%%%%%%%%%%%%%%%%%%%%%%%%%%%%%%%%%%%%%%%%%%%%%%%%%%%%%%%%%%%%%%%%%%%
%%%%%%%%%%%%%%%%%%%%%%%%%%%%%%%%%%%%%%%%%%%%%%%%%%%%%%%%%%%%%%%%%%%%%%%%%%%%%%%
%%%%%%%%%%%%%%%%%%%%%%%%%%%%%%%%%%%%%%%%%%%%%%%%%%%%%%%%%%%%%%%%%%%%%%%%%%%%%%%
%%%%%%%%%%%%%%%%%%%%%%%%%%%%%%%%%%%%%%%%%%%%%%%%%%%%%%%%%%%%%%%%%%%%%%%%%%%%%%%
\cleardoublepage

\noindent{}{\huge\textit{Circuito RC}}

\vspace{15mm}

\begin{fullwidth}
\noindent{}\makebox[0.6\linewidth]{Turma:\enspace\hrulefill}\makebox[0.4\textwidth]{  Data:\enspace\hrulefill}
\vspace{5mm}

\noindent{}\makebox[0.6\linewidth]{Aluno(a):\enspace\hrulefill}\makebox[0.4\textwidth]{  Matrícula:\enspace\hrulefill}

\noindent{}\makebox[0.6\linewidth]{Aluno(a):\enspace\hrulefill}\makebox[0.4\textwidth]{  Matrícula:\enspace\hrulefill}

\noindent{}\makebox[0.6\linewidth]{Aluno(a):\enspace\hrulefill}\makebox[0.4\textwidth]{  Matrícula:\enspace\hrulefill}

\noindent{}\makebox[0.6\linewidth]{Aluno(a):\enspace\hrulefill}\makebox[0.4\textwidth]{  Matrícula:\enspace\hrulefill}

\noindent{}\makebox[0.6\linewidth]{Aluno(a):\enspace\hrulefill}\makebox[0.4\textwidth]{  Matrícula:\enspace\hrulefill}
\end{fullwidth}

\vspace{5mm}

%%%%%%%%%%%%%%%%%%%%%%%%%%%%%%%%%%%%%%%%%%%%%%%%%%%%%%%%%%%%%%%%%%%%%%%%%%%%%%%
\section{Questionário}
%%%%%%%%%%%%%%%%%%%%%%%%%%%%%%%%%%%%%%%%%%%%%%%%%%%%%%%%%%%%%%%%%%%%%%%%%%%%%%%

\begin{question}[type={exam}]{2}
Preencha as colunas de dados experimentais das tabelas com o número adequado de algarismos significativos e unidades.
\end{question}

\begin{question}[type={exam}]{3}
Para o carregamento do capacitor:
    \begin{enumerate}[label=\roman*.]
        \item Elabore um gráfico de tensão em função do tempo para o capacitor.
        \item Faça um ajuste dos dados experimentais e obtenha o valor do produto $RC$. Você pode optar por fazer algum tipo de mudança de variáveis, se necessário. Nesse caso, faça um gráfico mostrando o resultado considerando a mudança de variáveis adotada.
    \end{enumerate}
\end{question}

\begin{question}[type={exam}]{3}
Para o descarregamento do capacitor:
    \begin{enumerate}[label=\roman*.]
        \item Elabore um gráfico de tensão em função do tempo para o capacitor.
        \item Faça um ajuste dos dados experimentais e obtenha o valor do produto $RC$. Você pode optar por fazer algum tipo de mudança de variáveis, se necessário. Nesse caso, faça um gráfico mostrando o resultado considerando a mudança de variáveis adotada.
    \end{enumerate}
\end{question}

\begin{question}[type={exam}]{2}
Considerando os objetivos do experimento, listados na Seção~\ref{Sec:ObjetivosCargaEDescargaCapacitor}, e os resultados obtidos nas questões anteriores, discuta quais objetivos foram atingidos com sucesso, justificando suas conclusões. Se algum objetivo não foi atingido, discuta quais são os possíveis motivos do insucesso e que providências podem ser tomadas para que eles sejam alcançados.
\end{question}

\vfill
%%%%%%%%%%%%%%%%%%%%%%%%%%%%%%%%%%%%%%%%%%%%%%%%%%%%%%%%%%%%%%%%%%%%%%%%%%%%%%%
\pagebreak
\section{Tabelas}
%%%%%%%%%%%%%%%%%%%%%%%%%%%%%%%%%%%%%%%%%%%%%%%%%%%%%%%%%%%%%%%%%%%%%%%%%%%%%%%

\begin{table*}
	\begin{center}
		\begin{tabular}{lp{25mm}p{25mm}lp{25mm}p{25mm}l}
		\toprule
		& \multicolumn{2}{l}{\textbf{Valores nominais}} \\
		\cmidrule{2-3}
		& $R$ \cellcolor[gray]{0.89} & \cellcolor[gray]{0.92} \\
		& $C$ \cellcolor[gray]{0.95} & \cellcolor[gray]{0.97} \\
		& $V_b$ \cellcolor[gray]{0.89} & \cellcolor[gray]{0.92} \\
		\cmidrule{2-3}
\\
		& \multicolumn{2}{l}{\textbf{Carregamento}} & & \multicolumn{2}{l}{\textbf{Descarregamento}} \\
		\cmidrule{2-3} \cmidrule{5-6}
		& $t$ & $V_C$ & & $t$ & $V_C$ \\
		\cmidrule{2-3} \cmidrule{5-6}
		& \cellcolor[gray]{0.89} & \cellcolor[gray]{0.92} & & \cellcolor[gray]{0.89} & \cellcolor[gray]{0.92} & \\
		& \cellcolor[gray]{0.95} & \cellcolor[gray]{0.97} & & \cellcolor[gray]{0.95} & \cellcolor[gray]{0.97} & \\
		& \cellcolor[gray]{0.89} & \cellcolor[gray]{0.92} & & \cellcolor[gray]{0.89} & \cellcolor[gray]{0.92} & \\
		& \cellcolor[gray]{0.95} & \cellcolor[gray]{0.97} & & \cellcolor[gray]{0.95} & \cellcolor[gray]{0.97} & \\
		& \cellcolor[gray]{0.89} & \cellcolor[gray]{0.92} & & \cellcolor[gray]{0.89} & \cellcolor[gray]{0.92} & \\
		& \cellcolor[gray]{0.95} & \cellcolor[gray]{0.97} & & \cellcolor[gray]{0.95} & \cellcolor[gray]{0.97} & \\
		& \cellcolor[gray]{0.89} & \cellcolor[gray]{0.92} & & \cellcolor[gray]{0.89} & \cellcolor[gray]{0.92} & \\
		& \cellcolor[gray]{0.95} & \cellcolor[gray]{0.97} & & \cellcolor[gray]{0.95} & \cellcolor[gray]{0.97} & \\
		& \cellcolor[gray]{0.89} & \cellcolor[gray]{0.92} & & \cellcolor[gray]{0.89} & \cellcolor[gray]{0.92} & \\
		& \cellcolor[gray]{0.95} & \cellcolor[gray]{0.97} & & \cellcolor[gray]{0.95} & \cellcolor[gray]{0.97} & \\
		& \cellcolor[gray]{0.89} & \cellcolor[gray]{0.92} & & \cellcolor[gray]{0.89} & \cellcolor[gray]{0.92} & \\
		& \cellcolor[gray]{0.95} & \cellcolor[gray]{0.97} & & \cellcolor[gray]{0.95} & \cellcolor[gray]{0.97} & \\
		& \cellcolor[gray]{0.89} & \cellcolor[gray]{0.92} & & \cellcolor[gray]{0.89} & \cellcolor[gray]{0.92} & \\
		& \cellcolor[gray]{0.95} & \cellcolor[gray]{0.97} & & \cellcolor[gray]{0.95} & \cellcolor[gray]{0.97} & \\
		& \cellcolor[gray]{0.89} & \cellcolor[gray]{0.92} & & \cellcolor[gray]{0.89} & \cellcolor[gray]{0.92} & \\
		& \cellcolor[gray]{0.95} & \cellcolor[gray]{0.97} & & \cellcolor[gray]{0.95} & \cellcolor[gray]{0.97} & \\
		& \cellcolor[gray]{0.89} & \cellcolor[gray]{0.92} & & \cellcolor[gray]{0.89} & \cellcolor[gray]{0.92} & \\
		& \cellcolor[gray]{0.95} & \cellcolor[gray]{0.97} & & \cellcolor[gray]{0.95} & \cellcolor[gray]{0.97} & \\
		& \cellcolor[gray]{0.89} & \cellcolor[gray]{0.92} & & \cellcolor[gray]{0.89} & \cellcolor[gray]{0.92} & \\
		& \cellcolor[gray]{0.95} & \cellcolor[gray]{0.97} & & \cellcolor[gray]{0.95} & \cellcolor[gray]{0.97} & \\
		& \cellcolor[gray]{0.89} & \cellcolor[gray]{0.92} & & \cellcolor[gray]{0.89} & \cellcolor[gray]{0.92} & \\
		& \cellcolor[gray]{0.95} & \cellcolor[gray]{0.97} & & \cellcolor[gray]{0.95} & \cellcolor[gray]{0.97} & \\
		& \cellcolor[gray]{0.89} & \cellcolor[gray]{0.92} & & \cellcolor[gray]{0.89} & \cellcolor[gray]{0.92} & \\
		& \cellcolor[gray]{0.95} & \cellcolor[gray]{0.97} & & \cellcolor[gray]{0.95} & \cellcolor[gray]{0.97} & \\
		& \cellcolor[gray]{0.89} & \cellcolor[gray]{0.92} & & \cellcolor[gray]{0.89} & \cellcolor[gray]{0.92} & \\
\bottomrule
		\end{tabular}
	\caption[][2mm]{Dados obtidos para o comprimento do elástico e a massa correspondente.}\label{Tab:CargaEDescargaCapacitor}
	\end{center}
\end{table*}

