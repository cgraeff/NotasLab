%%%%%%%%%%%%%%%%%%%%%%%%%%%%%%%%%%%%%%%%%%%%%%%%%%%%%%%%%%%%%%%%%%%%%%%%%%%%%%%
\chapter{Movimento retilíneo uniforme (MRU) e uniformemente variado (MRUV)}
\label{Chap:ExpMRUMRUV}
%%%%%%%%%%%%%%%%%%%%%%%%%%%%%%%%%%%%%%%%%%%%%%%%%%%%%%%%%%%%%%%%%%%%%%%%%%%%%%%

\begin{fullwidth}\it
	Realizaremos dois experimentos distintos de forma a verificar as caracteristicas do Movimento Retilíneo Uniforme -- MRU e do Movimento Retilíneo Uniformente Variado -- MRUV. No primeiro caso, verificaremos a velocidade de uma esfera que se move em um fluido, sendo que ela atingirá uma velocidade terminal constante\footnote[][15mm]{Veremos o conceito de \emph{velocidade terminal} em detalhes na Experiência~\ref{Chap:ExpArrasto}.}. No segundo realizaremos um experimento de queda livre. Utilizaremos os conceitos de medidas, algarismos significativos, e gráficos.
\end{fullwidth}

%%%%%%%%%%%%%%%%%%%%%%%%%%%%%%%%%%%%%%%%%%%%%%%%%%%%%%%%%%%%%%%%%%%%%%%%%%%%%%%
\section{Física do experimento}
%%%%%%%%%%%%%%%%%%%%%%%%%%%%%%%%%%%%%%%%%%%%%%%%%%%%%%%%%%%%%%%%%%%%%%%%%%%%%%%

Copiar das notas de aula.

%%%%%%%%%%%%%%%%%%%%%%%%%%%%%%%%%%%%%%%%%%%%%
%\subsection{Se necessário, usar subsections}
%%%%%%%%%%%%%%%%%%%%%%%%%%%%%%%%%%%%%%%%%%%%%

%%%%%%%%%%%%%%%%%%%%%%%%%%%%%%%%%%%%%%%%%%%%%%%%%%%%%%%%%%%%%%%%%%%%%%%%%%%%%%%
\section{Experimento}
%%%%%%%%%%%%%%%%%%%%%%%%%%%%%%%%%%%%%%%%%%%%%%%%%%%%%%%%%%%%%%%%%%%%%%%%%%%%%%%

\textbf{Breve descrição do experimento aqui. Não está claro em nenhum lugar o que os alunos vão efetivamente fazer.}

%%%%%%%%%%%%%%%%%%%%%%
\subsection{Objetivos}
%%%%%%%%%%%%%%%%%%%%%%

\begin{itemize}
     \item Verificar as características do Movimento Retilíneo Uniforme -- MRU através do gráfico do tempo em função do deslocamento.
     \item Verificar as características do Movimento Retilíneo Uniformemente Variado -- MRUV através do gráfico do tempo em função do deslocamento.
     \item Calcular a velocidade do MRU;
     \item Determinar a aceleração da gravidade no MRUV;
\end{itemize}

%%%%%%%%%%%%%%%%%%%%%%%%%%%%%%%%%%%%%%%%%%%%%%%%%%%%%%%%%%%%%%%%%%%%%%%%%%%%%%%
\section{Material Necessário}
%%%%%%%%%%%%%%%%%%%%%%%%%%%%%%%%%%%%%%%%%%%%%%%%%%%%%%%%%%%%%%%%%%%%%%%%%%%%%%%

\begin{itemize}
	\item Item 1;
	\item Item 2.
\end{itemize}

%%%%%%%%%%%%%%%%%%%%%%%%%%%%%%%%%%%%%%%%%%%%%%%%%%%%%%%%%%%%%%%%%%%%%%%%%%%%%%%
\section{Procedimento Experimental}
%%%%%%%%%%%%%%%%%%%%%%%%%%%%%%%%%%%%%%%%%%%%%%%%%%%%%%%%%%%%%%%%%%%%%%%%%%%%%%%

%%%%%%%%%%%%%%%%%%%%%%%%%%%%%%%%%%%%%%%%%%%%
\subsection{Procedimento Experimental - MRU}
%%%%%%%%%%%%%%%%%%%%%%%%%%%%%%%%%%%%%%%%%%%%

\begin{enumerate}
	\item Prepare o plano inclinado para com um ângulo de \np[\tcdegree]{30,0};
	\item Com um imã, atraia a esfera para a parte superior antes da posição \np[cm]{0,0};
	\item Solte a esfera e inicie o cronômetro quando a ela passar pela posição \np[cm]{0,0}. Pare o cronômetro quando a ela passar pela posição final \np[cm]{10,0}. Anote os valores na Tabela~\ref{DadosMRU};
	\item Repita este procedimento para os seguintes valores de posição final: \np[cm]{15}, \np[cm]{20}, \np[cm]{25}, \np[cm]{30}, \np[cm]{35}, \np[cm]{40}.
\end{enumerate}

%%%%%%%%%%%%%%%%%%%%%%%%%%%%%%%%%%%%%%%%%%%%
\subsection{Procedimento Experimental: MRUV}
%%%%%%%%%%%%%%%%%%%%%%%%%%%%%%%%%%%%%%%%%%%%

\begin{enumerate}
	\item Prenda a esfera no eletroimã;
	\item Posicione o primeiro sensor de forma que ele fique muito próximo de ativar o cronômetro (erga o sensor com o cronômetro ligado e, quando o sensor ativar, desça um pouco). Verifique a posição inicial do sensor e anote na Tabela~\ref{DadosMRUV};
	\item Posicione o segundo sensor abaixo do primeiro (aproximadamente \numprint[cm]{5,0}) e anote o valor na tabela;
	\item Desligue o eletroimã e anote o valor do tempo na tabela, incluindo o erro de escala do cronômetro;
	\item Repita a medida de tempo mais três vezes e faça a média.;
	\item Desloque o segundo sensor mais \np[cm]{5,0} e repita os passos acima. Desloque o sensor até completar a tabela.
\end{enumerate}

%%%%%%%%%%%%%%%%%%%%%%%%%%%%%%%%%%%%%%%%%%%%%%%%%%%%%%%%%%%%%%%%%%%%%%%%%%%%%%%
%%%%%%%%%%%%%%%%%%%%%%%%%%%%%%%%%%%%%%%%%%%%%%%%%%%%%%%%%%%%%%%%%%%%%%%%%%%%%%%
%%%%%%%%%%%%%%%%%%%%%%%%%%%%%%%%%%%%%%%%%%%%%%%%%%%%%%%%%%%%%%%%%%%%%%%%%%%%%%%
%%%%%%%%%%%%%%%%%%%%%%%%%%%%%%%%%%%%%%%%%%%%%%%%%%%%%%%%%%%%%%%%%%%%%%%%%%%%%%%
\cleardoublepage

\noindent{}{\huge\textit{Movimento retilíneo uniforme (MRU) e uniformemente variado (MRUV)}}

\vspace{15mm}

\begin{fullwidth}
\noindent{}\makebox[0.6\linewidth]{Turma:\enspace\hrulefill}\makebox[0.4\textwidth]{  Data:\enspace\hrulefill}
\vspace{5mm}

\noindent{}\makebox[0.6\linewidth]{Aluno(a):\enspace\hrulefill}\makebox[0.4\textwidth]{  Matrícula:\enspace\hrulefill}

\noindent{}\makebox[0.6\linewidth]{Aluno(a):\enspace\hrulefill}\makebox[0.4\textwidth]{  Matrícula:\enspace\hrulefill}

\noindent{}\makebox[0.6\linewidth]{Aluno(a):\enspace\hrulefill}\makebox[0.4\textwidth]{  Matrícula:\enspace\hrulefill}

\noindent{}\makebox[0.6\linewidth]{Aluno(a):\enspace\hrulefill}\makebox[0.4\textwidth]{  Matrícula:\enspace\hrulefill}

\noindent{}\makebox[0.6\linewidth]{Aluno(a):\enspace\hrulefill}\makebox[0.4\textwidth]{  Matrícula:\enspace\hrulefill}
\end{fullwidth}

\vspace{5mm}

%%%%%%%%%%%%%%%%%%%%%%%%%%%%%%%%%%%%%%%%%%%%%%%%%%%%%%%%%%%%%%%%%%%%%%%%%%%%%%%
\section{Questionário}
%%%%%%%%%%%%%%%%%%%%%%%%%%%%%%%%%%%%%%%%%%%%%%%%%%%%%%%%%%%%%%%%%%%%%%%%%%%%%%%
\emph{Nas questões seguintes, apresente os cálculos requisitados de maneira clara e sucinta, para que o professor possa acompanhar o raciocínio desenvolvido.}
\vspace{5mm}

\begin{question}[type={exam}]{2}
Preencha as colunas da Tabela~\ref{DadosMRU} com o número adequado de algarismos significativos, erros e unidades. Considere o erro do cronômetro manual como sendo \np[s]{0,2}.
\end{question}

\begin{question}[type={exam}]{1}
 Calcule a velocidade (última coluna da Tabela~\ref{DadosMRU}), incluindo o erro propagado. 
\end{question}

\begin{question}[type={exam}]{2.5}
Elabore um gráfico dos valores de \emph{tempo} registrados em função do \emph{distância} percorrida pela esfera contida no tubo do plano inclinado, isto é, com a distância no eixo $x$ e o tempo no eixo $y$.
\end{question}

\begin{question}[type={exam}]{1.5}
Preencha as colunas da Tabela~\ref{DadosMRUV} com o número adequado de algarismos significativos, erros e unidades.
\end{question}

\begin{question}[type={exam}]{1}
Utilizando as equações da cinemática, calcule a aceleração da gravidade, incluindo o erro (última coluna da Tabela~\ref{DadosMRUV}).
\end{question}

\begin{question}[type={exam}]{2.5}
Construa um gráfico do \emph{tempo} cronometrado em função da \emph{distância} percorrida pela esfera em queda livre.
\end{question}
\vfill
%%%%%%%%%%%%%%%%%%%%%%%%%%%%%%%%%%%%%%%%%%%%%%%%%%%%%%%%%%%%%%%%%%%%%%%%%%%%%%%
\pagebreak
\section{Tabelas}
%%%%%%%%%%%%%%%%%%%%%%%%%%%%%%%%%%%%%%%%%%%%%%%%%%%%%%%%%%%%%%%%%%%%%%%%%%%%%%%

\begin{table*}[!ht]
\centering
\begin{tabular}{lp{25mm}p{25mm}p{25mm}p{25mm}p{25mm}l}
\toprule
	&\multicolumn{4}{l}{\textbf{Dados Experimentais}} \\
	\cmidrule{2-6}
	& $x_0$ & $x_f$ & $t_1$ & $t_2$ & $t_3$ & \\
	\cmidrule{2-6}
	& \cellcolor[gray]{0.89} & \cellcolor[gray]{0.92} & \cellcolor[gray]{0.89} & \cellcolor[gray]{0.92} & \cellcolor[gray]{0.89} \\
	& \cellcolor[gray]{0.95} & \cellcolor[gray]{0.97} & \cellcolor[gray]{0.95} & \cellcolor[gray]{0.97} & \cellcolor[gray]{0.95} \\
	& \cellcolor[gray]{0.89} & \cellcolor[gray]{0.92} & \cellcolor[gray]{0.89} & \cellcolor[gray]{0.92} & \cellcolor[gray]{0.89} \\
	& \cellcolor[gray]{0.95} & \cellcolor[gray]{0.97} & \cellcolor[gray]{0.95} & \cellcolor[gray]{0.97} & \cellcolor[gray]{0.95} \\
	& \cellcolor[gray]{0.89} & \cellcolor[gray]{0.92} & \cellcolor[gray]{0.89} & \cellcolor[gray]{0.92} & \cellcolor[gray]{0.89} \\
	& \cellcolor[gray]{0.95} & \cellcolor[gray]{0.97} & \cellcolor[gray]{0.95} & \cellcolor[gray]{0.97} & \cellcolor[gray]{0.95} \\
	& \cellcolor[gray]{0.89} & \cellcolor[gray]{0.92} & \cellcolor[gray]{0.89} & \cellcolor[gray]{0.92} & \cellcolor[gray]{0.89} \\
	& \cellcolor[gray]{0.95} & \cellcolor[gray]{0.97} & \cellcolor[gray]{0.95} & \cellcolor[gray]{0.97} & \cellcolor[gray]{0.95} \\
	& \cellcolor[gray]{0.89} & \cellcolor[gray]{0.92} & \cellcolor[gray]{0.89} & \cellcolor[gray]{0.92} & \cellcolor[gray]{0.89} \\
	& \cellcolor[gray]{0.95} & \cellcolor[gray]{0.97} & \cellcolor[gray]{0.95} & \cellcolor[gray]{0.97} & \cellcolor[gray]{0.95} \\
	\cmidrule{2-6}
\\
	& \multicolumn{3}{l}{\textbf{Dados calculados}} \\
	\cmidrule{2-4}
	& $\Delta x$ & $\mean{t}$ & $\Delta x / \mean{t}$ \\
	\cmidrule{2-4}
	& \cellcolor[gray]{0.89} & \cellcolor[gray]{0.92} & \cellcolor[gray]{0.89} \\ 
	& \cellcolor[gray]{0.95} & \cellcolor[gray]{0.97} & \cellcolor[gray]{0.95} \\ 
	& \cellcolor[gray]{0.89} & \cellcolor[gray]{0.92} & \cellcolor[gray]{0.89} \\ 
	& \cellcolor[gray]{0.95} & \cellcolor[gray]{0.97} & \cellcolor[gray]{0.95} \\ 
	& \cellcolor[gray]{0.89} & \cellcolor[gray]{0.92} & \cellcolor[gray]{0.89} \\ 
	& \cellcolor[gray]{0.95} & \cellcolor[gray]{0.97} & \cellcolor[gray]{0.95} \\ 
	& \cellcolor[gray]{0.89} & \cellcolor[gray]{0.92} & \cellcolor[gray]{0.89} \\ 
	& \cellcolor[gray]{0.95} & \cellcolor[gray]{0.97} & \cellcolor[gray]{0.95} \\ 
	& \cellcolor[gray]{0.89} & \cellcolor[gray]{0.92} & \cellcolor[gray]{0.89} \\ 
	& \cellcolor[gray]{0.95} & \cellcolor[gray]{0.97} & \cellcolor[gray]{0.95} \\ 
	\cmidrule{2-4}
\bottomrule
\end{tabular}
\caption[][5mm]{Valores de tempo e deslocamento para o MRU.}
\label{DadosMRU}
\end{table*}

\begin{table*}[!ht]
\centering
\begin{tabular}{lp{25mm}p{25mm}p{25mm}p{25mm}p{25mm}l}
\toprule
	&\multicolumn{4}{l}{\textbf{Dados Experimentais}} \\
	\cmidrule{2-6}
	& $x_0$ & $x_f$ & $t_1$ & $t_2$ & $t_3$ & \\
	\cmidrule{2-6}
	& \cellcolor[gray]{0.89} & \cellcolor[gray]{0.92} & \cellcolor[gray]{0.89} & \cellcolor[gray]{0.92} & \cellcolor[gray]{0.89} \\
	& \cellcolor[gray]{0.95} & \cellcolor[gray]{0.97} & \cellcolor[gray]{0.95} & \cellcolor[gray]{0.97} & \cellcolor[gray]{0.95} \\
	& \cellcolor[gray]{0.89} & \cellcolor[gray]{0.92} & \cellcolor[gray]{0.89} & \cellcolor[gray]{0.92} & \cellcolor[gray]{0.89} \\
	& \cellcolor[gray]{0.95} & \cellcolor[gray]{0.97} & \cellcolor[gray]{0.95} & \cellcolor[gray]{0.97} & \cellcolor[gray]{0.95} \\
	& \cellcolor[gray]{0.89} & \cellcolor[gray]{0.92} & \cellcolor[gray]{0.89} & \cellcolor[gray]{0.92} & \cellcolor[gray]{0.89} \\
	& \cellcolor[gray]{0.95} & \cellcolor[gray]{0.97} & \cellcolor[gray]{0.95} & \cellcolor[gray]{0.97} & \cellcolor[gray]{0.95} \\
	& \cellcolor[gray]{0.89} & \cellcolor[gray]{0.92} & \cellcolor[gray]{0.89} & \cellcolor[gray]{0.92} & \cellcolor[gray]{0.89} \\
	& \cellcolor[gray]{0.95} & \cellcolor[gray]{0.97} & \cellcolor[gray]{0.95} & \cellcolor[gray]{0.97} & \cellcolor[gray]{0.95} \\
	& \cellcolor[gray]{0.89} & \cellcolor[gray]{0.92} & \cellcolor[gray]{0.89} & \cellcolor[gray]{0.92} & \cellcolor[gray]{0.89} \\
	& \cellcolor[gray]{0.95} & \cellcolor[gray]{0.97} & \cellcolor[gray]{0.95} & \cellcolor[gray]{0.97} & \cellcolor[gray]{0.95} \\
	\cmidrule{2-6}
\\
	& \multicolumn{3}{l}{\textbf{Dados calculados}} \\
	\cmidrule{2-4}
	& $\Delta x$ & $\mean{t}$ & $g$ \\
	\cmidrule{2-4}
	& \cellcolor[gray]{0.89} & \cellcolor[gray]{0.92} & \cellcolor[gray]{0.89} \\ 
	& \cellcolor[gray]{0.95} & \cellcolor[gray]{0.97} & \cellcolor[gray]{0.95} \\ 
	& \cellcolor[gray]{0.89} & \cellcolor[gray]{0.92} & \cellcolor[gray]{0.89} \\ 
	& \cellcolor[gray]{0.95} & \cellcolor[gray]{0.97} & \cellcolor[gray]{0.95} \\ 
	& \cellcolor[gray]{0.89} & \cellcolor[gray]{0.92} & \cellcolor[gray]{0.89} \\ 
	& \cellcolor[gray]{0.95} & \cellcolor[gray]{0.97} & \cellcolor[gray]{0.95} \\ 
	& \cellcolor[gray]{0.89} & \cellcolor[gray]{0.92} & \cellcolor[gray]{0.89} \\ 
	& \cellcolor[gray]{0.95} & \cellcolor[gray]{0.97} & \cellcolor[gray]{0.95} \\ 
	& \cellcolor[gray]{0.89} & \cellcolor[gray]{0.92} & \cellcolor[gray]{0.89} \\ 
	& \cellcolor[gray]{0.95} & \cellcolor[gray]{0.97} & \cellcolor[gray]{0.95} \\ 
	\cmidrule{2-4}
\bottomrule
\end{tabular}
\caption[][5mm]{Dados do MRUV}
\label{DadosMRUV}
\end{table*}

