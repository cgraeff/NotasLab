%%%%%%%%%%%%%%%%%%%%%%%%%%%%%%%%%%%%%%%%%%%%%%%%%%%%%%%%%%%%%%%%%%%%%%%%%%%%%%%
\chapter{Lei de Ohm} % Sem "Experiência 01" ou qualquer outro número
\label{Chap:LeiDeOhm}        % para poder trocar a ordem com facilidade
%%%%%%%%%%%%%%%%%%%%%%%%%%%%%%%%%%%%%%%%%%%%%%%%%%%%%%%%%%%%%%%%%%%%%%%%%%%%%%%

\begin{fullwidth}\it
	Faremos um experimento simples onde verificaremos a corrente em função da tensão para dois tipos de resistores, um ôhmico e outro não-ôhmico. Buscaremos observar que a resistência pode ser ou não uma constante, dependendo do material em questão. Veremos um modelo físico para materiais condutores e para a resistência, obtendo uma expressão para a velocidade de deriva dos elétrons. Definiremos a densidade de corrente $J$ e sua relação com o campo elétrico em um condutor e consequente classificação dos condutores como ôhmicos e não-ôhmicos. Definiremos também os conceitos de resistividade e condutibilidade. Para analisar os dados obtidos, usaremos gráficos e regressões lineares.
\end{fullwidth}

%%%%%%%%%%%%%%%%%%%%%%%%%%%%%%%%%%%%%%%%%%%%%%%%%%%%%%%%%%%%%%%%%%%%%%%%%%%%%%%
\section{Modelo microscópico para a corrente elétrica}
%%%%%%%%%%%%%%%%%%%%%%%%%%%%%%%%%%%%%%%%%%%%%%%%%%%%%%%%%%%%%%%%%%%%%%%%%%%%%%%

Microscopicamente, a corrente elétrica nada mais é do que o deslocamento de portadores de carga ---~geralmente elétrons livres, mas podendo ser íons positivos ou mesmo, no caso de semicondutores, ``buracos'' onde não há um elétron~---. Quando não há corrente, podemos considerar que os portadores de carga se comportam de uma maneira similar a um gás, movendo-se aleatoriamente devido a colisões com a rede de átomos que compõe o condutor.

Quando o condutor é sujeito a um campo elétrico, as cargas se deslocam de maneira a neutralizá-lo, porém se ele se deve à diferença de potencial elétrico dos polos de uma bateria, por exemplo, o campo nunca é neutralizado e a corrente se mantém fluindo. Esse fluxo de portadores de corrente não é livre, mas ainda está sujeito às sucessivas colisões com os átomos que compõe o condutor. No entanto, devido ao campo elétrico e à consequente força que atua sobre as cargas, o movimento tem uma direção na qual as cargas são aceleradas até uma eventual nova colisão. Devido às colisões, a velocidade não aumenta indefinidamente, mas atinge um valor denominado \emph{velocidade de deriva}. Uma consequência notável dessas colisões é o \emph{efeito joule}, isto é, o aquecimento característico de um condutor sujeito a uma corrente.

A velocidade de deriva pode ser calculada ao considerarmos uma seção cilíndrica com área de seção reta $A$ e comprimento $\ell$ de um condutor portanto corrente. A quantidade total de carga $Q$ contida nesse segmento é dada por
\begin{equation}
    Q = n A \ell q,
\end{equation}
%
onde $n$ é a densidade de portadores por unidade de volume e $q$ é a carga de cada portador. Se considerarmos que os portadores se movem com velocidade $v_d$, então em um tempo $\Delta t$ eles se deslocam uma distância $\ell = v_d \Delta$. Assim, podemos escrever
\begin{equation}
    Q = nA v_d \Delta t q.
\end{equation}
%
Dividindo ambos os lados da equação por $\Delta t$, obtemos a corrente média no condutor:
\begin{equation}
    \mean{i} = \frac{Q}{\Delta t} = nqv_d A,
\end{equation}
%
De onde podemos escrever, fazendo $\mean{i} = i = \text{cte}$
\begin{equation}
    v_d = \frac{i}{nqA}.
\end{equation}
%
Note que a velocidade de deriva não é constante, mas depende do valor de corrente e do material condutor utilizado, uma vez que a densidade de portadores de corrente $n$ é diferente para cada material.

%%%%%%%%%%%%%%%%%%%%%%%%%%
\subsection{Condutividade, resistividade, e resistência}
%%%%%%%%%%%%%%%%%%%%%%%%%%

Podemos definir a densidade de corrente $J$ como sendo a razão entre a corrente e a área de seção reta do condutor
\begin{equation}
    J = \frac{i}{A},
\end{equation}
%
ou, em termos da densidade de portadores e da velocidade de deriva,\footnote{Note que essa definição para a densidade de corrente assume que ela é uniforme em toda a área $A$.}
\begin{equation}
    J = nqv_d.
\end{equation}
%
Para alguns materiais, a relação entre a densidade de corrente $J$ e o campo elétrico $E$ dentro do condutor pode ser descrita de acordo com
\begin{equation}
    J = \sigma E,
\end{equation}
%
onde $\sigma$ representa a \emph{condutividade} do material. Caso o condutor obedeça a essa relação, dizemos que ele é um \emph{condutor ôhmico}. Caso contrário, dizemos que ele é um \emph{condutor não ôhmico}.

Podemos obter uma relação mais prática se observarmos que para um segmento de um material condutor com área de seção reta $A$ e comprimento $\ell$ a diferença de potencial entre suas extremidades pode ser escrita como
\begin{equation}
    \Delta V = E\ell.
\end{equation}
%
Logo, a densidade de corrente pode ser escrita como
\begin{align}
    J &= \sigma E \\
    &= \sigma \frac{\Delta V}{\ell}.
\end{align}
%
Isolando $\Delta V$ e substituindo $J = i / A$, obtemos
\begin{equation}
    \Delta V = \left(\frac{\ell}{\sigma A}\right) i.
\end{equation}
%
A quantidade entre parêntesis na equação acima é o que denominamos como \emph{resistência} do segmento de material condutor:
\begin{equation}
    R = \frac{\ell}{\sigma A}.
\end{equation}
%
Em circuitos eletrônicos, é comum que se usem componentes denominados como \emph{resistores} e que são feitos de um material com baixa condutibilidade, como grafite, algumas cerâmicas, ou óxidos metálicos. Considerando a resistência $R$, podemos escrever a relação
\begin{equation}
    R = \frac{\Delta V}{i},
\end{equation}
%
que é comumente empregada quando tratamos um sistema que inclui um resistor.

Note que no caso de materias ôhmicos, o valor de condutividade é constante. Consequentemente, a resistência $R$ de um resistor feito de um material ôhmico também é constante e denominamos tal componente como um \emph{resistor ôhmico}. Podemos identificar um resistor desse tipo facilmente, bastanto verificar se a relação entre a corrente que passa por ele e a tensão a que ele é submetido é linear.

Finalmente, é interessante notar que o inverso da condutibilidade é denominado como \emph{resistividade} $\rho$:
\begin{equation}
    \rho \equiv \frac{1}{\sigma}.
\end{equation}
%
Em termos da resistividade, temos que a resistência é dada por
\begin{equation}
    R = \frac{\rho \ell}{A}.
\end{equation}
%
Essa grandeza pode ser mais conveniente para a análise da resistência de um material longo, como um fio, uma vez que podemos estar interessados em conhecer a resistência por unidade de comprimento. Nesse caso,
\begin{equation}
    \frac{R}{\ell} = \frac{\rho}{A}.
\end{equation}

%%%%%%%%%%%%%%%%%%%%%%%%%%%%%%%%%%%%%%%%%%%%%
%\subsection{Se necessário, usar subsections}
%%%%%%%%%%%%%%%%%%%%%%%%%%%%%%%%%%%%%%%%%%%%%

%%%%%%%%%%%%%%%%%%%%%%%%%%%%%%%%%%%%%%%%%%%%%%%%%%%%%%%%%%%%%%%%%%%%%%%%%%%%%%%
\section{Experimento}
%%%%%%%%%%%%%%%%%%%%%%%%%%%%%%%%%%%%%%%%%%%%%%%%%%%%%%%%%%%%%%%%%%%%%%%%%%%%%%%

%%%%%%%%%%%%%%%%%%%%%%
\subsection{Objetivos}
\label{Sec:ObjetivosLeiDeOhm}
%%%%%%%%%%%%%%%%%%%%%%

\begin{itemize}
	\item Verificar a relação linear entre tensão e corrente para resistores ôhmicos;
	\item Determinar a resistência de um resistor ôhmico através de uma regressão linear;
	\item Verificar a relação não entre tensão e corrente para resistores não ôhmicos;
	\item Observar a relação entre resistência e corrente, assim como a relação entre a resistência e a tensão, buscando determinar se há ou não algum padrão claro de comportamento;
	%\item Verificar a resistência de associações de resistores em série e em paralelo.
\end{itemize}

%%%%%%%%%%%%%%%%%%%%%%%%%%%%%%%%%%%%%%%%%%%%%%%%%%%%%%%%%%%%%%%%%%%%%%%%%%%%%%%
\section{Material Necessário}
%%%%%%%%%%%%%%%%%%%%%%%%%%%%%%%%%%%%%%%%%%%%%%%%%%%%%%%%%%%%%%%%%%%%%%%%%%%%%%%

\begin{itemize}
	\item Multímetro;
	\item Fonte de tensão ajustável;
	\item Protoboard e resistores;
	\item Lâmpadas incandescentes e soquetes; 
	\item Cabos de ligação.
\end{itemize}

%%%%%%%%%%%%%%%%%%%%%%%%%%%%%%%%%%%%%%%%%%%%%%%%%%%%%%%%%%%%%%%%%%%%%%%%%%%%%%%
\section{Procedimento Experimental}
%%%%%%%%%%%%%%%%%%%%%%%%%%%%%%%%%%%%%%%%%%%%%%%%%%%%%%%%%%%%%%%%%%%%%%%%%%%%%%%

%%%%%%%%%%%%%%%%%%%%%%%%%%%%%%%%%%%%%%%%%%%%%
\subsection{Resistores ôhmicos e não-ôhmicos}
%%%%%%%%%%%%%%%%%%%%%%%%%%%%%%%%%%%%%%%%%%%%%

\begin{marginfigure}[2cm]
    \centering
    \begin{circuitikz}[american, scale = 0.9]          	
        \draw (0,0) to[battery1, v=$V$] (0,3)
                    to[switch] (2,3)
                    to[R, l=$R$] (2,0)
                    to[smeter, t=$A$] (0,0);
    	\draw (2,3) -- (4,3) to[smeter, t=V] (4,0) -- (2,0);
    \end{circuitikz}
    \caption{Circuito para a verificação do comportamento de um resistor ôhmico.}
\end{marginfigure}

\paragraph{Coleta de dados para um resistor:}
\begin{enumerate}
	\item Ajuste a fonte de tensão regulável em seu valor mínimo de tensão e máximo de corrente; 
	\item Ajuste o multímetro na função de amperímetro, na escala de \np[mA]{200};
	\item Ligue um dos terminais da fonte ao multímetro, na porta marcada como \texttt{COM};
	\item Conecte um cabo ao terminal do multímetro marcado como \texttt{mA} e o ligue a um resistor;
	\item Ligue o outro terminal da fonte ao resistor;
	\item Ligue a fonte e ajuste a tensão para aproximadamente \np[V]{1,0} e anote o valor na Tabela~\ref{Tab:ResistoresOhmicosENaoOhmicos};
	\item Verifique o valor correspondente de corrente e o anote na Tabela~\ref{Tab:ResistoresOhmicosENaoOhmicos};
	\item Calcule o valor de potência dissipada e o anote na Tabela~\ref{Tab:ResistoresOhmicosENaoOhmicos}. \textbf{Atenção:} o valor de potência não deve exceder \np[W]{0,5} durante toda a tomada de dados, caso contrário, o componente poderá ficar muito quente e causar queimaduras e/ou ser danificado;
	\item Aumente o valor de tensão da fonte regulável em aproximadamente \np[V]{1,0} e repita os passos acima, completando a Tabela~\ref{Tab:ResistoresOhmicosENaoOhmicos} até \np[V]{10}, ou até que a potência exceda \np[W]{0,5}. 
\end{enumerate}

\paragraph{Coleta de dados para uma lâmpada incandescente:}
\begin{marginfigure}[3cm]
    \centering
    \begin{circuitikz}[american, scale = 0.9]          	
        \draw (0,0) to[battery1, v=$V$] (0,3)
                    to[switch] (2,3)
                    to[bulb, l=$R$] (2,0)
                    to[smeter, t=$A$] (0,0);
    	\draw (2,3) -- (4,3) to[smeter, t=V] (4,0) -- (2,0);
    \end{circuitikz}
    \caption{Circuito para a verificação do comportamento de um resistor não-ôhmico.}
\end{marginfigure}

\begin{enumerate}
	\item Ajuste a fonte de tensão regulável em seu valor mínimo de tensão e máximo de corrente; 
	\item Ajuste o multímetro na função de amperímetro, na escala de \np[A]{20};
	\item Ligue um dos terminais da fonte ao multímetro, na porta marcada como \texttt{COM};
	\item Conecte um cabo ao terminal do multímetro marcado como \texttt{20A} e o ligue aos cabos do soquete da lâmpada;
	\item Ligue o outro terminal da fonte ao resistor;
	\item Ligue a fonte e ajuste a tensão para aproximadamente \np[V]{2,0} e anote o valor na Tabela~\ref{Tab:ResistoresOhmicosENaoOhmicos};
	\item Verifique o valor correspondente de corrente e o anote na Tabela~\ref{Tab:ResistoresOhmicosENaoOhmicos};
	\item Calcule o valor de potência dissipada e o anote na Tabela~\ref{Tab:ResistoresOhmicosENaoOhmicos};
	\item Aumente o valor de tensão da fonte regulável em aproximadamente \np[V]{1,0} e repita os passos acima, completando a Tabela~\ref{Tab:ResistoresOhmicosENaoOhmicos} até \np[V]{15}. 
\end{enumerate}

%%%%%%%%%%%%%%%%%%%%%%%%%%%%%%%%%%%%%
%\subsection{Associação de resistores} % Na real, seria muito melhor fazermos um experimento de resistividade
%%%%%%%%%%%%%%%%%%%%%%%%%%%%%%%%%%%%%

%%%%%%%%%%%%%%%%%%%%%%%%%%%%%%%%%%%%%%%%%%%%%%%%%%%%%%%%%%%%%%%%%%%%%%%%%%%%%%%
%%%%%%%%%%%%%%%%%%%%%%%%%%%%%%%%%%%%%%%%%%%%%%%%%%%%%%%%%%%%%%%%%%%%%%%%%%%%%%%
%%%%%%%%%%%%%%%%%%%%%%%%%%%%%%%%%%%%%%%%%%%%%%%%%%%%%%%%%%%%%%%%%%%%%%%%%%%%%%%
%%%%%%%%%%%%%%%%%%%%%%%%%%%%%%%%%%%%%%%%%%%%%%%%%%%%%%%%%%%%%%%%%%%%%%%%%%%%%%%
\cleardoublepage

\noindent{}{\huge\textit{Lei de Ohm}}

\vspace{15mm}

\begin{fullwidth}
\noindent{}\makebox[0.6\linewidth]{Turma:\enspace\hrulefill}\makebox[0.4\textwidth]{  Data:\enspace\hrulefill}
\vspace{5mm}

\noindent{}\makebox[0.6\linewidth]{Aluno(a):\enspace\hrulefill}\makebox[0.4\textwidth]{  Matrícula:\enspace\hrulefill}

\noindent{}\makebox[0.6\linewidth]{Aluno(a):\enspace\hrulefill}\makebox[0.4\textwidth]{  Matrícula:\enspace\hrulefill}

\noindent{}\makebox[0.6\linewidth]{Aluno(a):\enspace\hrulefill}\makebox[0.4\textwidth]{  Matrícula:\enspace\hrulefill}

\noindent{}\makebox[0.6\linewidth]{Aluno(a):\enspace\hrulefill}\makebox[0.4\textwidth]{  Matrícula:\enspace\hrulefill}

\noindent{}\makebox[0.6\linewidth]{Aluno(a):\enspace\hrulefill}\makebox[0.4\textwidth]{  Matrícula:\enspace\hrulefill}
\end{fullwidth}

\vspace{5mm}

%%%%%%%%%%%%%%%%%%%%%%%%%%%%%%%%%%%%%%%%%%%%%%%%%%%%%%%%%%%%%%%%%%%%%%%%%%%%%%%
\section{Questionário}
%%%%%%%%%%%%%%%%%%%%%%%%%%%%%%%%%%%%%%%%%%%%%%%%%%%%%%%%%%%%%%%%%%%%%%%%%%%%%%%

\begin{question}[type={exam}]{2}
Preencha as colunas de dados experimentais das tabelas com o número adequado de algarismos significativos e unidades.
\end{question}

\begin{question}[type={exam}]{3}
\begin{enumerate}[label=\roman*.]
\item Faça um gráfico para os dados de corrente $i$ em função da tensão $V$ para o resistor.
\item Realize uma regressão linear dos dados experimentais contidos no gráfico.
\item Através dos coeficientes da regressão linear, determine a resistência.
\end{enumerate}
\end{question}

\begin{question}[type={exam}]{2.5}
Considerando os dados obtidos para a lâmpada incandescente, elabore os seguintes gráficos:
\begin{enumerate}[label=\roman*.]
\item Um gráfico para os dados de corrente $i$ em função da tensão $V$.
\item Um gráfico para os dados de resistência $R$ em função da tensão $V$.
\item Um gráfico para os dados de resistência $R$ em função da corrente $i$.
\end{enumerate}
\end{question}

\begin{question}[type={exam}]{2.5}
Considerando os objetivos do experimento, listados na Seção~\ref{Sec:ObjetivosLeiDeOhm}, e os resultados obtidos nas questões anteriores, discuta quais objetivos foram atingidos com sucesso, justificando suas conclusões. Se algum objetivo não foi atingido, discuta quais são os possíveis motivos do insucesso e que providências podem ser tomadas para que eles sejam alcançados.
\end{question}


\vfill
%%%%%%%%%%%%%%%%%%%%%%%%%%%%%%%%%%%%%%%%%%%%%%%%%%%%%%%%%%%%%%%%%%%%%%%%%%%%%%%
\pagebreak
\section{Tabelas}
%%%%%%%%%%%%%%%%%%%%%%%%%%%%%%%%%%%%%%%%%%%%%%%%%%%%%%%%%%%%%%%%%%%%%%%%%%%%%%%

\begin{table*}
	\begin{center}
		\begin{tabular}{lp{20mm}p{20mm}p{20mm}lp{20mm}p{20mm}p{20mm}l}
		\toprule
		& \multicolumn{3}{l}{\textbf{Resistor}} & & \multicolumn{3}{l}{\textbf{Lâmpada}} \\
		\cmidrule{2-4} \cmidrule{6-8}
		& $V$ & $i$ & $P$ & & $V$ & $i$ & $R = V/i$ \\
		\cmidrule{2-4} \cmidrule{6-8}
		& \cellcolor[gray]{0.89} & \cellcolor[gray]{0.92} & \cellcolor[gray]{0.89} & & \cellcolor[gray]{0.89} & \cellcolor[gray]{0.92} & \cellcolor[gray]{0.89} & \\
		& \cellcolor[gray]{0.95} & \cellcolor[gray]{0.97} & \cellcolor[gray]{0.95} & & \cellcolor[gray]{0.95} & \cellcolor[gray]{0.97} & \cellcolor[gray]{0.95} & \\
		& \cellcolor[gray]{0.89} & \cellcolor[gray]{0.92} & \cellcolor[gray]{0.89} & & \cellcolor[gray]{0.89} & \cellcolor[gray]{0.92} & \cellcolor[gray]{0.89} & \\
		& \cellcolor[gray]{0.95} & \cellcolor[gray]{0.97} & \cellcolor[gray]{0.95} & & \cellcolor[gray]{0.95} & \cellcolor[gray]{0.97} & \cellcolor[gray]{0.95} & \\
		& \cellcolor[gray]{0.89} & \cellcolor[gray]{0.92} & \cellcolor[gray]{0.89} & & \cellcolor[gray]{0.89} & \cellcolor[gray]{0.92} & \cellcolor[gray]{0.89} & \\
		& \cellcolor[gray]{0.95} & \cellcolor[gray]{0.97} & \cellcolor[gray]{0.95} & & \cellcolor[gray]{0.95} & \cellcolor[gray]{0.97} & \cellcolor[gray]{0.95} & \\
		& \cellcolor[gray]{0.89} & \cellcolor[gray]{0.92} & \cellcolor[gray]{0.89} & & \cellcolor[gray]{0.89} & \cellcolor[gray]{0.92} & \cellcolor[gray]{0.89} & \\
		& \cellcolor[gray]{0.95} & \cellcolor[gray]{0.97} & \cellcolor[gray]{0.95} & & \cellcolor[gray]{0.95} & \cellcolor[gray]{0.97} & \cellcolor[gray]{0.95} & \\
		& \cellcolor[gray]{0.89} & \cellcolor[gray]{0.92} & \cellcolor[gray]{0.89} & & \cellcolor[gray]{0.89} & \cellcolor[gray]{0.92} & \cellcolor[gray]{0.89} & \\
		& \cellcolor[gray]{0.95} & \cellcolor[gray]{0.97} & \cellcolor[gray]{0.95} & & \cellcolor[gray]{0.95} & \cellcolor[gray]{0.97} & \cellcolor[gray]{0.95} & \\
		& \cellcolor[gray]{0.89} & \cellcolor[gray]{0.92} & \cellcolor[gray]{0.89} & & \cellcolor[gray]{0.89} & \cellcolor[gray]{0.92} & \cellcolor[gray]{0.89} & \\
		& \cellcolor[gray]{0.95} & \cellcolor[gray]{0.97} & \cellcolor[gray]{0.95} & & \cellcolor[gray]{0.95} & \cellcolor[gray]{0.97} & \cellcolor[gray]{0.95} & \\
		& \cellcolor[gray]{0.89} & \cellcolor[gray]{0.92} & \cellcolor[gray]{0.89} & & \cellcolor[gray]{0.89} & \cellcolor[gray]{0.92} & \cellcolor[gray]{0.89} & \\
		& \cellcolor[gray]{0.95} & \cellcolor[gray]{0.97} & \cellcolor[gray]{0.95} & & \cellcolor[gray]{0.95} & \cellcolor[gray]{0.97} & \cellcolor[gray]{0.95} & \\
\bottomrule
		\end{tabular}
	\caption[][2mm]{Dados obtidos para o comprimento do elástico e a massa correspondente.}\label{Tab:ResistoresOhmicosENaoOhmicos}
	\end{center}
\end{table*}

