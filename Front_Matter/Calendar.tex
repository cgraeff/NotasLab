\thispagestyle{plain}
\begin{fullwidth}
\begin{center}
{\noindent\LARGE\textsc{Cronograma}} \\
\end{center}
\end{fullwidth}

\vspace{1cm}
\begin{fullwidth}
\it
As aulas seguirão o planejamento abaixo. No calendário ao lado, estão circuladas as datas das provas.
\end{fullwidth}

%%%
% Engenharia Mecânica
%%%

\begin{marginfigure}[4cm]
\centering
Março\\
\begin{tikzpicture}
\calendar (mycal) [dates=2020-03-01 to 2020-03-last, week list, day headings=gray,day letter headings] if (Saturday,Sunday) [gray];
\end{tikzpicture}
\end{marginfigure}
\begin{marginfigure}
\centering
Abril\\
\begin{tikzpicture}
\calendar (mycal) [dates=2020-04-01 to 2020-04-last,week list, day headings=gray,day letter headings] if (Saturday,Sunday,equals=2020-04-10, equals=2020-04-20, equals=2020-04-21) [gray];
\end{tikzpicture}
\end{marginfigure}
\begin{marginfigure}
\centering
Maio\\
\begin{tikzpicture}
\calendar (mycal) [dates=2020-05-01 to 2020-05-last,week list, day headings=gray,day letter headings] if (Saturday,Sunday,equals=2020-05-01,equals=2020-05-04,equals=2020-05-05,equals=2020-05-06,equals=2020-05-07,equals=2020-05-08, equals=2020-05-25, equals=2020-05-26, equals=2020-05-27) [gray];
%\draw[dashed] (mycal-2019-10-16) circle (6pt);
\end{tikzpicture}
\end{marginfigure}
\begin{marginfigure}
\centering
Junho\\
\begin{tikzpicture}
\calendar (mycal) [dates=2020-06-01 to 2020-06-last,week list, day headings=gray,day letter headings] if (Saturday,Sunday, equals=2020-06-11, equals=2020-06-12, equals=2020-06-29) [gray];
\draw (mycal-2020-06-30) circle (6pt);
\end{tikzpicture}
\end{marginfigure}
\begin{marginfigure}
\centering
Julho\\
\begin{tikzpicture}
\calendar (mycal) [dates=2020-07-01 to 2020-07-last,week list, day headings=gray,day letter headings] if (Saturday,Sunday, equals=2020-07-15, equals=2020-07-16, equals=2020-07-17, equals=2020-07-18, equals=2020-07-19, equals=2020-07-20, equals=2020-07-21, equals=2020-07-22, equals=2020-07-23, equals=2020-07-24, equals=2020-07-25, equals=2020-07-26, equals=2020-07-27, equals=2020-07-28, equals=2020-07-29, equals=2020-07-30, equals=2020-07-31) [gray];
\end{tikzpicture}
\end{marginfigure}
\vspace{1cm}
\begin{center}
\Large\textsc{Engenharia Mecânica}
\end{center}

As aulas seguirão o planejamento abaixo.
\begin{center}
\begin{longtable}{ccp{70mm}}
\toprule
Aula & Data & Conteúdo \\
\midrule
\endhead
\bottomrule
\endfoot
 1 & 03/03 & Turmas A e B: Apresentação da disciplina. \\
 2 & 10/03 & Turma A: Exp. 1, Medidas. \\
 3 & 17/03 & Turma B: Exp. 1, Medidas. \\
 4 & 24/03 & Turma A: Exp. 2, MRU e MRUV. \\ 
 5 & 31/03 & Turma B: Exp. 2, MRU e MRUV. \\
 6 & 07/04 & Turma A: Exp. 3, Lei de Hooke. \\
 7 & 14/04 & Turma B: Exp. 3, Lei de Hooke. \\
 8 & 21/04 & \emph{Feriado}. \\
 9 & 28/04 & Turma A: Exp. 4, Leis de Newton. \\
-- & 05/05 & \emph{Recesso}. \\
 9 & 12/05 & Turma B: Exp. 4, Leis de Newton. \\
10 & 19/05 & Turma A: Exp. 5, Atrito. \\
-- & 26/05 & \emph{Semana Acadêmica de Engenharia Mecânica e Manutenção Industrial (V SAEMMI)}. \\
11 & 02/06 & Turma B: Exp. 5, Atrito. \\
12 & 09/06 & Turma A: Exp. 6, Arrasto. \\
13 & 16/06 & Turma B: Exp. 6, Arrasto. \\
14 & 23/06 & Turma A: Exp. 7, Trabalho e Energia. \\
15 & 30/06 & Turma B: Exp. 7, Trabalho e Energia. \\
16 & 07/07 & Turmas A e B: Prova. \\
17 & 14/07 & Entrega da nota da prova de laboratório.
\end{longtable}
\end{center}

\clearpage

\begin{marginfigure}[4cm]
\centering
Março\\
\begin{tikzpicture}
\calendar (mycal) [dates=2020-03-01 to 2020-03-last, week list, day headings=gray,day letter headings] if (Saturday,Sunday) [gray];
\end{tikzpicture}
\end{marginfigure}
\begin{marginfigure}
\centering
Abril\\
\begin{tikzpicture}
\calendar (mycal) [dates=2020-04-01 to 2020-04-last,week list, day headings=gray,day letter headings] if (Saturday,Sunday,equals=2020-04-10, equals=2020-04-20, equals=2020-04-21) [gray];
\end{tikzpicture}
\end{marginfigure}
\begin{marginfigure}
\centering
Maio\\
\begin{tikzpicture}
\calendar (mycal) [dates=2020-05-01 to 2020-05-last,week list, day headings=gray,day letter headings] if (Saturday,Sunday,equals=2020-05-01,equals=2020-05-04,equals=2020-05-05,equals=2020-05-06,equals=2020-05-07,equals=2020-05-08, equals=2020-05-25, equals=2020-05-26, equals=2020-05-27) [gray];
%\draw[dashed] (mycal-2019-10-16) circle (6pt);
\end{tikzpicture}
\end{marginfigure}
\begin{marginfigure}
\centering
Junho\\
\begin{tikzpicture}
\calendar (mycal) [dates=2020-06-01 to 2020-06-last,week list, day headings=gray,day letter headings] if (Saturday,Sunday, equals=2020-06-11, equals=2020-06-12, equals=2020-06-29) [gray];
\draw (mycal-2020-06-24) circle (6pt);
\end{tikzpicture}
\end{marginfigure}
\begin{marginfigure}
\centering
Julho\\
\begin{tikzpicture}
\calendar (mycal) [dates=2020-07-01 to 2020-07-last,week list, day headings=gray,day letter headings] if (Saturday,Sunday, equals=2020-07-15, equals=2020-07-16, equals=2020-07-17, equals=2020-07-18, equals=2020-07-19, equals=2020-07-20, equals=2020-07-21, equals=2020-07-22, equals=2020-07-23, equals=2020-07-24, equals=2020-07-25, equals=2020-07-26, equals=2020-07-27, equals=2020-07-28, equals=2020-07-29, equals=2020-07-30, equals=2020-07-31) [gray];
\end{tikzpicture}
\end{marginfigure}
\vspace{1cm}
\begin{center}
\Large\textsc{Engenharia Elétrica}
\end{center}

As aulas seguirão o planejamento abaixo.
\begin{center}
\begin{longtable}{ccp{70mm}}
\toprule
Aula & Data & Conteúdo \\
\midrule
\endhead
\bottomrule
\endfoot
 1 & 04/03 & Turmas A e B: Apresentação da disciplina. \\
 2 & 11/03 & Turma A: Exp. 1, Medidas. \\
 3 & 18/03 & Turma B: Exp. 1, Medidas. \\
 4 & 25/03 & Turma A: Exp. 2, MRU e MRUV. \\ 
 5 & 01/04 & Turma B: Exp. 2, MRU e MRUV. \\
 6 & 08/04 & Turma A: Exp. 3, Lei de Hooke. \\
 7 & 15/04 & Turma B: Exp. 3, Lei de Hooke. \\
 8 & 22/04 & Turma A: Exp. 4, Leis de Newton. \\
 9 & 29/04 & Turma B: Exp. 4, Leis de Newton. \\
-- & 06/05 & \emph{Recesso}. \\
 9 & 13/05 & Turma A: Exp. 5, Atrito. \\
10 & 20/05 & Turma B: Exp. 5, Atrito. \\
11 & 27/05 & Turma A: Exp. 6, Arrasto. \\
12 & 03/06 & Turma B: Exp. 6, Arrasto. \\
13 & 10/06 & Turma A: Exp. 7, Trabalho e Energia. \\
14 & 17/06 & Turma B: Exp. 7, Trabalho e Energia. \\
15 & 24/06 & Turmas A e B: Prova. \\
16 & 01/07 & Entrega da nota da prova de laboratório. \\
\end{longtable}
\end{center}

\cleardoublepage
