\thispagestyle{plain}
\begin{fullwidth}
\begin{center}
{\noindent\LARGE\textsc{Cronograma}} \\
\end{center}
\end{fullwidth}

\vspace{1cm}
\begin{fullwidth}
\it
As aulas seguirão o planejamento abaixo. No calendário ao lado, estão circuladas as datas das provas.
\end{fullwidth}

%%%
% Laboratório de Física 2 - MC
%%%

\begin{marginfigure}[4cm]
    \centering
    Agosto\\
    \begin{tikzpicture}
        \calendar (mycal)
        [
            dates=2023-08-01 to 2023-08-last,
            week list,
            day headings=gray,
            day letter headings
        ]
        if (Saturday, Sunday)
            [gray]
        if (at most=2023-08-08)
            [gray]
        ;
    \end{tikzpicture}
\end{marginfigure} %
%
\begin{marginfigure}
    \centering
    Setembro\\
    \begin{tikzpicture}
        \calendar (mycal)
        [
            dates=2023-09-01 to 2023-09-last,
            week list,
            day headings=gray,
            day letter headings
        ]
        if (Saturday, Sunday)
            [gray]
        if (equals=2023-09-07, equals=2023-09-08)
            [gray]
        ;
    \end{tikzpicture}
\end{marginfigure} %
%
\begin{marginfigure}
    \centering
    Outubro\\
    \begin{tikzpicture}
        \calendar (mycal)
        [
            dates=2023-10-01 to 2023-10-last,
            week list,
            day headings=gray,
            day letter headings
        ]
        if (Saturday, Sunday)
            [gray]
         if (equals=2023-10-12, equals=2023-10-13)
            [gray]
        ;
    \end{tikzpicture}
\end{marginfigure} %
%
\begin{marginfigure}
    \centering
    Novembro\\
    \begin{tikzpicture}
        \calendar (mycal)
        [
            dates=2023-11-01 to 2023-11-last,
            week list,
            day headings=gray,
            day letter headings
        ]
        if (Saturday, Sunday)
            [gray]
        if (equals= 2023-11-02, equals=2023-11-03, equals=2023-11-15)
            [gray]
        ;
       \draw (mycal-2023-11-30) circle (6pt);
    \end{tikzpicture}
\end{marginfigure} %
%
\begin{marginfigure}
    \centering
    Dezembro\\
    \begin{tikzpicture}
        \calendar (mycal)
        [
            dates=2023-12-01 to 2023-12-last,
            week list,
            day headings=gray,
            day letter headings
        ]
        if (Saturday, Sunday)
            [gray]
        if (equals= 2023-12-14, equals=2023-12-15)
            [gray]
        if (at most=2023-12-20)
            {}
        else
            [gray]
        ;
    \end{tikzpicture}
\end{marginfigure}
\vspace{1cm}
\begin{center}
\Large\textsc{Laboratório de Física 2 - Engenharia Mecânica}
\end{center}

\begin{center}
\begin{longtable}{ccp{70mm}}
\toprule
Aula & Data & Conteúdo \\
\midrule
\endhead
\bottomrule
\endfoot
 1 & 10/08 & Apresentação da disciplina. \\
 2 & 17/08 & Turma A: Exp. 1, Elasticidade. \\
 3 & 24/08 & --- \\ % Turma B: Exp. 1, Elasticidade. \\
 4 & 31/08 & Turma A: Exp. 2, Oscilações. \\
-- & 07/09 & \emph{Feriado.} \\
 5 & 14/09 & --- \\ % Turma B: Exp. 2, Oscilações. \\
 6 & 21/09 & Turma A: Exp. 3, Ondas estacionárias. \\
 7 & 28/09 & --- \\ % Turma B: Exp. 3, Ondas estacionárias . \\
 8 & 05/10 & Turma A: Exp. 4, Dilatação e lei de resfriamento. \\
-- & 12/10 & \emph{Feriado.} \\
 9 & 19/10 & --- \\ % Turma B: Exp. 4, Dilatação e lei de resfriamento. \\
10 & 26/10 & Turma A: Exp. 5, Calor específico . \\
11 & 02/11 & \emph{Feriado.} \\
12 & 09/11 & --- \\ % Turma B: Exp. 5, Calor específico. \\ 
13 & 16/11 & Turma A: Exp. 6, Zero Absoluto. \\
14 & 23/11 & --- \\ % Turma B: Exp. 7, Zero Absoluto. \\
15 & 30/11 & Turma A: \textbf{Prova de laboratório} \\ %Turmas A e B: \textbf{Prova de laboratório}. \\
16 & 07/12 & Apresentação das notas finais de laboratório. \\
17 & 14/12 & Feriado. \\
\end{longtable}
\end{center}

\clearpage

%%%
% Laboratório de Física 3 - CV
%%%

\begin{marginfigure}[1cm]
    \centering
    Agosto\\
    \begin{tikzpicture}
        \calendar (mycal)
        [
            dates=2023-08-01 to 2023-08-last,
            week list,
            day headings=gray,
            day letter headings
        ]
        if (Saturday, Sunday)
            [gray]
        if (at most=2023-08-08)
            [gray]
        ;
    \end{tikzpicture}
\end{marginfigure} %
%
\begin{marginfigure}[3mm]
    \centering
    Setembro\\
    \begin{tikzpicture}
        \calendar (mycal)
        [
            dates=2023-09-01 to 2023-09-last,
            week list,
            day headings=gray,
            day letter headings
        ]
        if (Saturday, Sunday)
            [gray]
        if (equals=2023-09-07, equals=2023-09-08)
            [gray]
        ;
    \end{tikzpicture}
\end{marginfigure} %
%
\begin{marginfigure}
    \centering
    Outubro\\
    \begin{tikzpicture}
        \calendar (mycal)
        [
            dates=2023-10-01 to 2023-10-last,
            week list,
            day headings=gray,
            day letter headings
        ]
        if (Saturday, Sunday)
            [gray]
         if (equals=2023-10-12, equals=2023-10-13)
            [gray]
        ;
    \end{tikzpicture}
\end{marginfigure} %
%
\begin{marginfigure}
    \centering
    Novembro\\
    \begin{tikzpicture}
        \calendar (mycal)
        [
            dates=2023-11-01 to 2023-11-last,
            week list,
            day headings=gray,
            day letter headings
        ]
        if (Saturday, Sunday)
            [gray]
        if (equals= 2023-11-02, equals=2023-11-03, equals=2023-11-15)
            [gray]
        ;
       \draw (mycal-2023-11-28) circle (6pt);
    \end{tikzpicture}
\end{marginfigure} %
%
\begin{marginfigure}
    \centering
    Dezembro\\
    \begin{tikzpicture}
        \calendar (mycal)
        [
            dates=2023-12-01 to 2023-12-last,
            week list,
            day headings=gray,
            day letter headings
        ]
        if (Saturday, Sunday)
            [gray]
        if (equals= 2023-12-14, equals=2023-12-15)
            [gray]
        if (at most=2023-12-20)
            {}
        else
            [gray]
        ;
    \end{tikzpicture}
\end{marginfigure}
\vspace{1cm}
\begin{center}
\Large\textsc{Laboratório de Física 3 - Engenharia Civil}
\end{center}

\begin{center}
\begin{longtable}{ccp{70mm}}
\toprule
Aula & Data & Conteúdo \\
\midrule
\endhead
\bottomrule
\endfoot
1 & 15/08 & Apresentação da disciplina. \\
2 & 22/08 & Turma A: Exp. 1, Introdução ao eletromagnetismo. \\
3 & 29/08 & Turma B: Exp. 1, Introdução ao eletromagnetismo. \\
4 & 05/09 & Turma A: Exp. 2, Superfícies equipotenciais. \\
5 & 12/09 & Turma B: Exp. 2, Superfícies equipotenciais. \\
6 & 19/09 & Turma A: Exp. 3, Capacitores de placas paralelas. \\
7 & 26/09 & Turma B: Exp. 3, Capacitores de placas paralelas. \\
8 & 03/10 & Turma A: Exp. 4, Lei de Ohm. \\
9 & 10/10 & Turma B: Exp. 4, Lei de Ohm. \\
10 & 17/10 & Turma A: Exp. 5, Circuito RC. \\
11 & 24/10 & \emph{Semana Acadêmica Eng. Civil.} \\
12 & 31/10 & Turma B: Exp. 5, Circuito RC. \\
13 & 07/11 & Turma A: Exp. 6, Campo magnético terrestre. \\
14 & 14/11 & Turma B: Exp. 6, Campo magnético terrestre. \\
15 & 21/11 & Turmas A e B: Aula para dúvidas. \\
16 & 28/11 & Turmas A e B: \textbf{Prova de laboratório}. \\
17 & 05/12 & Turmas A e B: Entrega das notas da prova de laboratório. \\ 
18 & 12/12 & Reposição de experimentos. \\
19 & 19/12 & Entrega das notas finais da disciplina. \\
\end{longtable}
\end{center}

\clearpage

%%%
% Laboratório de Física 3 - MC
%%%

\begin{marginfigure}[1cm]
    \centering
    Agosto\\
    \begin{tikzpicture}
        \calendar (mycal)
        [
            dates=2023-08-01 to 2023-08-last,
            week list,
            day headings=gray,
            day letter headings
        ]
        if (Saturday, Sunday)
            [gray]
        if (at most=2023-08-08)
            [gray]
        ;
    \end{tikzpicture}
\end{marginfigure} %
%
\begin{marginfigure}[3mm]
    \centering
    Setembro\\
    \begin{tikzpicture}
        \calendar (mycal)
        [
            dates=2023-09-01 to 2023-09-last,
            week list,
            day headings=gray,
            day letter headings
        ]
        if (Saturday, Sunday)
            [gray]
        if (equals=2023-09-07, equals=2023-09-08)
            [gray]
        ;
    \end{tikzpicture}
\end{marginfigure} %
%
\begin{marginfigure}
    \centering
    Outubro\\
    \begin{tikzpicture}
        \calendar (mycal)
        [
            dates=2023-10-01 to 2023-10-last,
            week list,
            day headings=gray,
            day letter headings
        ]
        if (Saturday, Sunday)
            [gray]
         if (equals=2023-10-12, equals=2023-10-13)
            [gray]
        ;
    \end{tikzpicture}
\end{marginfigure} %
%
\begin{marginfigure}
    \centering
    Novembro\\
    \begin{tikzpicture}
        \calendar (mycal)
        [
            dates=2023-11-01 to 2023-11-last,
            week list,
            day headings=gray,
            day letter headings
        ]
        if (Saturday, Sunday)
            [gray]
        if (equals= 2023-11-02, equals=2023-11-03, equals=2023-11-15)
            [gray]
        ;
    \end{tikzpicture}
\end{marginfigure} %
%
\begin{marginfigure}
    \centering
    Dezembro\\
    \begin{tikzpicture}
        \calendar (mycal)
        [
            dates=2023-12-01 to 2023-12-last,
            week list,
            day headings=gray,
            day letter headings
        ]
        if (Saturday, Sunday)
            [gray]
        if (equals= 2023-12-14, equals=2023-12-15)
            [gray]
        if (at most=2023-12-20)
            {}
        else
            [gray]
        ;
       \draw (mycal-2023-12-01) circle (6pt);
    \end{tikzpicture}
\end{marginfigure}
\vspace{1cm}
\begin{center}
\Large\textsc{Laboratório de Física 3 - Engenharia Mecânica}
\end{center}

\begin{center}
\begin{longtable}{ccp{70mm}}
\toprule
Aula & Data & Conteúdo \\
\midrule
\endhead
\bottomrule
\endfoot
1 & 11/08 & Apresentação da disciplina. \\
2 & 18/08 & Turma A: Exp. 1, Introdução ao eletromagnetismo. \\
3 & 25/08 & --- \\ % Turma B: Exp. 1, Introdução ao eletromagnetismo. \\
4 & 01/09 & Turma A: Exp. 2, Superfícies equipotenciais. \\
-- & 08/09 & \emph{Recesso}. \\
5 & 15/09 & --- \\ % Turma B: Exp. 2, Superfícies equipotenciais. \\
6 & 22/09 & Turma A: Exp. 3, Capacitores de placas paralelas. \\
7 & 29/09 & --- \\ % Turma B: Exp. 3, Capacitores de placas paralelas. \\
8 & 06/10 & Turma A: Exp. 4, Lei de Ohm. \\
-- & 13/10 & \emph{Recesso}. \\
9 & 20/10 & --- \\ % Turma B: Exp. 4, Lei de Ohm. \\
10 & 27/10 & Turma A: Exp. 5, Circuito RC. \\
-- & 03/11 & \emph{Recesso}. \\
11 & 10/11 & --- \\ % Turma B: Exp. 5, Circuito RC. \\
12 & 17/11 & Turma A: Exp. 6, Campo magnético terrestre. \\
13 & 24/11 & --- \\ % Turma B: Exp. 6, Campo magnético terrestre. \\
14 & 01/12 & Turma A: \textbf{Prova de laboratório} \\ % Turmas A e B: \textbf{Prova de laboratório}. \\
15 & 08/12 & Apresentação das notas finais da disciplina. \\
16 & 15/12 & \emph{Recesso}. \\
\end{longtable}
\end{center}

\cleardoublepage
