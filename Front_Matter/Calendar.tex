\thispagestyle{plain}
\begin{fullwidth}
\begin{center}
{\noindent\LARGE\textsc{Cronograma}} \\
\end{center}
\end{fullwidth}

\vspace{1cm}
\begin{fullwidth}
\it
As aulas seguirão o planejamento abaixo. No calendário ao lado, estão circuladas as datas das provas.
\end{fullwidth}

%%%
% Laboratório de Física 3 - CV
%%%

\begin{marginfigure}[4cm]
    \centering
    Março\\
    \begin{tikzpicture}
        \calendar (mycal)
        [
            dates=2024-03-01 to 2024-03-last,
            week list,
            day headings=gray,
            day letter headings
        ]
        if (Saturday, Sunday)
            [gray]
        if (equals=2024-03-01, equals=2024-03-29)
            [gray]
        ;
        \draw[dotted] (mycal-2024-03-05) circle (6pt);
        \draw[dashed] (mycal-2024-03-12) circle (6pt);
        \draw[dashed] (mycal-2024-03-26) circle (6pt);
    \end{tikzpicture}
\end{marginfigure} %
%
\begin{marginfigure}[3mm]
    \centering
    Abril\\
    \begin{tikzpicture}
        \calendar (mycal)
        [
            dates=2024-04-01 to 2024-04-last,
            week list,
            day headings=gray,
            day letter headings
        ]
        if (Saturday, Sunday)
            [gray]
        ;
        \draw[dashed] (mycal-2024-04-09) circle (6pt);
        \draw[dashed] (mycal-2024-04-23) circle (6pt);
    \end{tikzpicture}
\end{marginfigure} %
%
\begin{marginfigure}
    \centering
    Maio\\
    \begin{tikzpicture}
        \calendar (mycal)
        [
            dates=2024-05-01 to 2024-05-last,
            week list,
            day headings=gray,
            day letter headings
        ]
        if (Saturday, Sunday)
            [gray]
        if (equals=2024-05-01, equals=2024-05-30, equals=2024-05-31)
            [gray]
        ;
        \draw[dashed] (mycal-2024-05-07) circle (6pt);
        \draw[dashed] (mycal-2024-05-21) circle (6pt);
    \end{tikzpicture}
\end{marginfigure} %
%
\begin{marginfigure}
    \centering
    Junho\\
    \begin{tikzpicture}
        \calendar (mycal)
        [
            dates=2024-06-01 to 2024-06-last,
            week list,
            day headings=gray,
            day letter headings
        ]
        if (Saturday, Sunday)
            [gray]
        ;
        \draw[dotted] (mycal-2024-06-04) circle (6pt);
        \draw (mycal-2024-06-11) circle (6pt);
        \draw[dotted] (mycal-2024-06-18) circle (6pt);
        \draw[densely dotted] (mycal-2024-06-25) circle (6pt);
    \end{tikzpicture}
\end{marginfigure} %
%
\begin{marginfigure}
    \centering
    Julho\\
    \begin{tikzpicture}
        \calendar (mycal)
        [
            dates=2024-07-01 to 2024-07-last,
            week list,
            day headings=gray,
            day letter headings
        ]
        if (Saturday, Sunday)
            [gray]
        if (at most=2024-07-05)
            {}
        else
            [gray]
        ;
        \draw[dotted] (mycal-2024-07-02) circle (6pt);
    \end{tikzpicture}
\end{marginfigure}
\vspace{1cm}
\begin{center}
\Large\textsc{Laboratório de Física 3 - Engenharia Civil}
\end{center}

\begin{center}
\begin{longtable}{ccp{70mm}}
\toprule
Aula & Data & Conteúdo \\
\midrule
\endhead
\bottomrule
\endfoot
1 & 05/03 & Apresentação da disciplina. \\
2 & 12/03 & Exp. 1, Introdução ao eletromagnetismo. \\
3 & 19/03 & --- \\
4 & 26/03 & Exp. 2, Superfícies equipotenciais. \\
5 & 02/04 & --- \\
6 & 09/04 & Exp. 3, Capacitores de placas paralelas. \\
7 & 16/04 & --- \\
8 & 23/04 & Exp. 4, Lei de Ohm. \\
9 & 30/04 & --- \\
10 & 07/05 & Exp. 5, Circuito RC. \\
11 & 14/05 & --- \\
12 & 21/05 & Exp. 6, Campo magnético terrestre. \\
13 & 28/05 & --- \\
14 & 04/06 & Aula para dúvidas. \\
15 & 11/06 & \textbf{Prova de laboratório}. \\
16 & 18/06 & Entrega das notas da prova de laboratório. \\ 
17 & 25/06 & 2\textordfeminine~chamada de experimentos. \\
18 & 02/07 & Entrega das notas finais da disciplina. \\
\end{longtable}
\end{center}

\clearpage

%%%
% Laboratório de Física 3 - MC
%%%

\begin{marginfigure}[1cm]
    \centering
    Março\\
    \begin{tikzpicture}
        \calendar (mycal)
        [
            dates=2024-03-01 to 2024-03-last,
            week list,
            day headings=gray,
            day letter headings
        ]
        if (Saturday, Sunday)
            [gray]
        if (equals=2024-03-01, equals=2024-03-29)
            [gray]
        ;
        \draw[dotted] (mycal-2024-03-08) circle (6pt);
        \draw[dashed] (mycal-2024-03-15) circle (6pt);
    \end{tikzpicture}
\end{marginfigure} %
%
\begin{marginfigure}[3mm]
    \centering
    Abril\\
    \begin{tikzpicture}
        \calendar (mycal)
        [
            dates=2024-04-01 to 2024-04-last,
            week list,
            day headings=gray,
            day letter headings
        ]
        if (Saturday, Sunday)
            [gray]
        ;
        \draw[dashed] (mycal-2024-04-05) circle (6pt);
        \draw[dashed] (mycal-2024-04-19) circle (6pt);
    \end{tikzpicture}
\end{marginfigure} %
%
\begin{marginfigure}
    \centering
    Maio\\
    \begin{tikzpicture}
        \calendar (mycal)
        [
            dates=2024-05-01 to 2024-05-last,
            week list,
            day headings=gray,
            day letter headings
        ]
        if (Saturday, Sunday)
            [gray]
        if (equals=2024-05-01, equals=2024-05-30, equals=2024-05-31)
            [gray]
        ;
        \draw[dashed] (mycal-2024-05-03) circle (6pt);
        \draw[dashed] (mycal-2024-05-17) circle (6pt);
    \end{tikzpicture}
\end{marginfigure} %
%
\begin{marginfigure}
    \centering
    Junho\\
    \begin{tikzpicture}
        \calendar (mycal)
        [
            dates=2024-06-01 to 2024-06-last,
            week list,
            day headings=gray,
            day letter headings
        ]
        if (Saturday, Sunday)
            [gray]
        ;
        \draw[dashed] (mycal-2024-06-07) circle (6pt);
        \draw (mycal-2024-06-21) circle (6pt);
        \draw[dotted] (mycal-2024-06-28) circle (6pt);
    \end{tikzpicture}
\end{marginfigure} %
%
\begin{marginfigure}
    \centering
    Julho\\
    \begin{tikzpicture}
        \calendar (mycal)
        [
            dates=2024-07-01 to 2024-07-last,
            week list,
            day headings=gray,
            day letter headings
        ]
        if (Saturday, Sunday)
            [gray]
        if (at most=2024-07-05)
            {}
        else
            [gray]
        ;
        \draw[dotted] (mycal-2024-07-05) circle (6pt);
    \end{tikzpicture}
\end{marginfigure}
\vspace{1cm}
\begin{center}
\Large\textsc{Laboratório de Física 3 - Engenharia Mecânica}
\end{center}

\begin{center}
\begin{longtable}{ccp{70mm}}
\toprule
Aula & Data & Conteúdo \\
\midrule
\endhead
\bottomrule
\endfoot
1 & 08/03 & Apresentação da disciplina. \\
2 & 15/03 & Exp. 1, Introdução ao eletromagnetismo. \\
3 & 22/03 & --- \\
4 & 05/04 & Exp. 2, Superfícies equipotenciais. \\
5 & 12/04 & --- \\
6 & 19/04 & Exp. 3, Capacitores de placas paralelas. \\
7 & 26/04 & --- \\
8 & 03/05 & Exp. 4, Lei de Ohm. \\
9 & 10/05 & --- \\
10 & 17/05 & Exp. 5, Circuito RC. \\
11 & 24/05 & --- \\
12 & 07/06 & Exp. 6, Campo magnético terrestre. \\
13 & 14/06 & --- \\
14 & 21/06 & \textbf{Prova de laboratório}. \\
15 & 28/06 & Entrega das notas da prova de laboratório. \\ 
16 & 05/07 & Entrega das notas finais da disciplina. \\
\end{longtable}
\end{center}

\cleardoublepage
