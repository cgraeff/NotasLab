\DeclareMathOperator{\sen}{sen}
\DeclareMathOperator{\arcsen}{arcsen}
\newcommand{\mean}[1]{\left\langle{#1}\right\rangle}
\newcommand{\degree}[1]{\np[\tcdegree]{#1}}
\renewcommand{\mod}[1]{\ensuremath{|#1|}}
\newcommand{\versi}{\hat{\imath}}
\newcommand{\versj}{\hat{\jmath}}
\newcommand{\versk}{\hat{k}}
\newcommand{\fat}{f_{at}}
\newcommand{\vecfat}{\vec{f}_{at}}
\newcommand{\comment}[1]{\marginnote{\footnotesize\textbf{\texttt{#1}}}}

%\def\mathnote#1{%
%  \stepcounter{equation}\tag{\theequation{\rlap{\hspace\marginparsep\smash{\parbox[c]{\marginparwidth}{%
%  \footnotesize\emph{\textbf{#1}}}}}}}
%}

\newcommand{\mathnote}[2][0pt]{\marginnote{\footnotesize\emph{\textbf{#2}}}[#1]}

\newcommand*\keystroke[1]{%
  \tikz[baseline=(key.base)]
    \node[%
      draw,
      fill=white,
      drop shadow={shadow xshift=0.25ex,shadow yshift=-0.25ex,fill=black,opacity=0.75},
      rectangle,
      rounded corners=1pt,
      inner sep=1pt,
      line width=0.5pt,
      font=\footnotesize\sffamily
    ](key) {#1\strut}
  ;
}

\newenvironment{system}%
{\left\lbrace\begin{aligned}}%
{\end{aligned}\right.}

% Adaptado de https://tex.stackexchange.com/questions/385097/tikz-calendar-is-it-possible-to-mark-half-a-day/385283
\makeatletter%
\tikzoption{day headings}{\tikzstyle{day heading}=[#1]}
\tikzstyle{day heading}=[]
\tikzstyle{day letter headings}=[
execute before day scope={ \ifdate{day of month=1}{%
  \pgfmathsetlength{\pgf@ya}{\tikz@lib@cal@yshift}%
  \pgfmathsetlength\pgf@xa{\tikz@lib@cal@xshift}%
  \pgftransformyshift{-\pgf@ya}
  \foreach \d/\l in {0/S,1/T,2/Q,3/Q,4/S,5/S,6/D} {
    \pgf@xa=\d\pgf@xa%
    \pgftransformxshift{\pgf@xa}%
    \pgftransformyshift{\pgf@ya}%
    \node[every day,day heading]{\l};%
  } 
}{}%
}%
]

\makeatother%

% Obtido de https://tex.stackexchange.com/questions/40624/howto-typeset-the-direct-current-symbol-in-latex
\newcommand{\textdirectcurrent}{%
  \settowidth{\dimen0}{$=$}%
  \vbox to .85ex {\offinterlineskip
    \hbox to \dimen0{\leaders\hrule\hfill}
    \vskip.35ex
    \hbox to \dimen0{%
      \leaders\hrule\hskip.2\dimen0\hfill
      \leaders\hrule\hskip.2\dimen0\hfill
      \leaders\hrule\hskip.2\dimen0
    }
    \vfill
  }%
}
\newcommand{\mathdirectcurrent}{\mathrel{\textdirectcurrent}}

